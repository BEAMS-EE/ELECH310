\documentclass[11pt,a4paper,dvipsnames]{article}
\usepackage[utf8]{inputenc}
\usepackage[T1]{fontenc}
\usepackage{amsthm} %numéroter les questions
\usepackage[frenchb]{babel}
\usepackage{datetime}
\usepackage{xspace} % typographie IN
\usepackage{hyperref}% hyperliens
\usepackage[all]{hypcap} %lien pointe en haut des figures
\usepackage[french]{varioref} %voir x p y
\usepackage{fancyhdr}% en têtes
%\input cyracc.def
\usepackage{graphicx} %include pictures
\usepackage{pgfplots}

\usepackage{tikz}
\usetikzlibrary{calc}
\usetikzlibrary{babel}
\usepackage{circuitikz}
% \usepackage{gnuplottex}
\usepackage{float}
\usepackage{ifthen}

\usepackage[top=1.3 in, bottom=1.3 in, left=1.3 in, right=1.3 in]{geometry} % Yeah, that's bad to play with margins
\usepackage[]{pdfpages}

\usepackage[]{attachfile}

\usepackage{amsmath}

\usepackage{askmaps}
% \usepackage[usenames,dvipsnames,svgnames,table]{xcolor}
\usepackage[]{xcolor}

\newdateformat{mydate}{2016--2017}%hack pour remplacer \THEYEAR

%cyr
\newcommand\textcyr[1]{{\fontencoding{OT2}\fontfamily{wncyr}\selectfont #1}}


\newboolean{corrige}
\ifx\correction\undefined
\setboolean{corrige}{false}% pas de corrigé
\else
\setboolean{corrige}{true}%corrigé
\fi

%\setboolean{corrige}{false}% pas de corrigé

\newboolean{annexes}
\setboolean{annexes}{true}%annexes
%\setboolean{annexes}{false}% pas de annexes

\definecolor{darkblue}{rgb}{0,0,0.5}

\newboolean{mos}
%\setboolean{mos}{true}%annexes
\setboolean{mos}{false}% pas de annexes

\usepackage{aeguill} %guillemets

%% fancy header & foot
\pagestyle{fancy}
\lhead{[ELEC-H-310] Électronique numérique\\ TP 2}
\rhead{\mydate\today\\ page \thepage}
\chead{\ifthenelse{\boolean{corrige}}{Corrigé}{}}
\cfoot{}
%%

\pdfinfo{
/Author (Quentin Delhaye, Ken Hasselmann, ULB -- BEAMS)
/Title (TP 2, ELEC-H-310)
/ModDate (D:\pdfdate)
}

\hypersetup{
pdftitle={TP 2 [ELEC-H-310] Choucroute numérique},
pdfauthor={Quentin Delhaye, Ken Hasselmann, ©2014 ULB -- BEAMS  },
pdfsubject={}
}

\theoremstyle{definition}% questions pas en italique
\newtheorem{Q}{Question}[] % numéroter les questions [section] ou non []

\newcommand{\reponse}[1]{% pour intégrer une réponse : \reponse{texte} : sera inclus si \boolean{corrige}
	\ifthenelse {\boolean{corrige}} {\paragraph{Réponse :} \color{darkblue}   #1\color{black}} {}
 }

\newcommand{\addcontentslinenono}[4]{\addtocontents{#1}{\protect\contentsline{#2}{#3}{#4}{}}}

\date{\vspace{-1.7cm}\mydate\today}
\title{\vspace{-2cm} TP 2\\ Électronique numérique [ELEC-H-310] \ifthenelse{\boolean{corrige}}{~\\Corrigé}{}}

\setlength{\parskip}{0.2cm plus2mm minus1mm} %espacement entre §
\setlength{\parindent}{0pt}

% \renewcommand{\theenumi}{\alph{enumi}} % Change the 'enumerate' format to use letters.
\usepackage{enumitem}
\setlist[enumerate]{label=\alph*)}% If you want only the x-th level to use this format, use '[enumerate,x]'

\begin{document}
\pagestyle{empty}
\maketitle
\vspace*{1cm}

\textit{Note~:} On adopte la convention de notation suivante~: $\overline{ab} = \overline{a} \cdot \overline{b}$ et $\overline{(ab)} = \overline{a \cdot b} = \overline{a} + \overline{b}$
\begin{Q}
	Prouver, par comparaison des tables de vérité, l'égalité suivante~:
	\[\overline{a}c + \overline{abc} = \overline{ab} + \overline{a}c\]
	\label{Q:1}
	\reponse{On s'en convainc en voyant que $F1 = F2$

		\renewcommand{\arraystretch}{1.2}
		\begin{center}
			\begin{tabular}{|l|l|l|l|l|l|l|l|l|} \hline
				$a$ & $b$ & $c$ & $\overline{a}c$ & $\overline{abc}$ & $F1$ & $\overline{ab}$ & $\overline{a}c$ & $F2$ \\ \hline
				0 & 0 & 0 & 0 & 1 & 1 & 1 & 0 & 1 \\ \hline
				0 & 0 & 1 & 1 & 0 & 1 & 1 & 1 & 1 \\ \hline
				0 & 1 & 0 & 0 & 0 & 0 & 0 & 0 & 0 \\ \hline
				0 & 1 & 1 & 1 & 0 & 1 & 0 & 1 & 1 \\ \hline
				1 & 0 & 0 & 0 & 0 & 0 & 0 & 0 & 0 \\ \hline
				1 & 0 & 1 & 0 & 0 & 0 & 0 & 0 & 0 \\ \hline
				1 & 1 & 0 & 0 & 0 & 0 & 0 & 0 & 0 \\ \hline
				1 & 1 & 1 & 0 & 0 & 0 & 0 & 0 & 0 \\ \hline
			\end{tabular}
		\end{center}
	}%R
\end{Q}


\begin{Q}
	Simplifier par manipulation algébrique les expressions suivantes~:

	\begin{enumerate}
		\item $(a+b) \cdot (a+\overline{b})$
		\reponse{
			\begin{align*}
				(a+b) \cdot (a+\overline{b}) & = aa + a\overline{b} + ab + b\overline{b}\\
				& = a + a\overline{b} + ab + 0 \\
				& = a \cdot (1 + \overline{b} + b) \\
				& = a
			\end{align*}
		}
		\item $a+\overline{a}b$
		\reponse{
			\begin{align*}
				a+\overline{a}b & = (a+\overline{a}) \cdot (a+b) \\
				& = 1 \cdot (a+b) \\
				& = a + b
			\end{align*}
		}
		\item $\overline{ab}c + \overline{abc} + \overline{a}b\overline{c}$
		\reponse{
			\begin{align*}
				\overline{ab}c + \overline{abc} + \overline{a}b\overline{c} & = \overline{ab} \cdot (c + \overline{c}) + \overline{a}b\overline{c} \\
				& = \overline{ab} + \overline{a}b\overline{c} \\
				& = \overline{a} \cdot (\overline{b} + b\overline{c}) \\
				& = \overline{a} \cdot ((\overline{b} + b) \cdot (\overline{b} + \overline{c})) \\
				& = \overline{a} \cdot ( 1 \cdot (\overline{b} + \overline{c})) \\
				& = \overline{a} \cdot (\overline{b} + \overline{c})
			\end{align*}
		}
		\item $\overline{((a+b)\overline{cd} + e + \overline{f})}$
		\reponse{
			\begin{align*}
				\overline{((a+b) \cdot \overline{cd} + e + \overline{f})} & = (\overline{ab} + c + d) \cdot \overline{e}f\quad \mbox{(De Morgan)}
			\end{align*}
		}
		\item $\overline{a}bc + a\overline{bc} + \overline{abc} + a\overline{b}c + abc$
		\reponse{
			\begin{align*}
				\overline{a}bc + a\overline{bc} + \overline{abc} + a\overline{b}c + abc & = bc \cdot (a+\overline{a}) + a\overline{b}c + \overline{bc} \cdot (a+\overline{a}) \\
				& = bc + a\overline{b}c + \overline{bc}
			\end{align*}
		}
		\item $\overline{(ab + ac)} + \overline{ab}c$
		\reponse{
			\begin{align*}
				\overline{(ab + ac)} + \overline{ab}c & = (\overline{a} + \overline{b}) \cdot (\overline{a} + \overline{c}) + \overline{ab}c \\
				& = \overline{a} + \overline{ac} + \overline{ab} + \overline{bc} + \overline{ab}c \\
				& = \overline{a} \cdot (1 + \overline{c} + \overline{b} + \overline{b}c) + \overline{bc} \\
				& = \overline{a} + \overline{bc}
			\end{align*}
		}
		\item $\overline{(a + b)}\ \overline{(\overline{a} + b)}$
		\reponse{
			\begin{align*}
				\overline{(a + b)} \cdot \overline{(\overline{a} + b)} & = (\overline{ab}) \cdot (a\overline{b}) \\
				& = \overline{a} \cdot a \cdot \overline{b} \\
				& = 0
			\end{align*}
		}
		\item $a + \overline{a}b + \overline{ab}$
		\reponse{
			\begin{align*}
				a + \overline{a}b + \overline{ab} & = a + \overline{a} \cdot (b+\overline{b}) \\
				& = a + \overline{a} \\
				& = 1
			\end{align*}
		}
	\end{enumerate}
	\reponse{}%R
\end{Q}


\begin{Q}
	Écrire les expressions logiques suivantes sous forme de minterms (forme disjonctive normale)~:
	\begin{enumerate}
		\item $F(a,b,c,d) = a\overline{b}c + \overline{ab} + ab\overline{c}d$
		\reponse{
			\begin{multline*}
				a\overline{b}cd + a\overline{b}c\overline{d} + \overline{ab}cd + \overline{ab}c\overline{d} + \overline{abc}d + \overline{abcd} + ab\overline{c}d
			\end{multline*}
		}
		\item $F(a,b,c,d) = ab + \overline{b}c + cd$
		\reponse{
			\begin{multline*}
				% & abcd + abc\overline{d} + ab\overline{c}d + ab\overline{cd} + a\overline{b}cd + a\overline{b}c\overline{d} + \\
				% & \overline{ab}cd + \overline{ab}c\overline{d} + abcd + a\overline{b}cd + \overline{a}bcd + \overline{ab}cd
				abcd + abc\overline{d} + ab\overline{c}d + ab\overline{cd} + a\overline{b}cd + a\overline{b}c\overline{d} + \overline{ab}cd + \overline{ab}c\overline{d} + \overline{a}bcd
			\end{multline*}
		}
		\item $F(a,b,c,d) = a + d$
		\reponse{
			\begin{multline*}
				abcd + abc\overline{d} + ab\overline{c}d + ab\overline{cd} + a\overline{b}cd + a\overline{b}c\overline{d} + a\overline{bc}d + a\overline{bcd} + \overline{a}bcd + \overline{a}b\overline{c}d + \overline{ab}cd + \overline{abc}d
			\end{multline*}
		}
	\end{enumerate}
	\reponse{}%R
\end{Q}


\begin{Q}
	Simplifier $F(a,b)$ à l'aide des K-maps (rappel~: afin de remplir la table de Karnaugh, on peut développer la fonction logique dans l'une de ses formes canoniques.)~:
	\begin{enumerate}
		\item $F(a,b) = a + \overline{a}b + \overline{ab}$
		\reponse{
			\begin{center}
				\askmapii{$F=1$}{ab}{}{1111}{}
			\end{center}
		}
		\item $F(a,b) = (a + b) \cdot (a + \overline{b})$
		\reponse{
			\begin{center}
				\askmapii{$F=a$}{a{\raisebox{2ex}{\ \ b}}}{}{0011}{% raise Z input
				\color{red}\put(1.1,0.1){\dashbox{0.1}(0.8,1.8){}}%
				}
			\end{center}
		}
		\item $F(a,b) = a + \overline{a}b$
		\reponse{
			\begin{center}
				\askmapii{$F=a+b$}{a{\raisebox{2ex}{\ \ b}}}{}{0111}{% raise Z input
				\color{red}\put(0.1,0.1){\dashbox{0.1}(1.8,0.8){}}%
				\color{green}\put(1.15,0.15){\dashbox{0.2}(0.7,1.7){}}%
				}
			\end{center}
		}
	\end{enumerate}
	\reponse{}%R
\end{Q}


\begin{Q}
	Simplifier $F(a,b,c)$ à l'aide des K-maps~:
	\begin{enumerate}
		\item $F(a,b,c) = \overline{a}c + \overline{abc}$
		\reponse{%TODO Mettre les termes dans la même couleur que les cadres dans les K-maps.
			\begin{center}
				\askmapiii{$F=\overline{ab}+\overline{a}c$}{bca}{}{10100010}{% raise Z input
				\color{red}\put(0.1,1.1){\dashbox{0.1}(1.8,0.8){}}%
				\color{green}\put(1.15,1.15){\dashbox{0.2}(1.7,0.7){}}%
				}
			\end{center}
		}
		\item $F(a,b,c) = a\overline{b}c + \overline{a}b\overline{c} + \overline{a}bc + \overline{ab}c$
		\reponse{
			\begin{center}
				\askmapiii{$F=\overline{a}b+\overline{b}c$}{bca}{}{00111010}{% raise Z input
				\color{red}\put(1.1,0.1){\dashbox{0.1}(0.8,1.8){}}%
				\color{green}\put(2.1,1.1){\dashbox{0.2}(1.8,0.8){}}%
				}
			\end{center}
		}
		\item $F(a,b,c) = ab\overline{c} + \overline{abc} + \overline{a}b\overline{c} + a\overline{bc}$
		\reponse{
			\begin{center}
				\askmapiii{$F=\overline{c}$}{bca}{}{11001100}{% raise Z input
				\color{red}\put(0.1,0.1){\dashbox{0.1}(0.8,1.8){}}%
				\color{red}\put(3.1,0.1){\dashbox{0.1}(0.8,1.8){}}%
				}
			\end{center}
		}
	\end{enumerate}
	\reponse{}%
\end{Q}


\begin{Q}
	Simplifier $F(a,b,c,d)$ à l'aide des K-maps~:
	\begin{enumerate}
		\item $F(a,b,c,d) = abd + acd + bcd + ab + \overline{a}cd + \overline{ab}cd$
		\reponse{
			\begin{center}
				\askmapiv{$F=ab+cd$}{cdab}{}{0001000100011111}{% raise Z input
				\color{red}\put(0.1,1.1){\dashbox{0.1}(3.8,0.8){}}%
				\color{green}\put(2.1,0.1){\dashbox{0.2}(0.8,3.8){}}%
				}
			\end{center}
		}
		\item $F(a,b,c,d) = \overline{abcd} + \overline{ac}d + \overline{a}b\overline{c} + abc + a\overline{b}c + abcd$
		\reponse{
			\begin{center}
				\askmapiv{$F=ac+\overline{ac}$}{cdab}{}{1100110000110011}{% raise Z input
				\color{red}\put(0.1,2.1){\dashbox{0.1}(1.8,1.8){}}%
				\color{green}\put(2.1,0.1){\dashbox{0.2}(1.8,1.8){}}%
				}
			\end{center}
		}
		\item $F(a,b,c,d) = \overline{bcd} + \overline{a}c\overline{d} + \overline{ac}d + a\overline{d} + \overline{a}b\overline{d}$
		\reponse{
			\begin{center}
				\askmapiv{$F=\overline{d}+\overline{ac}$}{cdab}{}{1111110011110000}{% raise Z input
				\color{red}\put(0.1,0.1){\dashbox{0.1}(0.8,3.8){}}%
				\color{green}\put(0.15,2.15){\dashbox{0.2}(1.7,1.7){}}%
				\color{red}\put(3.1,0.1){\dashbox{0.1}(0.8,3.8){}}%
				}
			\end{center}
		}
	\end{enumerate}
\end{Q}

\begin{Q}
	Simplifier $F(a,b,c,d,e)$ à l'aide des K-maps~:
	\begin{enumerate}
		\item $F(a,b,c,d,e) = a\overline{e} + b\overline{e} + a\overline{b}ce + a\overline{b}cde + a\overline{b}c\overline{e} + \overline{acde} + \overline{abe} + \overline{ab}ce$
		\reponse{
			\begin{center}
				\askmapv{$F = \overline{e} + \overline{b}c$}{abcde}{}{10101111101010101010111110101010}{%
				\color{red}\put(0.1,3.1){\dashbox{0.1}(7.8,0.8){}}%
				\color{red}\put(0.1,0.1){\dashbox{0.1}(7.8,0.8){}}%
				\color{green}\put(1.1,0.2){\dashbox{0.2}(0.8,3.6){}}%
				\color{green}\put(5.1,0.2){\dashbox{0.2}(0.8,3.6){}}%
				}
			\end{center}
		}
	\end{enumerate}
\end{Q}

\end{document}
