\documentclass[11pt,a4paper,dvipsnames]{article}
\usepackage[utf8]{inputenc}
\usepackage[T1]{fontenc}
\usepackage{amsthm} %numéroter les questions
\usepackage[frenchb]{babel}
\usepackage{datetime}
\usepackage{xspace} % typographie IN
\usepackage{hyperref}% hyperliens
\usepackage[all]{hypcap} %lien pointe en haut des figures
\usepackage[french]{varioref} %voir x p y
\usepackage{fancyhdr}% en têtes
%\input cyracc.def
\usepackage{graphicx} %include pictures
\usepackage{pgfplots}

\usepackage{tikz}
\usetikzlibrary{calc,arrows,automata}
\usetikzlibrary{babel}
\usepackage{circuitikz}
% \usepackage{gnuplottex}
\usepackage{float}
\usepackage{ifthen}
\usepackage{pstricks}

\usepackage[top=1.3 in, bottom=1.3 in, left=1.3 in, right=1.3 in]{geometry} % Yeah, that's bad to play with margins
\usepackage[]{pdfpages}
\usepackage[]{attachfile}

\usepackage{amsmath}
\usepackage{amsfonts}
\usepackage{amssymb}
\usepackage{askmaps}
\usepackage{enumitem}
\setlist[enumerate]{label=\alph*)}% If you want only the x-th level to use this format, use '[enumerate,x]'
\usepackage{multirow}
\usepackage{bigdelim}%Braces in tabular

\newdateformat{mydate}{2016--2017}%hack pour remplacer \THEYEAR

%Numero du TP :
\def \tpnumber {TP 3 }

%corrigé ou non ?
\newboolean{corrige}
\ifx\correction\undefined
\setboolean{corrige}{false}% pas de corrigé
\else
\setboolean{corrige}{true}%corrigé
\fi

%\setboolean{corrige}{true}%  corrigé

\newboolean{annexes}
\setboolean{annexes}{true}%annexes
%\setboolean{annexes}{false}% pas de annexes

\definecolor{darkblue}{rgb}{0,0,0.5}

\newboolean{mos}
%\setboolean{mos}{true}%annexes
\setboolean{mos}{false}% pas de annexes

\usepackage{aeguill} %guillemets

%% fancy header & foot
\pagestyle{fancy}
\lhead{[ELEC-H-310] Électronique numérique\\ \tpnumber }
\rhead{\mydate\today\\ page \thepage}
\chead{\ifthenelse{\boolean{corrige}}{Corrigé}{}}
\cfoot{}
%%

\pdfinfo{
/Author (Quentin Delhaye, Ken Hasselmann, ULB -- BEAMS)
/Title (\tpnumber, ELEC-H-310)
/ModDate (D:\pdfdate)
}

\hypersetup{
pdftitle={\tpnumber [ELEC-H-310] Électronique numérique},
pdfauthor={Quentin Delhaye, Ken Hasselmann, ©2014 ULB -- BEAMS  },
pdfsubject={}
}

\theoremstyle{definition}% questions pas en italique
\newtheorem{Q}{Question}[] % numéroter les questions [section] ou non []

\newcommand{\reponse}[1]{% pour intégrer une réponse : \reponse{texte} : sera inclus si \boolean{corrige}
	\ifthenelse {\boolean{corrige}} {\paragraph{Réponse :} \color{darkblue}   #1\color{black}~\\} {}
 }

\newcommand{\addcontentslinenono}[4]{\addtocontents{#1}{\protect\contentsline{#2}{#3}{#4}{}}}

\date{\vspace{-1.7cm}\mydate\today}
\title{\vspace{-2cm} \tpnumber \\ Électronique numérique [ELEC-H-310] \ifthenelse{\boolean{corrige}}{~\\Corrigé}{}}

\setlength{\parskip}{0.2cm plus2mm minus1mm} %espacement entre §
\setlength{\parindent}{0pt}

\begin{document}
\pagestyle{empty}
\maketitle
\vspace*{-1cm}

%Debut du TP
\begin{Q}
Trouver les implicants premiers par la méthode de Quine - Mc Cluskey :

\textit{Note~: $\sum_m$ désigne une somme de mintermes, $\sum_d$ une somme de « don't care ».}

\begin{enumerate}
%	\item $f(a,b,c,d)=\sum_m(2,4,8,9,10,11,13,15)$
	\item $f(a,b,c,d)=\sum_m(0,4,5,8,12,13)$
	\reponse{
		Il convient d'abord de convertir les nombres décimaux dans leur forme binaire~:
		\begin{center}
			\begin{tabular}{c|cccc}
			 & a & b & c & d \\ \hline
			 0 & 0 & 0 & 0 & 0 \\
			 4 & 0 & 1 & 0 & 0 \\
			 5 & 0 & 1 & 0 & 1 \\
			 8 & 1 & 0 & 0 & 0 \\
			 12 & 1 & 1 & 0 & 0 \\
			 13 & 1 & 1 & 0 & 1 \\
			\end{tabular}
		\end{center}

		On peut ensuite établir les différents groupes et fusionner ceux-ci.
		La marque $\checkmark$ indique qu'une ligne a été regroupée avec une autre.
		Les nombres entre parenthèses indiquent quels nombres ont été fusionnés.

		\begin{minipage}{0.5\textwidth}
			\begin{tabular}{ccccccc}
			 & a & b & c & d & & \\
			 $G_0$ & 0 & 0 & 0 & 0 & 0 & $\checkmark$ \\ \hline
			 $G_1$ & 0 & 1 & 0 & 0 & 4 & $\checkmark$ \\
			 & 1 & 0 & 0 & 0 & 8 & $\checkmark$ \\ \hline
			 $G_2$ & 1 & 1 & 0 & 0 & 12 & $\checkmark$ \\
			 & 0 & 1 & 0 & 1 & 5 & $\checkmark$ \\ \hline
			 $G_3$ & 1 & 1 & 0 & 1 & 13 & $\checkmark$ \\
			\end{tabular}
		\end{minipage}%
		\begin{minipage}{0.5\textwidth}
			\begin{tabular}{ccccccc}
			 & a & b & c & d & & \\
			 $G_{0'}$ & 0 & -- & 0 & 0 & (0,4) & $\checkmark$ \\
			 & -- & 0 & 0 & 0 & (0,8) & $\checkmark$ \\ \hline
			 $G_{1'}$ & -- & 1 & 0 & 0 & (4,12) & $\checkmark$ \\
			 & 0 & 1 & 0 & -- & (4,5) & $\checkmark$ \\
			 & 1 & -- & 0 & 0 & (8,12) & $\checkmark$ \\ \hline
			 $G_{2'}$ & 1 & 1 & 0 & -- & (12,13) & $\checkmark$ \\
			 & -- & 1 & 0 & 1 & (5,13) & $\checkmark$ \\
			\end{tabular}
		\end{minipage}
		\begin{center}
			\begin{tabular}{cccccccc}
			$G_{0''}$ & -- & -- & 0 & 0 & (0,4|8,12) & Fusion & \rdelim\}{2}{10mm}[IP1] \\
			& -- & -- & 0 & 0 & (0,8|4,12) & Fusion \\ \hline
			$G_{1''}$ & -- & 1 & 0 & -- & (4,12|5,13) & Fusion & \rdelim\}{2}{10mm}[IP2] \\
			& -- & 1 & 0 & -- & (4,5|12,13) & Fusion \\
			\end{tabular}
		\end{center}

		Les implicants premiers sont donc $\overline{cd}$ (IP1) et $b\overline{c}$ (IP2).

	}
	\item $f(a,b,c,d)=\sum_m(2,3,4,10,12,13)+\sum_d(11,14,15)$
	\reponse{
		$f(a,b,c,d)=\overline{b}c+b\overline{c}\overline{d}+ac+ab$
	}
%	\item $f(a,b,c,d,e)=\sum_m(0,1,2,4,11,15,17,20,21,31)+\sum_d(5,6,16,18,22,27)$
	\item $f(a,b,c,d,e,f)=\sum_m(16,28,53,60,63)$
	\reponse{
		$f(a,b,c,d,e,f)=abcdef+bcd\overline{ef}+ab\overline{c}d\overline{e}f+\overline{a}b\overline{cdef}$
	}
%	\item $f(a,b,c,d,e,f)=\sum_m(1,2,3,15,17,18,19,26,32,48,63)$
\end{enumerate}
\end{Q}














\begin{Q}
Trouver la fonction simplifiée :
\begin{enumerate}
	\item $f(a,b,c,d)=\sum_m(2,3,4,10,12,13)+\sum_d(11,14,15)$
	\reponse{
		Tableau de couverture avec les implicants premiers de l'exercice précédent :
		$i_1$ = $b\overline{cd}$, $i_2$ = $\overline{b}c$, $i_3$ = $ac$ et $i_4$ = $ab$.

		\begin{center}
			\begin{tabular}{ccccc}
			 & $i_1$ & $i_2$ & $i_3$ & $i_4$ \\ \hline
			 0010 & & $\checkmark$ & & \\ \hline
			 0011 & & $\checkmark$ & & \\ \hline
			 0100 & $\checkmark$ & & & \\ \hline
			 1010 & & $\checkmark$ & $\checkmark$ & \\ \hline
			 1100 & $\checkmark$ &  &  & $\checkmark$ \\ \hline
			 1101 & & & & $\checkmark$ \\ \hline
			 % 1011 &  & $\checkmark$ & $\checkmark$ & \\ \hline
			 % 1110 & & & $\checkmark$ & $\checkmark$ \\ \hline
			 % 1111 & & & $\checkmark$ & $\checkmark$ \\
			\end{tabular}
		\end{center}

		On peut en déduire l'équation de couverture suivante~:
		\[1 = i_2 \cdot i_1 \cdot (i_2 + i_3) \cdot (i_1 + i_4) \cdot i_4 = i_2i_1i_4\]
		En utilisant les axiomes et théorèmes suivants~:
		\begin{align*}
		& x \cdot x = x\\
		& x \cdot (x + y) = x\\
		& (x + y) \cdot (x + z) = x + y \cdot z
		\end{align*}

		On trouve ainsi $f(a,b,c,d)=\overline{b}c+b\overline{cd}+ab$
	}

	\item $f(a,b,c,d)=\sum_m(0,2,4,5,10,11,13,15)+\sum_d(6,8)$
	\reponse {
		$f(a,b,c,d)=\overline{a}b\overline{c}+\overline{b}\overline{d}+a\overline{b}c+abd$
	ou $f(a,b,c,d)=a\overline{b}c+b\overline{c}d+abd+\overline{ad}$
	ou $f(a,b,c,d)=\overline{a}b\overline{c} + a\overline{b}c + abd + \overline{a}\overline{d}$
	ou $f(a,b,c,d)=b\overline{c}d + a\overline{b}c + acd + \overline{a}\overline{d}$
	ou $f(a,b,c,d)=b\overline{c}d + acd + \overline{a}\overline{d} + \overline{b}\overline{d}$
	ou $f(a,b,c,d)=\overline{a}b\overline{c} + acd + abd + \overline{b}\overline{d}$
	ou $f(a,b,c,d)=\overline{a}b\overline{c} + b\overline{c}d + acd + \overline{b}\overline{d}$
	}
%	\item $f(a,b,c,d)=\sum_m(1,5,6,9,11,12,14,15)+ \sum_d(2,3,4,13)$

\end{enumerate}

\end{Q}


















\begin{Q}
Pour les fonctions proposées dessiner les K-maps, optimiser les fonctions et prévoir les termes redondants (problème des aléas) :
	\begin{enumerate}
		\item $f(a,b,c,d)=\sum_m(0,1,2,6,8,9,10,14)$
		\reponse{
			\begin{center}
				\askmapiv{$F=c\overline{d} + \overline{bc} + \overline{bd}$}{c d a b}{}{1010101011110000}{% raise Z input
				\color{red}\put(0.1,0.1){\dashbox{0.1}(1.8,0.8){}}%
				\color{red}\put(0.1,3.1){\dashbox{0.1}(1.8,0.8){}}%
				\color{green}\put(3.1,0.1){\dashbox{0.1}(0.8,3.8){}}%
				\color{blue}\put(0.2,0.2){\dashbox{0.2}(0.6,0.6){}}%
				\color{blue}\put(0.2,3.2){\dashbox{0.2}(0.6,0.6){}}%
				\color{blue}\put(3.2,0.2){\dashbox{0.2}(0.6,0.6){}}%
				\color{blue}\put(3.2,3.2){\dashbox{0.2}(0.6,0.6){}}%
				}
			\end{center}
		}
		\item $f(a,b,c,d)=\sum_m(1,3,5,7,8,9,12,13)$
		\reponse{
			\begin{center}
				\askmapiv{$F=\overline{a}d + a\overline{c} + \overline{c}d$}{c d a b}{}{0011111100001100}{% raise Z input
				\color{red}\put(0.1,0.1){\dashbox{0.1}(1.8,1.8){}}%
				\color{green}\put(1.1,2.1){\dashbox{0.1}(1.8,1.8){}}%
				\color{blue}\put(1.2,0.2){\dashbox{0.2}(0.6,3.6){}}%
				}
			\end{center}
		}
	\end{enumerate}

\end{Q}




















\begin{Q}
Dresser la table de Huffman pour le graphe suivant :

\begin{center}
\begin{tikzpicture}[->,>=stealth',shorten >=1pt,auto,node distance=3.5cm,semithick,align=center]
  \tikzstyle{every state}=[fill=white,text=black]

  \node[state]         (A)              {$A$\\ Z=0};
  \node[state]         (B) [right of=A] {$B$\\ Z=0};
  \node[state]         (D) [below of=A] {$D$\\ Z=0};
  \node[state]         (C) [below of=B] {$C$\\ Z=1};

  \path (A) edge [bend left]  node {ab=01} (B)
            edge [bend right]  node {11} (D)
			edge [loop left]  node {00,10} (A)
        (B) edge [loop right] node {01,11} (B)
            edge [bend left]  node {00} (A)
			edge [bend left]  node {10} (C)
        (C) edge [bend left]  node {11} (D)
		    edge [loop right] node {10,00,01} (C)
        (D) edge [loop left]  node {11,00} (D)
            edge              node {01} (B)
			edge [bend left]  node {10} (C);
\end{tikzpicture}
\end{center}

\reponse{
\begin{center}
\begin{tabular}{lllllll}
\hline
\multicolumn{1}{|l|}{} & \multicolumn{1}{l|}{00}          & \multicolumn{1}{l|}{01}          & \multicolumn{1}{l|}{11}          & \multicolumn{1}{l|}{10}          & \multicolumn{1}{l|}{ab} & \multicolumn{1}{l|}{Z} \\ \hline
\multicolumn{1}{|l|}{A}      & \multicolumn{1}{l|}{\textbf{A}} & \multicolumn{1}{l|}{B}          & \multicolumn{1}{l|}{D}          & \multicolumn{1}{l|}{\textbf{A}} & \multicolumn{1}{l|}{}   & \multicolumn{1}{l|}{0} \\ \cline{1-5} \cline{7-7}
\multicolumn{1}{|l|}{B}      & \multicolumn{1}{l|}{A}          & \multicolumn{1}{l|}{\textbf{B}} & \multicolumn{1}{l|}{\textbf{B}}          & \multicolumn{1}{l|}{C}          & \multicolumn{1}{l|}{}   & \multicolumn{1}{l|}{0} \\ \cline{1-5} \cline{7-7}
\multicolumn{1}{|l|}{C}      & \multicolumn{1}{l|}{\textbf{C}}          & \multicolumn{1}{l|}{\textbf{C}}          & \multicolumn{1}{l|}{D} & \multicolumn{1}{l|}{\textbf{C}} & \multicolumn{1}{l|}{}   & \multicolumn{1}{l|}{1} \\ \cline{1-5} \cline{7-7}
\multicolumn{1}{|l|}{D}      & \multicolumn{1}{l|}{\textbf{D}}          & \multicolumn{1}{l|}{B} & \multicolumn{1}{l|}{\textbf{D}} & \multicolumn{1}{l|}{C}          & \multicolumn{1}{l|}{}   & \multicolumn{1}{l|}{0} \\ \cline{1-5} \cline{7-7}
                      &                                  &                                  &                                  &                                  &                         &
\end{tabular}
\end{center}
}
\end{Q}






















\begin{Q}
À partir de cette table de Huffman codée trouver les équations et le graphe d'états correspondant :

\begin{center}
\begin{tabular}{lllllll}
\hline
\multicolumn{1}{|l|}{$Y_1$ $Y_2$} & \multicolumn{1}{l|}{00}          & \multicolumn{1}{l|}{01}          & \multicolumn{1}{l|}{11}          & \multicolumn{1}{l|}{10}          & \multicolumn{1}{l|}{ab} & \multicolumn{1}{l|}{Z} \\ \hline
\multicolumn{1}{|l|}{00}      & \multicolumn{1}{l|}{\textbf{00}} & \multicolumn{1}{l|}{01}          & \multicolumn{1}{l|}{01}          & \multicolumn{1}{l|}{\textbf{00}} & \multicolumn{1}{l|}{}   & \multicolumn{1}{l|}{0} \\ \cline{1-5} \cline{7-7}
\multicolumn{1}{|l|}{01}      & \multicolumn{1}{l|}{00}          & \multicolumn{1}{l|}{\textbf{01}} & \multicolumn{1}{l|}{11}          & \multicolumn{1}{l|}{11}          & \multicolumn{1}{l|}{}   & \multicolumn{1}{l|}{1} \\ \cline{1-5} \cline{7-7}
\multicolumn{1}{|l|}{11}      & \multicolumn{1}{l|}{\textbf{11}}          & \multicolumn{1}{l|}{10}          & \multicolumn{1}{l|}{\textbf{11}} & \multicolumn{1}{l|}{\textbf{11}} & \multicolumn{1}{l|}{}   & \multicolumn{1}{l|}{1} \\ \cline{1-5} \cline{7-7}
\multicolumn{1}{|l|}{10}      & \multicolumn{1}{l|}{11}          & \multicolumn{1}{l|}{\textbf{10}} & \multicolumn{1}{l|}{\textbf{10}} & \multicolumn{1}{l|}{00}          & \multicolumn{1}{l|}{}   & \multicolumn{1}{l|}{1} \\ \cline{1-5} \cline{7-7}
$y_1$ $y_2$                      &                                  &                                  &                                  &                                  &                         &
\end{tabular}
\end{center}

\reponse{
	\begin{center}
	\begin{tikzpicture}[->,>=stealth',shorten >=1pt,auto,node distance=3.5cm,semithick,align=center]
	  \tikzstyle{every state}=[fill=white,text=black]

	  \node[state]         (A)              {$00$\\ Z=0};
	  \node[state]         (B) [right of=A] {$01$\\ Z=1};
	  \node[state]         (D) [below of=A] {$10$\\ Z=1};
	  \node[state]         (C) [below of=B] {$11$\\ Z=1};

	  \path (A) edge [bend left]  node {ab=01,11} (B)
	            % edge [bend right]  node {11} (D)
				edge [loop left]  node {00,10} (A)
	        (B) edge [loop right] node {01} (B)
	            edge [bend left]  node {00} (A)
				edge [bend left]  node {11,10} (C)
				edge [bend left]  node {00} (A)
	        (C) edge [bend left]  node {01} (D)
			    edge [loop right] node {10,00,11} (C)
	        (D) edge [loop left]  node {11,01} (D)
	            edge [bend left]  node {10} (A)
				edge [bend left]  node {00} (C);
	\end{tikzpicture}
	\end{center}

	Afin de déterminer l'équation correspondante, on peut dériver des tables de Karnaugh de la table de Huffman.

	\begin{center}
		\askmapiv{$Y_1=y_2a + y_1 \overline{a} + y_1b$}{a b $y_1$ $y_2$}{}{0011001101010111}{% raise Z input
		\color{red}\put(0.1,0.1){\dashbox{0.1}(1.8,1.8){}}%
		\color{green}\put(2.1,1.1){\dashbox{0.1}(1.8,1.8){}}%
		\color{blue}\put(1.1,0.2){\dashbox{0.2}(1.8,1.6){}}%
		}

		\askmapiv{$Y_2=y_1\overline{ab} + \overline{y_1}b + y_2a$}{a b $y_1$ $y_2$}{}{0011110001011101}{% raise Z input
		\color{red}\put(0.1,0.1){\dashbox{0.1}(0.8,1.8){}}%
		\color{green}\put(1.1,2.1){\dashbox{0.1}(1.8,1.8){}}%
		\color{blue}\put(2.1,1.1){\dashbox{0.2}(1.8,1.8){}}%
		}

		\askmapiv{$Z=y_2+y_1$}{a b $y_1$ $y_2$}{}{0-11-11101-1-111}{% raise Z input
		\color{red}\put(0.1,0.1){\dashbox{0.1}(3.8,1.8){}}%
		\color{blue}\put(0.2,1.1){\dashbox{0.2}(3.6,1.8){}}%
		}
	\end{center}
}
\end{Q}


%
% \begin{Q}
% 	L'approximation I(V) idéalisée avec seulement la tension de seuil est-elle une bonne approximation ?
% 	\label{Q:1}
% 	\reponse{Oui}%R
% \end{Q}


% \begin{figure}[H]
% 	\begin{center}
% 		\begin{circuitikz}\draw
% 			(0,0) node[anchor=east] {A} to [short,i>^=$I$] (1.5,0)
% 			(0,0) to [sDo, v=$V$] (2.5,0) node [anchor=west]{K}
% 		;\end{circuitikz}
% 	\end{center}
% \caption{Conventions électriques}
% \label{fig:zener_conv}
% \end{figure}

\end{document}
