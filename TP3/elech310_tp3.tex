\documentclass[11pt,a4paper,dvipsnames]{article}
\usepackage[utf8]{inputenc}
\usepackage[T1]{fontenc}
\usepackage{amsthm} %numéroter les questions
\usepackage[frenchb,english]{babel}
\usepackage{datetime}
\usepackage{xspace} % typographie IN
\usepackage{hyperref}% hyperliens
\usepackage[all]{hypcap} %lien pointe en haut des figures
\usepackage[french]{varioref} %voir x p y
\usepackage{fancyhdr}% en têtes
\usepackage{graphicx} %include pictures
\usepackage{pgfplots}

\usepackage{tikz}
\usetikzlibrary{calc,arrows,automata}
\usetikzlibrary{babel}
\usepackage{circuitikz}
% \usepackage{gnuplottex}
\usepackage{float}
\usepackage{ifthen}

\usepackage[top=1.3 in, bottom=1.3 in, left=1.3 in, right=1.3 in]{geometry}
\usepackage[]{pdfpages}
\usepackage[]{attachfile}

\usepackage{amsmath}
\usepackage{amsfonts}
\usepackage{amssymb}
\usepackage{enumitem}
\setlist[enumerate]{label=\alph*)}% If you want only the x-th level to use this format, use '[enumerate,x]'
\usepackage{multirow}
\usepackage{bigdelim}%Braces in tabular

\usepackage{aeguill} %guillemets
\usepackage{askmaps}


%%%%%%%%%%%%%%%%%%%%%%%%%%%%%%%%%%%%%%%%%%%%%%%%%%%%%%%%%%%%%%%%%%%%%%%%%%%%%%%%%%%%%%%%%%%
% READ THIS BEFORE CHANGING ANYTHING
%
%
% - TP number: can be changed using the command
%		\def \tpnumber {TP 1 }
%
% - Version: controlled by a new command:
%		\newcommand{\version}{v1.0.0}
%
% - Booleans: there are three booleans used in this document:
%	- 'corrige', controlled by defining the variable 'correction'
%	- 'fr', controlled by defining the variable 'french'
%	- 'en', controlled by defining the variable 'english'
% You can define those variables in a makefile using such a command:
% pdflatex -shell-escape -jobname="elech310_tp1_fr" "\def\french{} \documentclass[11pt,a4paper]{article}
\usepackage[utf8]{inputenc}
\usepackage[T1]{fontenc}
\usepackage{amsthm} %numéroter les questions
\usepackage[frenchb,english]{babel}
\usepackage{datetime}
\usepackage{xspace} % typographie IN
\usepackage{hyperref}% hyperliens
\usepackage[all]{hypcap} %lien pointe en haut des figures
\usepackage[french]{varioref} %voir x p y
\usepackage{fancyhdr}% en têtes
\usepackage{graphicx} %include pictures
\usepackage{pgfplots}

\usepackage{tikz}
\usetikzlibrary{calc}
\usetikzlibrary{babel}
\usepackage{circuitikz}
% \usepackage{gnuplottex}
\usepackage{float}
\usepackage{ifthen}

\usepackage[top=1.3 in, bottom=1.3 in, left=1.3 in, right=1.3 in]{geometry}
\usepackage[]{pdfpages}
\usepackage[]{attachfile}

\usepackage{amsmath}
\usepackage{enumitem}
\setlist[enumerate]{label=\alph*)}% If you want only the x-th level to use this format, use '[enumerate,x]'
\usepackage{multirow}

\usepackage{aeguill} %guillemets


%%%%%%%%%%%%%%%%%%%%%%%%%%%%%%%%%%%%%%%%%%%%%%%%%%%%%%%%%%%%%%%%%%%%%%%%%%%%%%%%%%%%%%%%%%%
% READ THIS BEFORE CHANGING ANYTHING
%
%
% - TP number: can be changed using the command
%		\def \tpnumber {TP 1 }
%
% - Version: controlled by a new command:
%		\newcommand{\version}{v1.0.0}
%
% - Booleans: there are three booleans used in this document:
%	- 'corrige', controlled by defining the variable 'correction'
%	- 'fr', controlled by defining the variable 'french'
%	- 'en', controlled by defining the variable 'english'
% You can define those variables in a makefile using such a command:
% pdflatex -shell-escape -jobname="elech310_tp1_fr" "\def\french{} \documentclass[11pt,a4paper]{article}
\usepackage[utf8]{inputenc}
\usepackage[T1]{fontenc}
\usepackage{amsthm} %numéroter les questions
\usepackage[frenchb,english]{babel}
\usepackage{datetime}
\usepackage{xspace} % typographie IN
\usepackage{hyperref}% hyperliens
\usepackage[all]{hypcap} %lien pointe en haut des figures
\usepackage[french]{varioref} %voir x p y
\usepackage{fancyhdr}% en têtes
\usepackage{graphicx} %include pictures
\usepackage{pgfplots}

\usepackage{tikz}
\usetikzlibrary{calc}
\usetikzlibrary{babel}
\usepackage{circuitikz}
% \usepackage{gnuplottex}
\usepackage{float}
\usepackage{ifthen}

\usepackage[top=1.3 in, bottom=1.3 in, left=1.3 in, right=1.3 in]{geometry}
\usepackage[]{pdfpages}
\usepackage[]{attachfile}

\usepackage{amsmath}
\usepackage{enumitem}
\setlist[enumerate]{label=\alph*)}% If you want only the x-th level to use this format, use '[enumerate,x]'
\usepackage{multirow}

\usepackage{aeguill} %guillemets


%%%%%%%%%%%%%%%%%%%%%%%%%%%%%%%%%%%%%%%%%%%%%%%%%%%%%%%%%%%%%%%%%%%%%%%%%%%%%%%%%%%%%%%%%%%
% READ THIS BEFORE CHANGING ANYTHING
%
%
% - TP number: can be changed using the command
%		\def \tpnumber {TP 1 }
%
% - Version: controlled by a new command:
%		\newcommand{\version}{v1.0.0}
%
% - Booleans: there are three booleans used in this document:
%	- 'corrige', controlled by defining the variable 'correction'
%	- 'fr', controlled by defining the variable 'french'
%	- 'en', controlled by defining the variable 'english'
% You can define those variables in a makefile using such a command:
% pdflatex -shell-escape -jobname="elech310_tp1_fr" "\def\french{} \documentclass[11pt,a4paper]{article}
\usepackage[utf8]{inputenc}
\usepackage[T1]{fontenc}
\usepackage{amsthm} %numéroter les questions
\usepackage[frenchb,english]{babel}
\usepackage{datetime}
\usepackage{xspace} % typographie IN
\usepackage{hyperref}% hyperliens
\usepackage[all]{hypcap} %lien pointe en haut des figures
\usepackage[french]{varioref} %voir x p y
\usepackage{fancyhdr}% en têtes
\usepackage{graphicx} %include pictures
\usepackage{pgfplots}

\usepackage{tikz}
\usetikzlibrary{calc}
\usetikzlibrary{babel}
\usepackage{circuitikz}
% \usepackage{gnuplottex}
\usepackage{float}
\usepackage{ifthen}

\usepackage[top=1.3 in, bottom=1.3 in, left=1.3 in, right=1.3 in]{geometry}
\usepackage[]{pdfpages}
\usepackage[]{attachfile}

\usepackage{amsmath}
\usepackage{enumitem}
\setlist[enumerate]{label=\alph*)}% If you want only the x-th level to use this format, use '[enumerate,x]'
\usepackage{multirow}

\usepackage{aeguill} %guillemets


%%%%%%%%%%%%%%%%%%%%%%%%%%%%%%%%%%%%%%%%%%%%%%%%%%%%%%%%%%%%%%%%%%%%%%%%%%%%%%%%%%%%%%%%%%%
% READ THIS BEFORE CHANGING ANYTHING
%
%
% - TP number: can be changed using the command
%		\def \tpnumber {TP 1 }
%
% - Version: controlled by a new command:
%		\newcommand{\version}{v1.0.0}
%
% - Booleans: there are three booleans used in this document:
%	- 'corrige', controlled by defining the variable 'correction'
%	- 'fr', controlled by defining the variable 'french'
%	- 'en', controlled by defining the variable 'english'
% You can define those variables in a makefile using such a command:
% pdflatex -shell-escape -jobname="elech310_tp1_fr" "\def\french{} \input{elech310_tp1.tex}"
%
%%%%%%%%%%%%%%%%%%%%%%%%%%%%%%%%%%%%%%%%%%%%%%%%%%%%%%%%%%%%%%%%%%%%%%%%%%%%%%%%%%%%%%%%%%%

%Numero du TP :
\def \tpnumber {TP 1 }

\newcommand{\version}{v1.0.0}


% ########     #####      #####    ##         #########     ###     ##      ##  
% ##     ##  ##     ##  ##     ##  ##         ##           ## ##    ###     ##  
% ##     ##  ##     ##  ##     ##  ##         ##          ##   ##   ## ##   ##  
% ########   ##     ##  ##     ##  ##         ######     ##     ##  ##  ##  ##  
% ##     ##  ##     ##  ##     ##  ##         ##         #########  ##   ## ##  
% ##     ##  ##     ##  ##     ##  ##         ##         ##     ##  ##     ###  
% ########     #####      #####    #########  #########  ##     ##  ##      ##  
\newboolean{corrige}
\ifx\correction\undefined
\setboolean{corrige}{false}% pas de corrigé
\else
\setboolean{corrige}{true}%corrigé
\fi

\newboolean{fr}
\ifx\french\undefined
\setboolean{fr}{false}% pas de corrigé
\else
\setboolean{fr}{true}%corrigé
\fi

\newboolean{en}
\ifx\english\undefined
\setboolean{en}{false}% pas de corrigé
\else
\setboolean{en}{true}%corrigé
\fi

%\setboolean{corrige}{false}% pas de corrigé

\definecolor{darkblue}{rgb}{0,0,0.5}

\pdfinfo{
/Author (ULB -- BEAMS)
/Title (\tpnumber, ELEC-H-310)
/ModDate (D:\pdfdate)
}

\hypersetup{
pdftitle={\tpnumber [ELEC-H-310] Choucroute numérique},
pdfauthor={ULB -- BEAMS},
pdfsubject={}
}

\theoremstyle{definition}% questions pas en italique
\newtheorem{Q}{Question}[] % numéroter les questions [section] ou non []


%  ######     #####    ##       ##  ##       ##     ###     ##      ##  ######      #######   
% ##     ##  ##     ##  ###     ###  ###     ###    ## ##    ###     ##  ##    ##   ##     ##  
% ##         ##     ##  ## ## ## ##  ## ## ## ##   ##   ##   ## ##   ##  ##     ##  ##         
% ##         ##     ##  ##  ###  ##  ##  ###  ##  ##     ##  ##  ##  ##  ##     ##   #######   
% ##         ##     ##  ##       ##  ##       ##  #########  ##   ## ##  ##     ##         ##  
% ##     ##  ##     ##  ##       ##  ##       ##  ##     ##  ##     ###  ##    ##   ##     ##  
%  #######     #####    ##       ##  ##       ##  ##     ##  ##      ##  ######      #######   
\newcommand{\reponse}[1]{% pour intégrer une réponse : \reponse{texte} : sera inclus si \boolean{corrige}
	\ifthenelse {\boolean{corrige}} {\fr{\paragraph{Réponse :}}\en{\paragraph{Answer:}} \color{darkblue}   #1\color{black}} {}
 }

 \newcommand{\fr}[1]{
 	\ifthenelse {\boolean{fr}} {#1} {}
 }

 \newcommand{\en}[1]{
 	\ifthenelse {\boolean{en}} {#1} {}
 }

\newcommand{\addcontentslinenono}[4]{\addtocontents{#1}{\protect\contentsline{#2}{#3}{#4}{}}}


%  #######   ##########  ##    ##   ##         #########  
% ##     ##      ##       ##  ##    ##         ##         
% ##             ##        ####     ##         ##         
%  #######       ##         ##      ##         ######     
%        ##      ##         ##      ##         ##         
% ##     ##      ##         ##      ##         ##         
%  #######       ##         ##      #########  #########  
%% fancy header & foot
\pagestyle{fancy}
\lhead{[ELEC-H-310] \fr{Électronique numérique}\en{Digital Electronics}\\ \tpnumber}
\rhead{\version\\ page \thepage}
\chead{\ifthenelse{\boolean{corrige}}{\fr{Corrigé}\en{Correction}}{}}
\cfoot{}
%%

\setlength{\parskip}{0.2cm plus2mm minus1mm} %espacement entre §
\setlength{\parindent}{0pt}

% ##########  ########   ##########  ##         #########  
%     ##         ##          ##      ##         ##         
%     ##         ##          ##      ##         ##         
%     ##         ##          ##      ##         ######     
%     ##         ##          ##      ##         ##         
%     ##         ##          ##      ##         ##         
%     ##      ########       ##      #########  ######### 
\date{\vspace{-1.7cm}\version}
\title{\vspace{-2cm} \tpnumber\\ \fr{Électronique numérique [ELEC-H-310] \ifthenelse{\boolean{corrige}}{~\\Corrigé}{}}%
\en{Digital Electronics [ELEC-H-310] \ifthenelse{\boolean{corrige}}{~\\Correction}{}}%
}




\begin{document}
\fr{\selectlanguage{french}}
\en{\selectlanguage{english}}

\maketitle
\vspace*{-1cm}

\begin{Q}
	\fr{Convertir dans les autres bases utiles les nombres suivants :}
	\en{Convert the following number in the relevant bases:}
	\begin{enumerate}
	\item $(82)_{10}$
	\item $(122)_{10}$
	\item $(1001110001)_2$
	\item $(F6D)_{16}$
	\item $(B65F)_{16}$
	\item $(0.625)_{10}$
	\end{enumerate}

	\reponse{
		\begin{center}
			\begin{tabular}{|l|l|l|l|}\hline
				Base 2 & Base 8 & Base 10 & Base 16 \\ \hline
				1010010 & 122 & \textbf{82} & 52 \\ \hline
				1111010 & 172 & \textbf{122} & 7A \\ \hline
				\textbf{1001110001} & 1161 & 625 & 271 \\ \hline
				111101101101 & 7555 & 3949 & \textbf{F6D} \\ \hline
				1011011001011111 & 133137 & 46687 & \textbf{B65F} \\ \hline
				0.101 & 0.5 & \textbf{0.625} & 0.A \\ \hline
			\end{tabular}
		\end{center}
	}
\end{Q}

\begin{Q}
\fr{Effectuer l’addition suivante dans toutes les bases utiles. Vérifier les résultats en
les convertissant en base 10 :}
\en{Compute this addition in all relevant bases. Check your results by converting them in base 10.}
$$(3633)_{10} + (254)_{10}$$

\reponse{~\\
	Base 2:
	\begin{tabular}{ccccccccccccc}
		& 1 & 1 & 1 & 0 & 0 & 0 & 1 & 1 & 0 & 0 & 0 & 1 \\
		+ & &   &   &   & 1 & 1 & 1 & 1 & 1 & 1 & 1 & 0\\ \hline
		& 1 & 1 & 1 & 1 & 0 & 0 & 1 & 0 & 1 & 1 & 1 & 1 \\
	\end{tabular}

	Base 8:
	\begin{tabular}{ccccc}
		  & 7 & 0 & 6 & 1 \\
		+ &   & 3 & 7 & 6 \\ \hline
		  & 7 & 4 & 5 & 7 \\
		\end{tabular}

	Base 16:
	\begin{tabular}{cccc}
		& E & 3 & 1 \\
		+ & & F & E \\ \hline
		& F & 2 & F\\
	\end{tabular}
}
\end{Q}


\begin{Q}
\fr{Représentation des nombres négatifs}
\en{Negative numbers representation}
\begin{enumerate}
	\item \fr{Représenter $(-14)_{10}$ sur 8 bits en base 2 dans les trois modes de représentation.}
	\en{Represent $(-14)_{10}$ on 8 bits in base 2 in all three representations.}
	\reponse{
		\begin{tabular}{|c|c|c|}\hline
			SVA & C1 & C2 \\ \hline
			10001110 & 11110001 & 11110010 \\ \hline
		\end{tabular}
	}
	\item \fr{Si on utilise 4 bits, quels sont, dans les 3 modes de représentation, les plus
	petites et les plus grandes valeurs représentables ? Comment se représente la
	valeur 0 ?}
	\en{If we use 4 bits, what are the highest and lowest values possible, in all three representations?
	How is represented the value 0?}
	\reponse{
		\begin{center}
			\begin{tabular}{|c|c|c|c|}\hline
				& min & 0 & max \\ \hline
				\multirow{2}{*}{SVA} & \multirow{2}{*}{1111} & 0000 & \multirow{2}{*}{0111} \\
				& & 1000 & \\ \hline
				\multirow{2}{*}{C1} & \multirow{2}{*}{10000} & 0000 & \multirow{2}{*}{0111} \\
				& & 1111 & \\ \hline
				C2 & 1000 & 0000 & 0111 \\ \hline
			\end{tabular}
		\end{center}

		\fr{À titre de bonus, ci-suit un tableau comparatif des différents modes de représentation (sur 4 bits):}
		\en{As a bonus, here is the full table comparing the three representations.}
		\begin{center}
			\begin{tabular}{|c|c|c|c|} \hline
				Base 10 & Signé & C1 & C2 \\ \hline
				7 & 0111 & 0111 & 0111 \\ \hline
				6 & 0110 & 0110 & 0110 \\ \hline
				5 & 0101 & 0101 & 0101 \\ \hline
				4 & 0100 & 0100 & 0100 \\ \hline
				3 & 0011 & 0011 & 0011 \\ \hline
				2 & 0010 & 0010 & 0010 \\ \hline
				1 & 0001 & 0001 & 0001 \\ \hline
				\multirow{2}{*}{0} & 0000 & 0000 & \multirow{2}{*}{0000} \\
				& 1000 & 1111 & \\ \hline
				-1 & 1001 & 1110 & 1111 \\ \hline
				-2 & 1010 & 1101 & 1110 \\ \hline
				-3 & 1011 & 1100 & 1101 \\ \hline
				-4 & 1100 & 1011 & 1100 \\ \hline
				-5 & 1101 & 1010 & 1011 \\ \hline
				-6 & 1110 & 1001 & 1010 \\ \hline
				-7 & 1111 & 1000 & 1001 \\ \hline
				-8 & N/A & N/A & 1000 \\ \hline
			\end{tabular}
		\end{center}
	}
\end{enumerate}
\end{Q}



\begin{Q}
\fr{En utilisant les axiomes prouver les théorèmes :}
\en{Proove the theorems using the axioms.}

\begin{minipage}[t]{0.5\linewidth}
	\centering Axiomes :
	$$A1a : x \in B, y \in B \rightarrow x+y \in B$$
	$$A1b : x \in B, y \in B \rightarrow x \cdot y \in B$$
	$$A2a : x+0 = x$$
	$$A2b : x \cdot 1 = x$$
	$$A3a : x \cdot (y+z) = x \cdot y+x \cdot z$$
	$$A3b : x+y \cdot z = (x+y)(x+z)$$
	$$A4a : x+y = y+x$$
	$$A4b : x \cdot y = y \cdot x$$
	$$A5a : x+\overline{x} = 1$$
	$$A5b : x \cdot \overline{x} = 0$$
	$$A6 : \exists x, y \in B : x \neq y$$
\end{minipage}
\begin{minipage}[t]{0.5\linewidth}
	\centering Théorèmes :
	$$T1a : x+x=x$$
	$$T1b : x \cdot x=x$$
	$$T2a : x+1=1$$
	$$T2b : x \cdot 0=0$$
	$$T3a : x+y \cdot x=x$$
	$$T3b : x \cdot (x+y)=x$$
	$$T4 : \overline{(\overline{x})}=x$$
\end{minipage}

\reponse{
\begin{itemize}
	\item $T1a : x+x=x$
	\begin{align*}
		x + x & = (x + x) \cdot 1&\mbox{, A2b}\\
		& = (x + x) \cdot (x + \overline{x})&\mbox{, A5a}\\
		& = x + x \cdot \overline{x}&\mbox{, A3b}\\
		& = x &\mbox{, A5b}
	\end{align*}

	\item $T1b : x \cdot x=x$
	\begin{align*}
		x \cdot x & = x \cdot x  + 0&\mbox{, A2a}\\
		& = x\cdot x + x \cdot \overline{x} &\mbox{, A5b}\\
		& = x \cdot (x + \overline{x}) &\mbox{, A3a} \\
		& = x \cdot 1&\mbox{, A5a}\\
		& = x&\mbox{, A2b}
	\end{align*}

	\item $T2a : x+1=1$
	\begin{align*}
		x + 1 & = (x+1) \cdot 1&\mbox{, A2b}\\
		& = (x+1) \cdot (x+\overline{x})&\mbox{, A5a}\\
		& = x + 1 \cdot \overline{x} &\mbox{, A3b}\\
		& = x + \overline{x} &\mbox{, A2b}\\
		& = 1 &\mbox{, A5a}
	\end{align*}

	\item $T2b : x \cdot 0=0$
	\begin{align*}
		x \cdot 0 & = x \cdot 0 + 0 &\mbox{, A2a} \\
		& = x \cdot 0 + x \cdot \overline{x}&\mbox{, A5b} \\
		& = x \cdot (0 + \overline{x}) &\mbox{, A3a} \\
		& = x \cdot (\overline{x}) &\mbox{, A2a} \\
		& = 0 &\mbox{, A5b}
	\end{align*}

	\item $T3a : x+y \cdot x=x$
	\begin{align*}
		x + x \cdot y & = x \cdot 1 + x \cdot y &\mbox{, A2b}\\
		& = x \cdot (1 + y)&\mbox{, A3a}\\
		& = x \cdot 1&\mbox{, T2a} \\
		& = x &\mbox{, A2b}
	\end{align*}

	\item $T3b : x \cdot (x+y)=x$
	\begin{align*}
		x \cdot (x+y) & = (x+0) \cdot (x+y) &\mbox{, A2a}\\
		& = x + 0 \cdot y &\mbox{, A3b}\\
		& = x + 0 &\mbox{, T2b} \\
		& = x &\mbox{, A2a}
	\end{align*}

	\item $T4 : \overline{(\overline{x})}=x$
	\begin{align*}
		\overline{(\overline{x})} & = \overline{(\overline{x})} \cdot 1 &\mbox{, A2b}\\
		& = \overline{(\overline{x})} \cdot (x + \overline{x}) &\mbox{, A5a}\\
		& = \overline{(\overline{x})} \cdot x + \overline{(\overline{x})} \cdot \overline{x} &\mbox{, A3a}\\
		& = \overline{(\overline{x})} \cdot x + 0 &\mbox{, A5b}\\
		& = \overline{(\overline{x})} \cdot x + x \cdot \overline{x} &\mbox{, A5b}\\
		& = x\cdot (\overline{(\overline{x})} + \overline{x})&\mbox{, A5a}\\
		& = x&\mbox{, A5a}\\
	\end{align*}
\end{itemize}
}

\end{Q}
%
% \begin{Q}
% 	L'approximation I(V) idéalisée avec seulement la tension de seuil est-elle une bonne approximation ?
% 	\label{Q:1}
% 	\reponse{Oui}%R
% \end{Q}


% \begin{figure}[H]
% 	\begin{center}
% 		\begin{circuitikz}\draw
% 			(0,0) node[anchor=east] {A} to [short,i>^=$I$] (1.5,0)
% 			(0,0) to [sDo, v=$V$] (2.5,0) node [anchor=west]{K}
% 		;\end{circuitikz}
% 	\end{center}
% \caption{Conventions électriques}
% \label{fig:zener_conv}
% \end{figure}

\end{document}
"
%
%%%%%%%%%%%%%%%%%%%%%%%%%%%%%%%%%%%%%%%%%%%%%%%%%%%%%%%%%%%%%%%%%%%%%%%%%%%%%%%%%%%%%%%%%%%

%Numero du TP :
\def \tpnumber {TP 1 }

\newcommand{\version}{v1.0.0}


% ########     #####      #####    ##         #########     ###     ##      ##  
% ##     ##  ##     ##  ##     ##  ##         ##           ## ##    ###     ##  
% ##     ##  ##     ##  ##     ##  ##         ##          ##   ##   ## ##   ##  
% ########   ##     ##  ##     ##  ##         ######     ##     ##  ##  ##  ##  
% ##     ##  ##     ##  ##     ##  ##         ##         #########  ##   ## ##  
% ##     ##  ##     ##  ##     ##  ##         ##         ##     ##  ##     ###  
% ########     #####      #####    #########  #########  ##     ##  ##      ##  
\newboolean{corrige}
\ifx\correction\undefined
\setboolean{corrige}{false}% pas de corrigé
\else
\setboolean{corrige}{true}%corrigé
\fi

\newboolean{fr}
\ifx\french\undefined
\setboolean{fr}{false}% pas de corrigé
\else
\setboolean{fr}{true}%corrigé
\fi

\newboolean{en}
\ifx\english\undefined
\setboolean{en}{false}% pas de corrigé
\else
\setboolean{en}{true}%corrigé
\fi

%\setboolean{corrige}{false}% pas de corrigé

\definecolor{darkblue}{rgb}{0,0,0.5}

\pdfinfo{
/Author (ULB -- BEAMS)
/Title (\tpnumber, ELEC-H-310)
/ModDate (D:\pdfdate)
}

\hypersetup{
pdftitle={\tpnumber [ELEC-H-310] Choucroute numérique},
pdfauthor={ULB -- BEAMS},
pdfsubject={}
}

\theoremstyle{definition}% questions pas en italique
\newtheorem{Q}{Question}[] % numéroter les questions [section] ou non []


%  ######     #####    ##       ##  ##       ##     ###     ##      ##  ######      #######   
% ##     ##  ##     ##  ###     ###  ###     ###    ## ##    ###     ##  ##    ##   ##     ##  
% ##         ##     ##  ## ## ## ##  ## ## ## ##   ##   ##   ## ##   ##  ##     ##  ##         
% ##         ##     ##  ##  ###  ##  ##  ###  ##  ##     ##  ##  ##  ##  ##     ##   #######   
% ##         ##     ##  ##       ##  ##       ##  #########  ##   ## ##  ##     ##         ##  
% ##     ##  ##     ##  ##       ##  ##       ##  ##     ##  ##     ###  ##    ##   ##     ##  
%  #######     #####    ##       ##  ##       ##  ##     ##  ##      ##  ######      #######   
\newcommand{\reponse}[1]{% pour intégrer une réponse : \reponse{texte} : sera inclus si \boolean{corrige}
	\ifthenelse {\boolean{corrige}} {\fr{\paragraph{Réponse :}}\en{\paragraph{Answer:}} \color{darkblue}   #1\color{black}} {}
 }

 \newcommand{\fr}[1]{
 	\ifthenelse {\boolean{fr}} {#1} {}
 }

 \newcommand{\en}[1]{
 	\ifthenelse {\boolean{en}} {#1} {}
 }

\newcommand{\addcontentslinenono}[4]{\addtocontents{#1}{\protect\contentsline{#2}{#3}{#4}{}}}


%  #######   ##########  ##    ##   ##         #########  
% ##     ##      ##       ##  ##    ##         ##         
% ##             ##        ####     ##         ##         
%  #######       ##         ##      ##         ######     
%        ##      ##         ##      ##         ##         
% ##     ##      ##         ##      ##         ##         
%  #######       ##         ##      #########  #########  
%% fancy header & foot
\pagestyle{fancy}
\lhead{[ELEC-H-310] \fr{Électronique numérique}\en{Digital Electronics}\\ \tpnumber}
\rhead{\version\\ page \thepage}
\chead{\ifthenelse{\boolean{corrige}}{\fr{Corrigé}\en{Correction}}{}}
\cfoot{}
%%

\setlength{\parskip}{0.2cm plus2mm minus1mm} %espacement entre §
\setlength{\parindent}{0pt}

% ##########  ########   ##########  ##         #########  
%     ##         ##          ##      ##         ##         
%     ##         ##          ##      ##         ##         
%     ##         ##          ##      ##         ######     
%     ##         ##          ##      ##         ##         
%     ##         ##          ##      ##         ##         
%     ##      ########       ##      #########  ######### 
\date{\vspace{-1.7cm}\version}
\title{\vspace{-2cm} \tpnumber\\ \fr{Électronique numérique [ELEC-H-310] \ifthenelse{\boolean{corrige}}{~\\Corrigé}{}}%
\en{Digital Electronics [ELEC-H-310] \ifthenelse{\boolean{corrige}}{~\\Correction}{}}%
}




\begin{document}
\fr{\selectlanguage{french}}
\en{\selectlanguage{english}}

\maketitle
\vspace*{-1cm}

\begin{Q}
	\fr{Convertir dans les autres bases utiles les nombres suivants :}
	\en{Convert the following number in the relevant bases:}
	\begin{enumerate}
	\item $(82)_{10}$
	\item $(122)_{10}$
	\item $(1001110001)_2$
	\item $(F6D)_{16}$
	\item $(B65F)_{16}$
	\item $(0.625)_{10}$
	\end{enumerate}

	\reponse{
		\begin{center}
			\begin{tabular}{|l|l|l|l|}\hline
				Base 2 & Base 8 & Base 10 & Base 16 \\ \hline
				1010010 & 122 & \textbf{82} & 52 \\ \hline
				1111010 & 172 & \textbf{122} & 7A \\ \hline
				\textbf{1001110001} & 1161 & 625 & 271 \\ \hline
				111101101101 & 7555 & 3949 & \textbf{F6D} \\ \hline
				1011011001011111 & 133137 & 46687 & \textbf{B65F} \\ \hline
				0.101 & 0.5 & \textbf{0.625} & 0.A \\ \hline
			\end{tabular}
		\end{center}
	}
\end{Q}

\begin{Q}
\fr{Effectuer l’addition suivante dans toutes les bases utiles. Vérifier les résultats en
les convertissant en base 10 :}
\en{Compute this addition in all relevant bases. Check your results by converting them in base 10.}
$$(3633)_{10} + (254)_{10}$$

\reponse{~\\
	Base 2:
	\begin{tabular}{ccccccccccccc}
		& 1 & 1 & 1 & 0 & 0 & 0 & 1 & 1 & 0 & 0 & 0 & 1 \\
		+ & &   &   &   & 1 & 1 & 1 & 1 & 1 & 1 & 1 & 0\\ \hline
		& 1 & 1 & 1 & 1 & 0 & 0 & 1 & 0 & 1 & 1 & 1 & 1 \\
	\end{tabular}

	Base 8:
	\begin{tabular}{ccccc}
		  & 7 & 0 & 6 & 1 \\
		+ &   & 3 & 7 & 6 \\ \hline
		  & 7 & 4 & 5 & 7 \\
		\end{tabular}

	Base 16:
	\begin{tabular}{cccc}
		& E & 3 & 1 \\
		+ & & F & E \\ \hline
		& F & 2 & F\\
	\end{tabular}
}
\end{Q}


\begin{Q}
\fr{Représentation des nombres négatifs}
\en{Negative numbers representation}
\begin{enumerate}
	\item \fr{Représenter $(-14)_{10}$ sur 8 bits en base 2 dans les trois modes de représentation.}
	\en{Represent $(-14)_{10}$ on 8 bits in base 2 in all three representations.}
	\reponse{
		\begin{tabular}{|c|c|c|}\hline
			SVA & C1 & C2 \\ \hline
			10001110 & 11110001 & 11110010 \\ \hline
		\end{tabular}
	}
	\item \fr{Si on utilise 4 bits, quels sont, dans les 3 modes de représentation, les plus
	petites et les plus grandes valeurs représentables ? Comment se représente la
	valeur 0 ?}
	\en{If we use 4 bits, what are the highest and lowest values possible, in all three representations?
	How is represented the value 0?}
	\reponse{
		\begin{center}
			\begin{tabular}{|c|c|c|c|}\hline
				& min & 0 & max \\ \hline
				\multirow{2}{*}{SVA} & \multirow{2}{*}{1111} & 0000 & \multirow{2}{*}{0111} \\
				& & 1000 & \\ \hline
				\multirow{2}{*}{C1} & \multirow{2}{*}{10000} & 0000 & \multirow{2}{*}{0111} \\
				& & 1111 & \\ \hline
				C2 & 1000 & 0000 & 0111 \\ \hline
			\end{tabular}
		\end{center}

		\fr{À titre de bonus, ci-suit un tableau comparatif des différents modes de représentation (sur 4 bits):}
		\en{As a bonus, here is the full table comparing the three representations.}
		\begin{center}
			\begin{tabular}{|c|c|c|c|} \hline
				Base 10 & Signé & C1 & C2 \\ \hline
				7 & 0111 & 0111 & 0111 \\ \hline
				6 & 0110 & 0110 & 0110 \\ \hline
				5 & 0101 & 0101 & 0101 \\ \hline
				4 & 0100 & 0100 & 0100 \\ \hline
				3 & 0011 & 0011 & 0011 \\ \hline
				2 & 0010 & 0010 & 0010 \\ \hline
				1 & 0001 & 0001 & 0001 \\ \hline
				\multirow{2}{*}{0} & 0000 & 0000 & \multirow{2}{*}{0000} \\
				& 1000 & 1111 & \\ \hline
				-1 & 1001 & 1110 & 1111 \\ \hline
				-2 & 1010 & 1101 & 1110 \\ \hline
				-3 & 1011 & 1100 & 1101 \\ \hline
				-4 & 1100 & 1011 & 1100 \\ \hline
				-5 & 1101 & 1010 & 1011 \\ \hline
				-6 & 1110 & 1001 & 1010 \\ \hline
				-7 & 1111 & 1000 & 1001 \\ \hline
				-8 & N/A & N/A & 1000 \\ \hline
			\end{tabular}
		\end{center}
	}
\end{enumerate}
\end{Q}



\begin{Q}
\fr{En utilisant les axiomes prouver les théorèmes :}
\en{Proove the theorems using the axioms.}

\begin{minipage}[t]{0.5\linewidth}
	\centering Axiomes :
	$$A1a : x \in B, y \in B \rightarrow x+y \in B$$
	$$A1b : x \in B, y \in B \rightarrow x \cdot y \in B$$
	$$A2a : x+0 = x$$
	$$A2b : x \cdot 1 = x$$
	$$A3a : x \cdot (y+z) = x \cdot y+x \cdot z$$
	$$A3b : x+y \cdot z = (x+y)(x+z)$$
	$$A4a : x+y = y+x$$
	$$A4b : x \cdot y = y \cdot x$$
	$$A5a : x+\overline{x} = 1$$
	$$A5b : x \cdot \overline{x} = 0$$
	$$A6 : \exists x, y \in B : x \neq y$$
\end{minipage}
\begin{minipage}[t]{0.5\linewidth}
	\centering Théorèmes :
	$$T1a : x+x=x$$
	$$T1b : x \cdot x=x$$
	$$T2a : x+1=1$$
	$$T2b : x \cdot 0=0$$
	$$T3a : x+y \cdot x=x$$
	$$T3b : x \cdot (x+y)=x$$
	$$T4 : \overline{(\overline{x})}=x$$
\end{minipage}

\reponse{
\begin{itemize}
	\item $T1a : x+x=x$
	\begin{align*}
		x + x & = (x + x) \cdot 1&\mbox{, A2b}\\
		& = (x + x) \cdot (x + \overline{x})&\mbox{, A5a}\\
		& = x + x \cdot \overline{x}&\mbox{, A3b}\\
		& = x &\mbox{, A5b}
	\end{align*}

	\item $T1b : x \cdot x=x$
	\begin{align*}
		x \cdot x & = x \cdot x  + 0&\mbox{, A2a}\\
		& = x\cdot x + x \cdot \overline{x} &\mbox{, A5b}\\
		& = x \cdot (x + \overline{x}) &\mbox{, A3a} \\
		& = x \cdot 1&\mbox{, A5a}\\
		& = x&\mbox{, A2b}
	\end{align*}

	\item $T2a : x+1=1$
	\begin{align*}
		x + 1 & = (x+1) \cdot 1&\mbox{, A2b}\\
		& = (x+1) \cdot (x+\overline{x})&\mbox{, A5a}\\
		& = x + 1 \cdot \overline{x} &\mbox{, A3b}\\
		& = x + \overline{x} &\mbox{, A2b}\\
		& = 1 &\mbox{, A5a}
	\end{align*}

	\item $T2b : x \cdot 0=0$
	\begin{align*}
		x \cdot 0 & = x \cdot 0 + 0 &\mbox{, A2a} \\
		& = x \cdot 0 + x \cdot \overline{x}&\mbox{, A5b} \\
		& = x \cdot (0 + \overline{x}) &\mbox{, A3a} \\
		& = x \cdot (\overline{x}) &\mbox{, A2a} \\
		& = 0 &\mbox{, A5b}
	\end{align*}

	\item $T3a : x+y \cdot x=x$
	\begin{align*}
		x + x \cdot y & = x \cdot 1 + x \cdot y &\mbox{, A2b}\\
		& = x \cdot (1 + y)&\mbox{, A3a}\\
		& = x \cdot 1&\mbox{, T2a} \\
		& = x &\mbox{, A2b}
	\end{align*}

	\item $T3b : x \cdot (x+y)=x$
	\begin{align*}
		x \cdot (x+y) & = (x+0) \cdot (x+y) &\mbox{, A2a}\\
		& = x + 0 \cdot y &\mbox{, A3b}\\
		& = x + 0 &\mbox{, T2b} \\
		& = x &\mbox{, A2a}
	\end{align*}

	\item $T4 : \overline{(\overline{x})}=x$
	\begin{align*}
		\overline{(\overline{x})} & = \overline{(\overline{x})} \cdot 1 &\mbox{, A2b}\\
		& = \overline{(\overline{x})} \cdot (x + \overline{x}) &\mbox{, A5a}\\
		& = \overline{(\overline{x})} \cdot x + \overline{(\overline{x})} \cdot \overline{x} &\mbox{, A3a}\\
		& = \overline{(\overline{x})} \cdot x + 0 &\mbox{, A5b}\\
		& = \overline{(\overline{x})} \cdot x + x \cdot \overline{x} &\mbox{, A5b}\\
		& = x\cdot (\overline{(\overline{x})} + \overline{x})&\mbox{, A5a}\\
		& = x&\mbox{, A5a}\\
	\end{align*}
\end{itemize}
}

\end{Q}
%
% \begin{Q}
% 	L'approximation I(V) idéalisée avec seulement la tension de seuil est-elle une bonne approximation ?
% 	\label{Q:1}
% 	\reponse{Oui}%R
% \end{Q}


% \begin{figure}[H]
% 	\begin{center}
% 		\begin{circuitikz}\draw
% 			(0,0) node[anchor=east] {A} to [short,i>^=$I$] (1.5,0)
% 			(0,0) to [sDo, v=$V$] (2.5,0) node [anchor=west]{K}
% 		;\end{circuitikz}
% 	\end{center}
% \caption{Conventions électriques}
% \label{fig:zener_conv}
% \end{figure}

\end{document}
"
%
%%%%%%%%%%%%%%%%%%%%%%%%%%%%%%%%%%%%%%%%%%%%%%%%%%%%%%%%%%%%%%%%%%%%%%%%%%%%%%%%%%%%%%%%%%%

%Numero du TP :
\def \tpnumber {TP 1 }

\newcommand{\version}{v1.0.0}


% ########     #####      #####    ##         #########     ###     ##      ##  
% ##     ##  ##     ##  ##     ##  ##         ##           ## ##    ###     ##  
% ##     ##  ##     ##  ##     ##  ##         ##          ##   ##   ## ##   ##  
% ########   ##     ##  ##     ##  ##         ######     ##     ##  ##  ##  ##  
% ##     ##  ##     ##  ##     ##  ##         ##         #########  ##   ## ##  
% ##     ##  ##     ##  ##     ##  ##         ##         ##     ##  ##     ###  
% ########     #####      #####    #########  #########  ##     ##  ##      ##  
\newboolean{corrige}
\ifx\correction\undefined
\setboolean{corrige}{false}% pas de corrigé
\else
\setboolean{corrige}{true}%corrigé
\fi

\newboolean{fr}
\ifx\french\undefined
\setboolean{fr}{false}% pas de corrigé
\else
\setboolean{fr}{true}%corrigé
\fi

\newboolean{en}
\ifx\english\undefined
\setboolean{en}{false}% pas de corrigé
\else
\setboolean{en}{true}%corrigé
\fi

%\setboolean{corrige}{false}% pas de corrigé

\definecolor{darkblue}{rgb}{0,0,0.5}

\pdfinfo{
/Author (ULB -- BEAMS)
/Title (\tpnumber, ELEC-H-310)
/ModDate (D:\pdfdate)
}

\hypersetup{
pdftitle={\tpnumber [ELEC-H-310] Choucroute numérique},
pdfauthor={ULB -- BEAMS},
pdfsubject={}
}

\theoremstyle{definition}% questions pas en italique
\newtheorem{Q}{Question}[] % numéroter les questions [section] ou non []


%  ######     #####    ##       ##  ##       ##     ###     ##      ##  ######      #######   
% ##     ##  ##     ##  ###     ###  ###     ###    ## ##    ###     ##  ##    ##   ##     ##  
% ##         ##     ##  ## ## ## ##  ## ## ## ##   ##   ##   ## ##   ##  ##     ##  ##         
% ##         ##     ##  ##  ###  ##  ##  ###  ##  ##     ##  ##  ##  ##  ##     ##   #######   
% ##         ##     ##  ##       ##  ##       ##  #########  ##   ## ##  ##     ##         ##  
% ##     ##  ##     ##  ##       ##  ##       ##  ##     ##  ##     ###  ##    ##   ##     ##  
%  #######     #####    ##       ##  ##       ##  ##     ##  ##      ##  ######      #######   
\newcommand{\reponse}[1]{% pour intégrer une réponse : \reponse{texte} : sera inclus si \boolean{corrige}
	\ifthenelse {\boolean{corrige}} {\fr{\paragraph{Réponse :}}\en{\paragraph{Answer:}} \color{darkblue}   #1\color{black}} {}
 }

 \newcommand{\fr}[1]{
 	\ifthenelse {\boolean{fr}} {#1} {}
 }

 \newcommand{\en}[1]{
 	\ifthenelse {\boolean{en}} {#1} {}
 }

\newcommand{\addcontentslinenono}[4]{\addtocontents{#1}{\protect\contentsline{#2}{#3}{#4}{}}}


%  #######   ##########  ##    ##   ##         #########  
% ##     ##      ##       ##  ##    ##         ##         
% ##             ##        ####     ##         ##         
%  #######       ##         ##      ##         ######     
%        ##      ##         ##      ##         ##         
% ##     ##      ##         ##      ##         ##         
%  #######       ##         ##      #########  #########  
%% fancy header & foot
\pagestyle{fancy}
\lhead{[ELEC-H-310] \fr{Électronique numérique}\en{Digital Electronics}\\ \tpnumber}
\rhead{\version\\ page \thepage}
\chead{\ifthenelse{\boolean{corrige}}{\fr{Corrigé}\en{Correction}}{}}
\cfoot{}
%%

\setlength{\parskip}{0.2cm plus2mm minus1mm} %espacement entre §
\setlength{\parindent}{0pt}

% ##########  ########   ##########  ##         #########  
%     ##         ##          ##      ##         ##         
%     ##         ##          ##      ##         ##         
%     ##         ##          ##      ##         ######     
%     ##         ##          ##      ##         ##         
%     ##         ##          ##      ##         ##         
%     ##      ########       ##      #########  ######### 
\date{\vspace{-1.7cm}\version}
\title{\vspace{-2cm} \tpnumber\\ \fr{Électronique numérique [ELEC-H-310] \ifthenelse{\boolean{corrige}}{~\\Corrigé}{}}%
\en{Digital Electronics [ELEC-H-310] \ifthenelse{\boolean{corrige}}{~\\Correction}{}}%
}




\begin{document}
\fr{\selectlanguage{french}}
\en{\selectlanguage{english}}

\maketitle
\vspace*{-1cm}

\begin{Q}
	\fr{Convertir dans les autres bases utiles les nombres suivants :}
	\en{Convert the following number in the relevant bases:}
	\begin{enumerate}
	\item $(82)_{10}$
	\item $(122)_{10}$
	\item $(1001110001)_2$
	\item $(F6D)_{16}$
	\item $(B65F)_{16}$
	\item $(0.625)_{10}$
	\end{enumerate}

	\reponse{
		\begin{center}
			\begin{tabular}{|l|l|l|l|}\hline
				Base 2 & Base 8 & Base 10 & Base 16 \\ \hline
				1010010 & 122 & \textbf{82} & 52 \\ \hline
				1111010 & 172 & \textbf{122} & 7A \\ \hline
				\textbf{1001110001} & 1161 & 625 & 271 \\ \hline
				111101101101 & 7555 & 3949 & \textbf{F6D} \\ \hline
				1011011001011111 & 133137 & 46687 & \textbf{B65F} \\ \hline
				0.101 & 0.5 & \textbf{0.625} & 0.A \\ \hline
			\end{tabular}
		\end{center}
	}
\end{Q}

\begin{Q}
\fr{Effectuer l’addition suivante dans toutes les bases utiles. Vérifier les résultats en
les convertissant en base 10 :}
\en{Compute this addition in all relevant bases. Check your results by converting them in base 10.}
$$(3633)_{10} + (254)_{10}$$

\reponse{~\\
	Base 2:
	\begin{tabular}{ccccccccccccc}
		& 1 & 1 & 1 & 0 & 0 & 0 & 1 & 1 & 0 & 0 & 0 & 1 \\
		+ & &   &   &   & 1 & 1 & 1 & 1 & 1 & 1 & 1 & 0\\ \hline
		& 1 & 1 & 1 & 1 & 0 & 0 & 1 & 0 & 1 & 1 & 1 & 1 \\
	\end{tabular}

	Base 8:
	\begin{tabular}{ccccc}
		  & 7 & 0 & 6 & 1 \\
		+ &   & 3 & 7 & 6 \\ \hline
		  & 7 & 4 & 5 & 7 \\
		\end{tabular}

	Base 16:
	\begin{tabular}{cccc}
		& E & 3 & 1 \\
		+ & & F & E \\ \hline
		& F & 2 & F\\
	\end{tabular}
}
\end{Q}


\begin{Q}
\fr{Représentation des nombres négatifs}
\en{Negative numbers representation}
\begin{enumerate}
	\item \fr{Représenter $(-14)_{10}$ sur 8 bits en base 2 dans les trois modes de représentation.}
	\en{Represent $(-14)_{10}$ on 8 bits in base 2 in all three representations.}
	\reponse{
		\begin{tabular}{|c|c|c|}\hline
			SVA & C1 & C2 \\ \hline
			10001110 & 11110001 & 11110010 \\ \hline
		\end{tabular}
	}
	\item \fr{Si on utilise 4 bits, quels sont, dans les 3 modes de représentation, les plus
	petites et les plus grandes valeurs représentables ? Comment se représente la
	valeur 0 ?}
	\en{If we use 4 bits, what are the highest and lowest values possible, in all three representations?
	How is represented the value 0?}
	\reponse{
		\begin{center}
			\begin{tabular}{|c|c|c|c|}\hline
				& min & 0 & max \\ \hline
				\multirow{2}{*}{SVA} & \multirow{2}{*}{1111} & 0000 & \multirow{2}{*}{0111} \\
				& & 1000 & \\ \hline
				\multirow{2}{*}{C1} & \multirow{2}{*}{10000} & 0000 & \multirow{2}{*}{0111} \\
				& & 1111 & \\ \hline
				C2 & 1000 & 0000 & 0111 \\ \hline
			\end{tabular}
		\end{center}

		\fr{À titre de bonus, ci-suit un tableau comparatif des différents modes de représentation (sur 4 bits):}
		\en{As a bonus, here is the full table comparing the three representations.}
		\begin{center}
			\begin{tabular}{|c|c|c|c|} \hline
				Base 10 & Signé & C1 & C2 \\ \hline
				7 & 0111 & 0111 & 0111 \\ \hline
				6 & 0110 & 0110 & 0110 \\ \hline
				5 & 0101 & 0101 & 0101 \\ \hline
				4 & 0100 & 0100 & 0100 \\ \hline
				3 & 0011 & 0011 & 0011 \\ \hline
				2 & 0010 & 0010 & 0010 \\ \hline
				1 & 0001 & 0001 & 0001 \\ \hline
				\multirow{2}{*}{0} & 0000 & 0000 & \multirow{2}{*}{0000} \\
				& 1000 & 1111 & \\ \hline
				-1 & 1001 & 1110 & 1111 \\ \hline
				-2 & 1010 & 1101 & 1110 \\ \hline
				-3 & 1011 & 1100 & 1101 \\ \hline
				-4 & 1100 & 1011 & 1100 \\ \hline
				-5 & 1101 & 1010 & 1011 \\ \hline
				-6 & 1110 & 1001 & 1010 \\ \hline
				-7 & 1111 & 1000 & 1001 \\ \hline
				-8 & N/A & N/A & 1000 \\ \hline
			\end{tabular}
		\end{center}
	}
\end{enumerate}
\end{Q}



\begin{Q}
\fr{En utilisant les axiomes prouver les théorèmes :}
\en{Proove the theorems using the axioms.}

\begin{minipage}[t]{0.5\linewidth}
	\centering Axiomes :
	$$A1a : x \in B, y \in B \rightarrow x+y \in B$$
	$$A1b : x \in B, y \in B \rightarrow x \cdot y \in B$$
	$$A2a : x+0 = x$$
	$$A2b : x \cdot 1 = x$$
	$$A3a : x \cdot (y+z) = x \cdot y+x \cdot z$$
	$$A3b : x+y \cdot z = (x+y)(x+z)$$
	$$A4a : x+y = y+x$$
	$$A4b : x \cdot y = y \cdot x$$
	$$A5a : x+\overline{x} = 1$$
	$$A5b : x \cdot \overline{x} = 0$$
	$$A6 : \exists x, y \in B : x \neq y$$
\end{minipage}
\begin{minipage}[t]{0.5\linewidth}
	\centering Théorèmes :
	$$T1a : x+x=x$$
	$$T1b : x \cdot x=x$$
	$$T2a : x+1=1$$
	$$T2b : x \cdot 0=0$$
	$$T3a : x+y \cdot x=x$$
	$$T3b : x \cdot (x+y)=x$$
	$$T4 : \overline{(\overline{x})}=x$$
\end{minipage}

\reponse{
\begin{itemize}
	\item $T1a : x+x=x$
	\begin{align*}
		x + x & = (x + x) \cdot 1&\mbox{, A2b}\\
		& = (x + x) \cdot (x + \overline{x})&\mbox{, A5a}\\
		& = x + x \cdot \overline{x}&\mbox{, A3b}\\
		& = x &\mbox{, A5b}
	\end{align*}

	\item $T1b : x \cdot x=x$
	\begin{align*}
		x \cdot x & = x \cdot x  + 0&\mbox{, A2a}\\
		& = x\cdot x + x \cdot \overline{x} &\mbox{, A5b}\\
		& = x \cdot (x + \overline{x}) &\mbox{, A3a} \\
		& = x \cdot 1&\mbox{, A5a}\\
		& = x&\mbox{, A2b}
	\end{align*}

	\item $T2a : x+1=1$
	\begin{align*}
		x + 1 & = (x+1) \cdot 1&\mbox{, A2b}\\
		& = (x+1) \cdot (x+\overline{x})&\mbox{, A5a}\\
		& = x + 1 \cdot \overline{x} &\mbox{, A3b}\\
		& = x + \overline{x} &\mbox{, A2b}\\
		& = 1 &\mbox{, A5a}
	\end{align*}

	\item $T2b : x \cdot 0=0$
	\begin{align*}
		x \cdot 0 & = x \cdot 0 + 0 &\mbox{, A2a} \\
		& = x \cdot 0 + x \cdot \overline{x}&\mbox{, A5b} \\
		& = x \cdot (0 + \overline{x}) &\mbox{, A3a} \\
		& = x \cdot (\overline{x}) &\mbox{, A2a} \\
		& = 0 &\mbox{, A5b}
	\end{align*}

	\item $T3a : x+y \cdot x=x$
	\begin{align*}
		x + x \cdot y & = x \cdot 1 + x \cdot y &\mbox{, A2b}\\
		& = x \cdot (1 + y)&\mbox{, A3a}\\
		& = x \cdot 1&\mbox{, T2a} \\
		& = x &\mbox{, A2b}
	\end{align*}

	\item $T3b : x \cdot (x+y)=x$
	\begin{align*}
		x \cdot (x+y) & = (x+0) \cdot (x+y) &\mbox{, A2a}\\
		& = x + 0 \cdot y &\mbox{, A3b}\\
		& = x + 0 &\mbox{, T2b} \\
		& = x &\mbox{, A2a}
	\end{align*}

	\item $T4 : \overline{(\overline{x})}=x$
	\begin{align*}
		\overline{(\overline{x})} & = \overline{(\overline{x})} \cdot 1 &\mbox{, A2b}\\
		& = \overline{(\overline{x})} \cdot (x + \overline{x}) &\mbox{, A5a}\\
		& = \overline{(\overline{x})} \cdot x + \overline{(\overline{x})} \cdot \overline{x} &\mbox{, A3a}\\
		& = \overline{(\overline{x})} \cdot x + 0 &\mbox{, A5b}\\
		& = \overline{(\overline{x})} \cdot x + x \cdot \overline{x} &\mbox{, A5b}\\
		& = x\cdot (\overline{(\overline{x})} + \overline{x})&\mbox{, A5a}\\
		& = x&\mbox{, A5a}\\
	\end{align*}
\end{itemize}
}

\end{Q}
%
% \begin{Q}
% 	L'approximation I(V) idéalisée avec seulement la tension de seuil est-elle une bonne approximation ?
% 	\label{Q:1}
% 	\reponse{Oui}%R
% \end{Q}


% \begin{figure}[H]
% 	\begin{center}
% 		\begin{circuitikz}\draw
% 			(0,0) node[anchor=east] {A} to [short,i>^=$I$] (1.5,0)
% 			(0,0) to [sDo, v=$V$] (2.5,0) node [anchor=west]{K}
% 		;\end{circuitikz}
% 	\end{center}
% \caption{Conventions électriques}
% \label{fig:zener_conv}
% \end{figure}

\end{document}
"
%
%%%%%%%%%%%%%%%%%%%%%%%%%%%%%%%%%%%%%%%%%%%%%%%%%%%%%%%%%%%%%%%%%%%%%%%%%%%%%%%%%%%%%%%%%%%

%Numero du TP :
\def \tpnumber {TP 3 }

\newcommand{\version}{v1.0.0}


% ########     #####      #####    ##         #########     ###     ##      ##  
% ##     ##  ##     ##  ##     ##  ##         ##           ## ##    ###     ##  
% ##     ##  ##     ##  ##     ##  ##         ##          ##   ##   ## ##   ##  
% ########   ##     ##  ##     ##  ##         ######     ##     ##  ##  ##  ##  
% ##     ##  ##     ##  ##     ##  ##         ##         #########  ##   ## ##  
% ##     ##  ##     ##  ##     ##  ##         ##         ##     ##  ##     ###  
% ########     #####      #####    #########  #########  ##     ##  ##      ##  
\newboolean{corrige}
\ifx\correction\undefined
\setboolean{corrige}{false}% pas de corrigé
\else
\setboolean{corrige}{true}%corrigé
\fi

\newboolean{fr}
\ifx\french\undefined
\setboolean{fr}{false}% pas de corrigé
\else
\setboolean{fr}{true}%corrigé
\fi

\newboolean{en}
\ifx\english\undefined
\setboolean{en}{false}% pas de corrigé
\else
\setboolean{en}{true}%corrigé
\fi

%\setboolean{corrige}{false}% pas de corrigé

\definecolor{darkblue}{rgb}{0,0,0.5}

\pdfinfo{
/Author (ULB -- BEAMS)
/Title (\tpnumber, ELEC-H-310)
/ModDate (D:\pdfdate)
}

\hypersetup{
pdftitle={\tpnumber [ELEC-H-310] Choucroute numérique},
pdfauthor={ULB -- BEAMS},
pdfsubject={}
}

\theoremstyle{definition}% questions pas en italique
\newtheorem{Q}{Question}[] % numéroter les questions [section] ou non []


%  ######     #####    ##       ##  ##       ##     ###     ##      ##  ######      #######   
% ##     ##  ##     ##  ###     ###  ###     ###    ## ##    ###     ##  ##    ##   ##     ##  
% ##         ##     ##  ## ## ## ##  ## ## ## ##   ##   ##   ## ##   ##  ##     ##  ##         
% ##         ##     ##  ##  ###  ##  ##  ###  ##  ##     ##  ##  ##  ##  ##     ##   #######   
% ##         ##     ##  ##       ##  ##       ##  #########  ##   ## ##  ##     ##         ##  
% ##     ##  ##     ##  ##       ##  ##       ##  ##     ##  ##     ###  ##    ##   ##     ##  
%  #######     #####    ##       ##  ##       ##  ##     ##  ##      ##  ######      #######   
\newcommand{\reponse}[1]{% pour intégrer une réponse : \reponse{texte} : sera inclus si \boolean{corrige}
	\ifthenelse {\boolean{corrige}} {\fr{\paragraph{Réponse :}}\en{\paragraph{Answer:}} \color{darkblue}   #1\color{black}} {}
 }

 \newcommand{\fr}[1]{
 	\ifthenelse {\boolean{fr}} {#1} {}
 }

 \newcommand{\en}[1]{
 	\ifthenelse {\boolean{en}} {#1} {}
 }

\newcommand{\addcontentslinenono}[4]{\addtocontents{#1}{\protect\contentsline{#2}{#3}{#4}{}}}


%  #######   ##########  ##    ##   ##         #########  
% ##     ##      ##       ##  ##    ##         ##         
% ##             ##        ####     ##         ##         
%  #######       ##         ##      ##         ######     
%        ##      ##         ##      ##         ##         
% ##     ##      ##         ##      ##         ##         
%  #######       ##         ##      #########  #########  
%% fancy header & foot
\pagestyle{fancy}
\lhead{[ELEC-H-310] \fr{Électronique numérique}\en{Digital Electronics}\\ \tpnumber}
\rhead{\version\\ page \thepage}
\chead{\ifthenelse{\boolean{corrige}}{\fr{Corrigé}\en{Correction}}{}}
\cfoot{}
%%

\setlength{\parskip}{0.2cm plus2mm minus1mm} %espacement entre §
\setlength{\parindent}{0pt}

% ##########  ########   ##########  ##         #########  
%     ##         ##          ##      ##         ##         
%     ##         ##          ##      ##         ##         
%     ##         ##          ##      ##         ######     
%     ##         ##          ##      ##         ##         
%     ##         ##          ##      ##         ##         
%     ##      ########       ##      #########  ######### 
\date{\vspace{-1.7cm}\version}
\title{\vspace{-2cm} \tpnumber\\ \fr{Électronique numérique [ELEC-H-310] \ifthenelse{\boolean{corrige}}{~\\Corrigé}{}}%
\en{Digital Electronics [ELEC-H-310] \ifthenelse{\boolean{corrige}}{~\\Correction}{}}%
}




\begin{document}
\fr{\selectlanguage{french}}
\en{\selectlanguage{english}}
\maketitle
\vspace*{-1cm}

%Debut du TP
\begin{Q}
\fr{Trouver les impliquants premiers par la méthode de Quine - Mc Cluskey:}
\en{Find all prime implicant using the Quine-McCluskey method.}

\fr{\textit{Note~: $\sum_m$ désigne une somme de mintermes, $\sum_d$ une somme de « don't care ».}}
\en{\textit{N.B.: $\sum_m$ is a sum of minterms, $\sum_d$ is a sum of `don't care'.}}

\begin{enumerate}
%	\item $f(a,b,c,d)=\sum_m(2,4,8,9,10,11,13,15)$
	\item $f(a,b,c,d)=\sum_m(0,4,5,8,12,13)$
	\reponse{
		\fr{Il convient d'abord de convertir les nombres décimaux dans leur forme binaire~:}
		\en{We first need to convert decimal numbers into binary.}
		\begin{center}
			\begin{tabular}{c|cccc}
			 & a & b & c & d \\ \hline
			 0 & 0 & 0 & 0 & 0 \\
			 4 & 0 & 1 & 0 & 0 \\
			 5 & 0 & 1 & 0 & 1 \\
			 8 & 1 & 0 & 0 & 0 \\
			 12 & 1 & 1 & 0 & 0 \\
			 13 & 1 & 1 & 0 & 1 \\
			\end{tabular}
		\end{center}

		\fr{On peut ensuite établir les différents groupes et fusionner ceux-ci.
		La marque $\checkmark$ indique qu'une ligne a été regroupée avec une autre.
		Les nombres entre parenthèses indiquent quels nombres ont été fusionnés.}
		\en{Then, we can group the elements and merge them.
		The `$\checkmark$' states that a line as been merged.
		The numbers between parenthesis show the values that have been merged.}

		\begin{minipage}{0.5\textwidth}
			\begin{tabular}{ccccccc}
			 & a & b & c & d & & \\
			 $G_0$ & 0 & 0 & 0 & 0 & 0 & $\checkmark$ \\ \hline
			 $G_1$ & 0 & 1 & 0 & 0 & 4 & $\checkmark$ \\
			 & 1 & 0 & 0 & 0 & 8 & $\checkmark$ \\ \hline
			 $G_2$ & 1 & 1 & 0 & 0 & 12 & $\checkmark$ \\
			 & 0 & 1 & 0 & 1 & 5 & $\checkmark$ \\ \hline
			 $G_3$ & 1 & 1 & 0 & 1 & 13 & $\checkmark$ \\
			\end{tabular}
		\end{minipage}%
		\begin{minipage}{0.5\textwidth}
			\begin{tabular}{ccccccc}
			 & a & b & c & d & & \\
			 $G_{0'}$ & 0 & -- & 0 & 0 & (0,4) & $\checkmark$ \\
			 & -- & 0 & 0 & 0 & (0,8) & $\checkmark$ \\ \hline
			 $G_{1'}$ & -- & 1 & 0 & 0 & (4,12) & $\checkmark$ \\
			 & 0 & 1 & 0 & -- & (4,5) & $\checkmark$ \\
			 & 1 & -- & 0 & 0 & (8,12) & $\checkmark$ \\ \hline
			 $G_{2'}$ & 1 & 1 & 0 & -- & (12,13) & $\checkmark$ \\
			 & -- & 1 & 0 & 1 & (5,13) & $\checkmark$ \\
			\end{tabular}
		\end{minipage}
		\begin{center}
			\begin{tabular}{cccccccc}
			$G_{0''}$ & -- & -- & 0 & 0 & (0,4|8,12) & Fusion & \rdelim\}{2}{10mm}[IP1] \\
			& -- & -- & 0 & 0 & (0,8|4,12) & Fusion \\ \hline
			$G_{1''}$ & -- & 1 & 0 & -- & (4,12|5,13) & Fusion & \rdelim\}{2}{10mm}[IP2] \\
			& -- & 1 & 0 & -- & (4,5|12,13) & Fusion \\
			\end{tabular}
		\end{center}

		\fr{Les impliquants premiers sont donc $\overline{cd}$ (IP1) et $b\overline{c}$ (IP2).}
		\en{The prime implicants hence are $\overline{cd}$ (IP1) and $b\overline{c}$ (IP2).}

	}
	\item $f(a,b,c,d)=\sum_m(2,3,4,10,12,13)+\sum_d(11,14,15)$
	\reponse{
		$f(a,b,c,d)=\overline{b}c+b\overline{c}\overline{d}+ac+ab$
	}
%	\item $f(a,b,c,d,e)=\sum_m(0,1,2,4,11,15,17,20,21,31)+\sum_d(5,6,16,18,22,27)$
	\item $f(a,b,c,d,e,f)=\sum_m(16,28,53,60,63)$
	\reponse{
		$f(a,b,c,d,e,f)=abcdef+bcd\overline{ef}+ab\overline{c}d\overline{e}f+\overline{a}b\overline{cdef}$
	}
%	\item $f(a,b,c,d,e,f)=\sum_m(1,2,3,15,17,18,19,26,32,48,63)$
\end{enumerate}
\end{Q}














\begin{Q}
\fr{Trouver la fonction simplifiée :}
\en{Find the simplified function:}
\begin{enumerate}
	\item $f(a,b,c,d)=\sum_m(2,3,4,10,12,13)+\sum_d(11,14,15)$
	\reponse{
		\fr{Tableau de couverture avec les impliquants premiers de l'exercice précédent :}
		\en{The coverage table with the prime implicants of the previous exercise:}
		$i_1$ = $b\overline{cd}$, $i_2$ = $\overline{b}c$, $i_3$ = $ac$ et $i_4$ = $ab$.

		\begin{center}
			\begin{tabular}{ccccc}
			 & $i_1$ & $i_2$ & $i_3$ & $i_4$ \\ \hline
			 0010 & & $\checkmark$ & & \\ \hline
			 0011 & & $\checkmark$ & & \\ \hline
			 0100 & $\checkmark$ & & & \\ \hline
			 1010 & & $\checkmark$ & $\checkmark$ & \\ \hline
			 1100 & $\checkmark$ &  &  & $\checkmark$ \\ \hline
			 1101 & & & & $\checkmark$ \\ \hline
			 % 1011 &  & $\checkmark$ & $\checkmark$ & \\ \hline
			 % 1110 & & & $\checkmark$ & $\checkmark$ \\ \hline
			 % 1111 & & & $\checkmark$ & $\checkmark$ \\
			\end{tabular}
		\end{center}

		\fr{On peut en déduire l'équation de couverture suivante~:}
		\en{We can deduce the following coverage function:}
		\[1 = i_2 \cdot i_1 \cdot (i_2 + i_3) \cdot (i_1 + i_4) \cdot i_4 = i_2i_1i_4\]
		\fr{En utilisant les axiomes et théorèmes suivants~:}
		\en{Using these axioms and theorems:}
		\begin{align*}
		& x \cdot x = x\\
		& x \cdot (x + y) = x\\
		& (x + y) \cdot (x + z) = x + y \cdot z
		\end{align*}

		\fr{On trouve ainsi}\en{We find} $f(a,b,c,d)=\overline{b}c+b\overline{cd}+ab$
	}

	\item $f(a,b,c,d)=\sum_m(0,2,4,5,10,11,13,15)+\sum_d(6,8)$
	\reponse {
		$f(a,b,c,d)=\overline{a}b\overline{c}+\overline{b}\overline{d}+a\overline{b}c+abd$
	ou $f(a,b,c,d)=a\overline{b}c+b\overline{c}d+abd+\overline{ad}$
	ou $f(a,b,c,d)=\overline{a}b\overline{c} + a\overline{b}c + abd + \overline{a}\overline{d}$
	ou $f(a,b,c,d)=b\overline{c}d + a\overline{b}c + acd + \overline{a}\overline{d}$
	ou $f(a,b,c,d)=b\overline{c}d + acd + \overline{a}\overline{d} + \overline{b}\overline{d}$
	ou $f(a,b,c,d)=\overline{a}b\overline{c} + acd + abd + \overline{b}\overline{d}$
	ou $f(a,b,c,d)=\overline{a}b\overline{c} + b\overline{c}d + acd + \overline{b}\overline{d}$
	}
%	\item $f(a,b,c,d)=\sum_m(1,5,6,9,11,12,14,15)+ \sum_d(2,3,4,13)$

\end{enumerate}

\end{Q}


















\begin{Q}
\fr{Pour les fonctions proposées dessiner les K-maps, optimiser les fonctions et prévoir les termes redondants (problème des aléas) :}
\en{Draw the K-maps of the following function, optimize the functions and find redundant terms to avoid glitches.}
	\begin{enumerate}
		\item $f(a,b,c,d)=\sum_m(0,1,2,6,8,9,10,14)$
		\reponse{
			\begin{center}
				\askmapiv{$F=c\overline{d} + \overline{bc} + \overline{bd}$}{c d a b}{}{1010101011110000}{% raise Z input
				\color{red}\put(0.1,0.1){\dashbox{0.1}(1.8,0.8){}}%
				\color{red}\put(0.1,3.1){\dashbox{0.1}(1.8,0.8){}}%
				\color{green}\put(3.1,0.1){\dashbox{0.1}(0.8,3.8){}}%
				\color{blue}\put(0.2,0.2){\dashbox{0.2}(0.6,0.6){}}%
				\color{blue}\put(0.2,3.2){\dashbox{0.2}(0.6,0.6){}}%
				\color{blue}\put(3.2,0.2){\dashbox{0.2}(0.6,0.6){}}%
				\color{blue}\put(3.2,3.2){\dashbox{0.2}(0.6,0.6){}}%
				}
			\end{center}
		}
		\item $f(a,b,c,d)=\sum_m(1,3,5,7,8,9,12,13)$
		\reponse{
			\begin{center}
				\askmapiv{$F=\overline{a}d + a\overline{c} + \overline{c}d$}{c d a b}{}{0011111100001100}{% raise Z input
				\color{red}\put(0.1,0.1){\dashbox{0.1}(1.8,1.8){}}%
				\color{green}\put(1.1,2.1){\dashbox{0.1}(1.8,1.8){}}%
				\color{blue}\put(1.2,0.2){\dashbox{0.2}(0.6,3.6){}}%
				}
			\end{center}
		}
	\end{enumerate}

\end{Q}




















\begin{Q}
\fr{Dresser la table de Huffman pour le graphe suivant :}
\en{Build a Huffman table for this graph:}

\begin{center}
\begin{tikzpicture}[->,>=stealth',shorten >=1pt,auto,node distance=3.5cm,semithick,align=center]
  \tikzstyle{every state}=[fill=white,text=black]

  \node[state]         (A)              {$A$\\ Z=0};
  \node[state]         (B) [right of=A] {$B$\\ Z=0};
  \node[state]         (D) [below of=A] {$D$\\ Z=0};
  \node[state]         (C) [below of=B] {$C$\\ Z=1};

  \path (A) edge [bend left]  node {ab=01} (B)
            edge [bend right]  node {11} (D)
			edge [loop left]  node {00,10} (A)
        (B) edge [loop right] node {01,11} (B)
            edge [bend left]  node {00} (A)
			edge [bend left]  node {10} (C)
        (C) edge [bend left]  node {11} (D)
		    edge [loop right] node {10,00,01} (C)
        (D) edge [loop left]  node {11,00} (D)
            edge              node {01} (B)
			edge [bend left]  node {10} (C);
\end{tikzpicture}
\end{center}

\reponse{
\begin{center}
\begin{tabular}{lllllll}
\hline
\multicolumn{1}{|l|}{} & \multicolumn{1}{l|}{00}          & \multicolumn{1}{l|}{01}          & \multicolumn{1}{l|}{11}          & \multicolumn{1}{l|}{10}          & \multicolumn{1}{l|}{ab} & \multicolumn{1}{l|}{Z} \\ \hline
\multicolumn{1}{|l|}{A}      & \multicolumn{1}{l|}{\textbf{A}} & \multicolumn{1}{l|}{B}          & \multicolumn{1}{l|}{D}          & \multicolumn{1}{l|}{\textbf{A}} & \multicolumn{1}{l|}{}   & \multicolumn{1}{l|}{0} \\ \cline{1-5} \cline{7-7}
\multicolumn{1}{|l|}{B}      & \multicolumn{1}{l|}{A}          & \multicolumn{1}{l|}{\textbf{B}} & \multicolumn{1}{l|}{\textbf{B}}          & \multicolumn{1}{l|}{C}          & \multicolumn{1}{l|}{}   & \multicolumn{1}{l|}{0} \\ \cline{1-5} \cline{7-7}
\multicolumn{1}{|l|}{C}      & \multicolumn{1}{l|}{\textbf{C}}          & \multicolumn{1}{l|}{\textbf{C}}          & \multicolumn{1}{l|}{D} & \multicolumn{1}{l|}{\textbf{C}} & \multicolumn{1}{l|}{}   & \multicolumn{1}{l|}{1} \\ \cline{1-5} \cline{7-7}
\multicolumn{1}{|l|}{D}      & \multicolumn{1}{l|}{\textbf{D}}          & \multicolumn{1}{l|}{B} & \multicolumn{1}{l|}{\textbf{D}} & \multicolumn{1}{l|}{C}          & \multicolumn{1}{l|}{}   & \multicolumn{1}{l|}{0} \\ \cline{1-5} \cline{7-7}
                      &                                  &                                  &                                  &                                  &                         &
\end{tabular}
\end{center}
}
\end{Q}






















\begin{Q}
\fr{À partir de cette table de Huffman codée trouver les équations et le graphe d'états correspondant :}
\en{From this coded Huffman table, find the corresponding state graph and equations.}

\begin{center}
\begin{tabular}{lllllll}
\hline
\multicolumn{1}{|l|}{$Y_1$ $Y_2$} & \multicolumn{1}{l|}{00}          & \multicolumn{1}{l|}{01}          & \multicolumn{1}{l|}{11}          & \multicolumn{1}{l|}{10}          & \multicolumn{1}{l|}{ab} & \multicolumn{1}{l|}{Z} \\ \hline
\multicolumn{1}{|l|}{00}      & \multicolumn{1}{l|}{\textbf{00}} & \multicolumn{1}{l|}{01}          & \multicolumn{1}{l|}{01}          & \multicolumn{1}{l|}{\textbf{00}} & \multicolumn{1}{l|}{}   & \multicolumn{1}{l|}{0} \\ \cline{1-5} \cline{7-7}
\multicolumn{1}{|l|}{01}      & \multicolumn{1}{l|}{00}          & \multicolumn{1}{l|}{\textbf{01}} & \multicolumn{1}{l|}{11}          & \multicolumn{1}{l|}{11}          & \multicolumn{1}{l|}{}   & \multicolumn{1}{l|}{1} \\ \cline{1-5} \cline{7-7}
\multicolumn{1}{|l|}{11}      & \multicolumn{1}{l|}{\textbf{11}}          & \multicolumn{1}{l|}{10}          & \multicolumn{1}{l|}{\textbf{11}} & \multicolumn{1}{l|}{\textbf{11}} & \multicolumn{1}{l|}{}   & \multicolumn{1}{l|}{1} \\ \cline{1-5} \cline{7-7}
\multicolumn{1}{|l|}{10}      & \multicolumn{1}{l|}{11}          & \multicolumn{1}{l|}{\textbf{10}} & \multicolumn{1}{l|}{\textbf{10}} & \multicolumn{1}{l|}{00}          & \multicolumn{1}{l|}{}   & \multicolumn{1}{l|}{1} \\ \cline{1-5} \cline{7-7}
$y_1$ $y_2$                      &                                  &                                  &                                  &                                  &                         &
\end{tabular}
\end{center}

\reponse{
	\begin{center}
	\begin{tikzpicture}[->,>=stealth',shorten >=1pt,auto,node distance=3.5cm,semithick,align=center]
	  \tikzstyle{every state}=[fill=white,text=black]

	  \node[state]         (A)              {$00$\\ Z=0};
	  \node[state]         (B) [right of=A] {$01$\\ Z=1};
	  \node[state]         (D) [below of=A] {$10$\\ Z=1};
	  \node[state]         (C) [below of=B] {$11$\\ Z=1};

	  \path (A) edge [bend left]  node {ab=01,11} (B)
	            % edge [bend right]  node {11} (D)
				edge [loop left]  node {00,10} (A)
	        (B) edge [loop right] node {01} (B)
	            edge [bend left]  node {00} (A)
				edge [bend left]  node {11,10} (C)
				edge [bend left]  node {00} (A)
	        (C) edge [bend left]  node {01} (D)
			    edge [loop right] node {10,00,11} (C)
	        (D) edge [loop left]  node {11,01} (D)
	            edge [bend left]  node {10} (A)
				edge [bend left]  node {00} (C);
	\end{tikzpicture}
	\end{center}

	\fr{Afin de déterminer l'équation correspondante, on peut dériver des tables de Karnaugh de la table de Huffman.}
	\en{In order to find the underlying equation, we can deduce Karnaugh tables from the Huffan table.}

	\begin{center}
		\askmapiv{$Y_1=y_2a + y_1 \overline{a} + y_1b$}{a b $y_1$ $y_2$}{}{0011001101010111}{% raise Z input
		\color{red}\put(0.1,0.1){\dashbox{0.1}(1.8,1.8){}}%
		\color{green}\put(2.1,1.1){\dashbox{0.1}(1.8,1.8){}}%
		\color{blue}\put(1.1,0.2){\dashbox{0.2}(1.8,1.6){}}%
		}

		\askmapiv{$Y_2=y_1\overline{ab} + \overline{y_1}b + y_2a$}{a b $y_1$ $y_2$}{}{0011110001011101}{% raise Z input
		\color{red}\put(0.1,0.1){\dashbox{0.1}(0.8,1.8){}}%
		\color{green}\put(1.1,2.1){\dashbox{0.1}(1.8,1.8){}}%
		\color{blue}\put(2.1,1.1){\dashbox{0.2}(1.8,1.8){}}%
		}

		\askmapiv{$Z=y_2+y_1$}{a b $y_1$ $y_2$}{}{0-11-11101-1-111}{% raise Z input
		\color{red}\put(0.1,0.1){\dashbox{0.1}(3.8,1.8){}}%
		\color{blue}\put(0.2,1.1){\dashbox{0.2}(3.6,1.8){}}%
		}
	\end{center}
}
\end{Q}

\end{document}
