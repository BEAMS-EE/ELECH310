\documentclass[11pt,a4paper,dvipsnames]{article}
\usepackage[utf8]{inputenc}
\usepackage[T1]{fontenc}
\usepackage{amsthm} %numéroter les questions
\usepackage[frenchb,english]{babel}
\usepackage{datetime}
\usepackage{xspace} % typographie IN
\usepackage{hyperref}% hyperliens
\usepackage[all]{hypcap} %lien pointe en haut des figures
\usepackage[french]{varioref} %voir x p y
\usepackage{fancyhdr}% en têtes
\usepackage{graphicx} %include pictures
\usepackage{pgfplots}

\usepackage{tikz}
\usetikzlibrary{calc,arrows,automata}
\usetikzlibrary{babel}
\usepackage{circuitikz}
% \usepackage{gnuplottex}
\usepackage{float}
\usepackage{ifthen}

\usepackage[top=1.3 in, bottom=1.3 in, left=1.3 in, right=1.3 in]{geometry}
\usepackage[]{pdfpages}
\usepackage[]{attachfile}

\usepackage{amsmath}
\usepackage{amsfonts}
\usepackage{amssymb}
\usepackage{enumitem}
\setlist[enumerate]{label=\alph*)}% If you want only the x-th level to use this format, use '[enumerate,x]'
\usepackage{multirow}
\usepackage{bigdelim}%Braces in tabular

\usepackage{aeguill} %guillemets
\usepackage{askmaps}
\usepackage{colortbl}
\usepackage{subfig}
\usepackage{caption}


%%%%%%%%%%%%%%%%%%%%%%%%%%%%%%%%%%%%%%%%%%%%%%%%%%%%%%%%%%%%%%%%%%%%%%%%%%%%%%%%%%%%%%%%%%%
% READ THIS BEFORE CHANGING ANYTHING
%
%
% - TP number: can be changed using the command
%		\def \tpnumber {TP 1 }
%
% - Version: controlled by a new command:
%		\newcommand{\version}{v1.0.0}
%
% - Booleans: there are three booleans used in this document:
%	- 'corrige', controlled by defining the variable 'correction'
%	- 'fr', controlled by defining the variable 'french'
%	- 'en', controlled by defining the variable 'english'
% You can define those variables in a makefile using such a command:
% pdflatex -shell-escape -jobname="elech310_tp1_fr" "\def\french{} \documentclass[11pt,a4paper]{article}
\usepackage[utf8]{inputenc}
\usepackage[T1]{fontenc}
\usepackage{amsthm} %numéroter les questions
\usepackage[frenchb,english]{babel}
\usepackage{datetime}
\usepackage{xspace} % typographie IN
\usepackage{hyperref}% hyperliens
\usepackage[all]{hypcap} %lien pointe en haut des figures
\usepackage[french]{varioref} %voir x p y
\usepackage{fancyhdr}% en têtes
\usepackage{graphicx} %include pictures
\usepackage{pgfplots}

\usepackage{tikz}
\usetikzlibrary{calc}
\usetikzlibrary{babel}
\usepackage{circuitikz}
% \usepackage{gnuplottex}
\usepackage{float}
\usepackage{ifthen}

\usepackage[top=1.3 in, bottom=1.3 in, left=1.3 in, right=1.3 in]{geometry}
\usepackage[]{pdfpages}
\usepackage[]{attachfile}

\usepackage{amsmath}
\usepackage{enumitem}
\setlist[enumerate]{label=\alph*)}% If you want only the x-th level to use this format, use '[enumerate,x]'
\usepackage{multirow}

\usepackage{aeguill} %guillemets


%%%%%%%%%%%%%%%%%%%%%%%%%%%%%%%%%%%%%%%%%%%%%%%%%%%%%%%%%%%%%%%%%%%%%%%%%%%%%%%%%%%%%%%%%%%
% READ THIS BEFORE CHANGING ANYTHING
%
%
% - TP number: can be changed using the command
%		\def \tpnumber {TP 1 }
%
% - Version: controlled by a new command:
%		\newcommand{\version}{v1.0.0}
%
% - Booleans: there are three booleans used in this document:
%	- 'corrige', controlled by defining the variable 'correction'
%	- 'fr', controlled by defining the variable 'french'
%	- 'en', controlled by defining the variable 'english'
% You can define those variables in a makefile using such a command:
% pdflatex -shell-escape -jobname="elech310_tp1_fr" "\def\french{} \documentclass[11pt,a4paper]{article}
\usepackage[utf8]{inputenc}
\usepackage[T1]{fontenc}
\usepackage{amsthm} %numéroter les questions
\usepackage[frenchb,english]{babel}
\usepackage{datetime}
\usepackage{xspace} % typographie IN
\usepackage{hyperref}% hyperliens
\usepackage[all]{hypcap} %lien pointe en haut des figures
\usepackage[french]{varioref} %voir x p y
\usepackage{fancyhdr}% en têtes
\usepackage{graphicx} %include pictures
\usepackage{pgfplots}

\usepackage{tikz}
\usetikzlibrary{calc}
\usetikzlibrary{babel}
\usepackage{circuitikz}
% \usepackage{gnuplottex}
\usepackage{float}
\usepackage{ifthen}

\usepackage[top=1.3 in, bottom=1.3 in, left=1.3 in, right=1.3 in]{geometry}
\usepackage[]{pdfpages}
\usepackage[]{attachfile}

\usepackage{amsmath}
\usepackage{enumitem}
\setlist[enumerate]{label=\alph*)}% If you want only the x-th level to use this format, use '[enumerate,x]'
\usepackage{multirow}

\usepackage{aeguill} %guillemets


%%%%%%%%%%%%%%%%%%%%%%%%%%%%%%%%%%%%%%%%%%%%%%%%%%%%%%%%%%%%%%%%%%%%%%%%%%%%%%%%%%%%%%%%%%%
% READ THIS BEFORE CHANGING ANYTHING
%
%
% - TP number: can be changed using the command
%		\def \tpnumber {TP 1 }
%
% - Version: controlled by a new command:
%		\newcommand{\version}{v1.0.0}
%
% - Booleans: there are three booleans used in this document:
%	- 'corrige', controlled by defining the variable 'correction'
%	- 'fr', controlled by defining the variable 'french'
%	- 'en', controlled by defining the variable 'english'
% You can define those variables in a makefile using such a command:
% pdflatex -shell-escape -jobname="elech310_tp1_fr" "\def\french{} \documentclass[11pt,a4paper]{article}
\usepackage[utf8]{inputenc}
\usepackage[T1]{fontenc}
\usepackage{amsthm} %numéroter les questions
\usepackage[frenchb,english]{babel}
\usepackage{datetime}
\usepackage{xspace} % typographie IN
\usepackage{hyperref}% hyperliens
\usepackage[all]{hypcap} %lien pointe en haut des figures
\usepackage[french]{varioref} %voir x p y
\usepackage{fancyhdr}% en têtes
\usepackage{graphicx} %include pictures
\usepackage{pgfplots}

\usepackage{tikz}
\usetikzlibrary{calc}
\usetikzlibrary{babel}
\usepackage{circuitikz}
% \usepackage{gnuplottex}
\usepackage{float}
\usepackage{ifthen}

\usepackage[top=1.3 in, bottom=1.3 in, left=1.3 in, right=1.3 in]{geometry}
\usepackage[]{pdfpages}
\usepackage[]{attachfile}

\usepackage{amsmath}
\usepackage{enumitem}
\setlist[enumerate]{label=\alph*)}% If you want only the x-th level to use this format, use '[enumerate,x]'
\usepackage{multirow}

\usepackage{aeguill} %guillemets


%%%%%%%%%%%%%%%%%%%%%%%%%%%%%%%%%%%%%%%%%%%%%%%%%%%%%%%%%%%%%%%%%%%%%%%%%%%%%%%%%%%%%%%%%%%
% READ THIS BEFORE CHANGING ANYTHING
%
%
% - TP number: can be changed using the command
%		\def \tpnumber {TP 1 }
%
% - Version: controlled by a new command:
%		\newcommand{\version}{v1.0.0}
%
% - Booleans: there are three booleans used in this document:
%	- 'corrige', controlled by defining the variable 'correction'
%	- 'fr', controlled by defining the variable 'french'
%	- 'en', controlled by defining the variable 'english'
% You can define those variables in a makefile using such a command:
% pdflatex -shell-escape -jobname="elech310_tp1_fr" "\def\french{} \input{elech310_tp1.tex}"
%
%%%%%%%%%%%%%%%%%%%%%%%%%%%%%%%%%%%%%%%%%%%%%%%%%%%%%%%%%%%%%%%%%%%%%%%%%%%%%%%%%%%%%%%%%%%

%Numero du TP :
\def \tpnumber {TP 1 }

\newcommand{\version}{v1.0.0}


% ########     #####      #####    ##         #########     ###     ##      ##  
% ##     ##  ##     ##  ##     ##  ##         ##           ## ##    ###     ##  
% ##     ##  ##     ##  ##     ##  ##         ##          ##   ##   ## ##   ##  
% ########   ##     ##  ##     ##  ##         ######     ##     ##  ##  ##  ##  
% ##     ##  ##     ##  ##     ##  ##         ##         #########  ##   ## ##  
% ##     ##  ##     ##  ##     ##  ##         ##         ##     ##  ##     ###  
% ########     #####      #####    #########  #########  ##     ##  ##      ##  
\newboolean{corrige}
\ifx\correction\undefined
\setboolean{corrige}{false}% pas de corrigé
\else
\setboolean{corrige}{true}%corrigé
\fi

\newboolean{fr}
\ifx\french\undefined
\setboolean{fr}{false}% pas de corrigé
\else
\setboolean{fr}{true}%corrigé
\fi

\newboolean{en}
\ifx\english\undefined
\setboolean{en}{false}% pas de corrigé
\else
\setboolean{en}{true}%corrigé
\fi

%\setboolean{corrige}{false}% pas de corrigé

\definecolor{darkblue}{rgb}{0,0,0.5}

\pdfinfo{
/Author (ULB -- BEAMS)
/Title (\tpnumber, ELEC-H-310)
/ModDate (D:\pdfdate)
}

\hypersetup{
pdftitle={\tpnumber [ELEC-H-310] Choucroute numérique},
pdfauthor={ULB -- BEAMS},
pdfsubject={}
}

\theoremstyle{definition}% questions pas en italique
\newtheorem{Q}{Question}[] % numéroter les questions [section] ou non []


%  ######     #####    ##       ##  ##       ##     ###     ##      ##  ######      #######   
% ##     ##  ##     ##  ###     ###  ###     ###    ## ##    ###     ##  ##    ##   ##     ##  
% ##         ##     ##  ## ## ## ##  ## ## ## ##   ##   ##   ## ##   ##  ##     ##  ##         
% ##         ##     ##  ##  ###  ##  ##  ###  ##  ##     ##  ##  ##  ##  ##     ##   #######   
% ##         ##     ##  ##       ##  ##       ##  #########  ##   ## ##  ##     ##         ##  
% ##     ##  ##     ##  ##       ##  ##       ##  ##     ##  ##     ###  ##    ##   ##     ##  
%  #######     #####    ##       ##  ##       ##  ##     ##  ##      ##  ######      #######   
\newcommand{\reponse}[1]{% pour intégrer une réponse : \reponse{texte} : sera inclus si \boolean{corrige}
	\ifthenelse {\boolean{corrige}} {\fr{\paragraph{Réponse :}}\en{\paragraph{Answer:}} \color{darkblue}   #1\color{black}} {}
 }

 \newcommand{\fr}[1]{
 	\ifthenelse {\boolean{fr}} {#1} {}
 }

 \newcommand{\en}[1]{
 	\ifthenelse {\boolean{en}} {#1} {}
 }

\newcommand{\addcontentslinenono}[4]{\addtocontents{#1}{\protect\contentsline{#2}{#3}{#4}{}}}


%  #######   ##########  ##    ##   ##         #########  
% ##     ##      ##       ##  ##    ##         ##         
% ##             ##        ####     ##         ##         
%  #######       ##         ##      ##         ######     
%        ##      ##         ##      ##         ##         
% ##     ##      ##         ##      ##         ##         
%  #######       ##         ##      #########  #########  
%% fancy header & foot
\pagestyle{fancy}
\lhead{[ELEC-H-310] \fr{Électronique numérique}\en{Digital Electronics}\\ \tpnumber}
\rhead{\version\\ page \thepage}
\chead{\ifthenelse{\boolean{corrige}}{\fr{Corrigé}\en{Correction}}{}}
\cfoot{}
%%

\setlength{\parskip}{0.2cm plus2mm minus1mm} %espacement entre §
\setlength{\parindent}{0pt}

% ##########  ########   ##########  ##         #########  
%     ##         ##          ##      ##         ##         
%     ##         ##          ##      ##         ##         
%     ##         ##          ##      ##         ######     
%     ##         ##          ##      ##         ##         
%     ##         ##          ##      ##         ##         
%     ##      ########       ##      #########  ######### 
\date{\vspace{-1.7cm}\version}
\title{\vspace{-2cm} \tpnumber\\ \fr{Électronique numérique [ELEC-H-310] \ifthenelse{\boolean{corrige}}{~\\Corrigé}{}}%
\en{Digital Electronics [ELEC-H-310] \ifthenelse{\boolean{corrige}}{~\\Correction}{}}%
}




\begin{document}
\fr{\selectlanguage{french}}
\en{\selectlanguage{english}}

\maketitle
\vspace*{-1cm}

\begin{Q}
	\fr{Convertir dans les autres bases utiles les nombres suivants :}
	\en{Convert the following number in the relevant bases:}
	\begin{enumerate}
	\item $(82)_{10}$
	\item $(122)_{10}$
	\item $(1001110001)_2$
	\item $(F6D)_{16}$
	\item $(B65F)_{16}$
	\item $(0.625)_{10}$
	\end{enumerate}

	\reponse{
		\begin{center}
			\begin{tabular}{|l|l|l|l|}\hline
				Base 2 & Base 8 & Base 10 & Base 16 \\ \hline
				1010010 & 122 & \textbf{82} & 52 \\ \hline
				1111010 & 172 & \textbf{122} & 7A \\ \hline
				\textbf{1001110001} & 1161 & 625 & 271 \\ \hline
				111101101101 & 7555 & 3949 & \textbf{F6D} \\ \hline
				1011011001011111 & 133137 & 46687 & \textbf{B65F} \\ \hline
				0.101 & 0.5 & \textbf{0.625} & 0.A \\ \hline
			\end{tabular}
		\end{center}
	}
\end{Q}

\begin{Q}
\fr{Effectuer l’addition suivante dans toutes les bases utiles. Vérifier les résultats en
les convertissant en base 10 :}
\en{Compute this addition in all relevant bases. Check your results by converting them in base 10.}
$$(3633)_{10} + (254)_{10}$$

\reponse{~\\
	Base 2:
	\begin{tabular}{ccccccccccccc}
		& 1 & 1 & 1 & 0 & 0 & 0 & 1 & 1 & 0 & 0 & 0 & 1 \\
		+ & &   &   &   & 1 & 1 & 1 & 1 & 1 & 1 & 1 & 0\\ \hline
		& 1 & 1 & 1 & 1 & 0 & 0 & 1 & 0 & 1 & 1 & 1 & 1 \\
	\end{tabular}

	Base 8:
	\begin{tabular}{ccccc}
		  & 7 & 0 & 6 & 1 \\
		+ &   & 3 & 7 & 6 \\ \hline
		  & 7 & 4 & 5 & 7 \\
		\end{tabular}

	Base 16:
	\begin{tabular}{cccc}
		& E & 3 & 1 \\
		+ & & F & E \\ \hline
		& F & 2 & F\\
	\end{tabular}
}
\end{Q}


\begin{Q}
\fr{Représentation des nombres négatifs}
\en{Negative numbers representation}
\begin{enumerate}
	\item \fr{Représenter $(-14)_{10}$ sur 8 bits en base 2 dans les trois modes de représentation.}
	\en{Represent $(-14)_{10}$ on 8 bits in base 2 in all three representations.}
	\reponse{
		\begin{tabular}{|c|c|c|}\hline
			SVA & C1 & C2 \\ \hline
			10001110 & 11110001 & 11110010 \\ \hline
		\end{tabular}
	}
	\item \fr{Si on utilise 4 bits, quels sont, dans les 3 modes de représentation, les plus
	petites et les plus grandes valeurs représentables ? Comment se représente la
	valeur 0 ?}
	\en{If we use 4 bits, what are the highest and lowest values possible, in all three representations?
	How is represented the value 0?}
	\reponse{
		\begin{center}
			\begin{tabular}{|c|c|c|c|}\hline
				& min & 0 & max \\ \hline
				\multirow{2}{*}{SVA} & \multirow{2}{*}{1111} & 0000 & \multirow{2}{*}{0111} \\
				& & 1000 & \\ \hline
				\multirow{2}{*}{C1} & \multirow{2}{*}{10000} & 0000 & \multirow{2}{*}{0111} \\
				& & 1111 & \\ \hline
				C2 & 1000 & 0000 & 0111 \\ \hline
			\end{tabular}
		\end{center}

		\fr{À titre de bonus, ci-suit un tableau comparatif des différents modes de représentation (sur 4 bits):}
		\en{As a bonus, here is the full table comparing the three representations.}
		\begin{center}
			\begin{tabular}{|c|c|c|c|} \hline
				Base 10 & Signé & C1 & C2 \\ \hline
				7 & 0111 & 0111 & 0111 \\ \hline
				6 & 0110 & 0110 & 0110 \\ \hline
				5 & 0101 & 0101 & 0101 \\ \hline
				4 & 0100 & 0100 & 0100 \\ \hline
				3 & 0011 & 0011 & 0011 \\ \hline
				2 & 0010 & 0010 & 0010 \\ \hline
				1 & 0001 & 0001 & 0001 \\ \hline
				\multirow{2}{*}{0} & 0000 & 0000 & \multirow{2}{*}{0000} \\
				& 1000 & 1111 & \\ \hline
				-1 & 1001 & 1110 & 1111 \\ \hline
				-2 & 1010 & 1101 & 1110 \\ \hline
				-3 & 1011 & 1100 & 1101 \\ \hline
				-4 & 1100 & 1011 & 1100 \\ \hline
				-5 & 1101 & 1010 & 1011 \\ \hline
				-6 & 1110 & 1001 & 1010 \\ \hline
				-7 & 1111 & 1000 & 1001 \\ \hline
				-8 & N/A & N/A & 1000 \\ \hline
			\end{tabular}
		\end{center}
	}
\end{enumerate}
\end{Q}



\begin{Q}
\fr{En utilisant les axiomes prouver les théorèmes :}
\en{Proove the theorems using the axioms.}

\begin{minipage}[t]{0.5\linewidth}
	\centering Axiomes :
	$$A1a : x \in B, y \in B \rightarrow x+y \in B$$
	$$A1b : x \in B, y \in B \rightarrow x \cdot y \in B$$
	$$A2a : x+0 = x$$
	$$A2b : x \cdot 1 = x$$
	$$A3a : x \cdot (y+z) = x \cdot y+x \cdot z$$
	$$A3b : x+y \cdot z = (x+y)(x+z)$$
	$$A4a : x+y = y+x$$
	$$A4b : x \cdot y = y \cdot x$$
	$$A5a : x+\overline{x} = 1$$
	$$A5b : x \cdot \overline{x} = 0$$
	$$A6 : \exists x, y \in B : x \neq y$$
\end{minipage}
\begin{minipage}[t]{0.5\linewidth}
	\centering Théorèmes :
	$$T1a : x+x=x$$
	$$T1b : x \cdot x=x$$
	$$T2a : x+1=1$$
	$$T2b : x \cdot 0=0$$
	$$T3a : x+y \cdot x=x$$
	$$T3b : x \cdot (x+y)=x$$
	$$T4 : \overline{(\overline{x})}=x$$
\end{minipage}

\reponse{
\begin{itemize}
	\item $T1a : x+x=x$
	\begin{align*}
		x + x & = (x + x) \cdot 1&\mbox{, A2b}\\
		& = (x + x) \cdot (x + \overline{x})&\mbox{, A5a}\\
		& = x + x \cdot \overline{x}&\mbox{, A3b}\\
		& = x &\mbox{, A5b}
	\end{align*}

	\item $T1b : x \cdot x=x$
	\begin{align*}
		x \cdot x & = x \cdot x  + 0&\mbox{, A2a}\\
		& = x\cdot x + x \cdot \overline{x} &\mbox{, A5b}\\
		& = x \cdot (x + \overline{x}) &\mbox{, A3a} \\
		& = x \cdot 1&\mbox{, A5a}\\
		& = x&\mbox{, A2b}
	\end{align*}

	\item $T2a : x+1=1$
	\begin{align*}
		x + 1 & = (x+1) \cdot 1&\mbox{, A2b}\\
		& = (x+1) \cdot (x+\overline{x})&\mbox{, A5a}\\
		& = x + 1 \cdot \overline{x} &\mbox{, A3b}\\
		& = x + \overline{x} &\mbox{, A2b}\\
		& = 1 &\mbox{, A5a}
	\end{align*}

	\item $T2b : x \cdot 0=0$
	\begin{align*}
		x \cdot 0 & = x \cdot 0 + 0 &\mbox{, A2a} \\
		& = x \cdot 0 + x \cdot \overline{x}&\mbox{, A5b} \\
		& = x \cdot (0 + \overline{x}) &\mbox{, A3a} \\
		& = x \cdot (\overline{x}) &\mbox{, A2a} \\
		& = 0 &\mbox{, A5b}
	\end{align*}

	\item $T3a : x+y \cdot x=x$
	\begin{align*}
		x + x \cdot y & = x \cdot 1 + x \cdot y &\mbox{, A2b}\\
		& = x \cdot (1 + y)&\mbox{, A3a}\\
		& = x \cdot 1&\mbox{, T2a} \\
		& = x &\mbox{, A2b}
	\end{align*}

	\item $T3b : x \cdot (x+y)=x$
	\begin{align*}
		x \cdot (x+y) & = (x+0) \cdot (x+y) &\mbox{, A2a}\\
		& = x + 0 \cdot y &\mbox{, A3b}\\
		& = x + 0 &\mbox{, T2b} \\
		& = x &\mbox{, A2a}
	\end{align*}

	\item $T4 : \overline{(\overline{x})}=x$
	\begin{align*}
		\overline{(\overline{x})} & = \overline{(\overline{x})} \cdot 1 &\mbox{, A2b}\\
		& = \overline{(\overline{x})} \cdot (x + \overline{x}) &\mbox{, A5a}\\
		& = \overline{(\overline{x})} \cdot x + \overline{(\overline{x})} \cdot \overline{x} &\mbox{, A3a}\\
		& = \overline{(\overline{x})} \cdot x + 0 &\mbox{, A5b}\\
		& = \overline{(\overline{x})} \cdot x + x \cdot \overline{x} &\mbox{, A5b}\\
		& = x\cdot (\overline{(\overline{x})} + \overline{x})&\mbox{, A5a}\\
		& = x&\mbox{, A5a}\\
	\end{align*}
\end{itemize}
}

\end{Q}
%
% \begin{Q}
% 	L'approximation I(V) idéalisée avec seulement la tension de seuil est-elle une bonne approximation ?
% 	\label{Q:1}
% 	\reponse{Oui}%R
% \end{Q}


% \begin{figure}[H]
% 	\begin{center}
% 		\begin{circuitikz}\draw
% 			(0,0) node[anchor=east] {A} to [short,i>^=$I$] (1.5,0)
% 			(0,0) to [sDo, v=$V$] (2.5,0) node [anchor=west]{K}
% 		;\end{circuitikz}
% 	\end{center}
% \caption{Conventions électriques}
% \label{fig:zener_conv}
% \end{figure}

\end{document}
"
%
%%%%%%%%%%%%%%%%%%%%%%%%%%%%%%%%%%%%%%%%%%%%%%%%%%%%%%%%%%%%%%%%%%%%%%%%%%%%%%%%%%%%%%%%%%%

%Numero du TP :
\def \tpnumber {TP 1 }

\newcommand{\version}{v1.0.0}


% ########     #####      #####    ##         #########     ###     ##      ##  
% ##     ##  ##     ##  ##     ##  ##         ##           ## ##    ###     ##  
% ##     ##  ##     ##  ##     ##  ##         ##          ##   ##   ## ##   ##  
% ########   ##     ##  ##     ##  ##         ######     ##     ##  ##  ##  ##  
% ##     ##  ##     ##  ##     ##  ##         ##         #########  ##   ## ##  
% ##     ##  ##     ##  ##     ##  ##         ##         ##     ##  ##     ###  
% ########     #####      #####    #########  #########  ##     ##  ##      ##  
\newboolean{corrige}
\ifx\correction\undefined
\setboolean{corrige}{false}% pas de corrigé
\else
\setboolean{corrige}{true}%corrigé
\fi

\newboolean{fr}
\ifx\french\undefined
\setboolean{fr}{false}% pas de corrigé
\else
\setboolean{fr}{true}%corrigé
\fi

\newboolean{en}
\ifx\english\undefined
\setboolean{en}{false}% pas de corrigé
\else
\setboolean{en}{true}%corrigé
\fi

%\setboolean{corrige}{false}% pas de corrigé

\definecolor{darkblue}{rgb}{0,0,0.5}

\pdfinfo{
/Author (ULB -- BEAMS)
/Title (\tpnumber, ELEC-H-310)
/ModDate (D:\pdfdate)
}

\hypersetup{
pdftitle={\tpnumber [ELEC-H-310] Choucroute numérique},
pdfauthor={ULB -- BEAMS},
pdfsubject={}
}

\theoremstyle{definition}% questions pas en italique
\newtheorem{Q}{Question}[] % numéroter les questions [section] ou non []


%  ######     #####    ##       ##  ##       ##     ###     ##      ##  ######      #######   
% ##     ##  ##     ##  ###     ###  ###     ###    ## ##    ###     ##  ##    ##   ##     ##  
% ##         ##     ##  ## ## ## ##  ## ## ## ##   ##   ##   ## ##   ##  ##     ##  ##         
% ##         ##     ##  ##  ###  ##  ##  ###  ##  ##     ##  ##  ##  ##  ##     ##   #######   
% ##         ##     ##  ##       ##  ##       ##  #########  ##   ## ##  ##     ##         ##  
% ##     ##  ##     ##  ##       ##  ##       ##  ##     ##  ##     ###  ##    ##   ##     ##  
%  #######     #####    ##       ##  ##       ##  ##     ##  ##      ##  ######      #######   
\newcommand{\reponse}[1]{% pour intégrer une réponse : \reponse{texte} : sera inclus si \boolean{corrige}
	\ifthenelse {\boolean{corrige}} {\fr{\paragraph{Réponse :}}\en{\paragraph{Answer:}} \color{darkblue}   #1\color{black}} {}
 }

 \newcommand{\fr}[1]{
 	\ifthenelse {\boolean{fr}} {#1} {}
 }

 \newcommand{\en}[1]{
 	\ifthenelse {\boolean{en}} {#1} {}
 }

\newcommand{\addcontentslinenono}[4]{\addtocontents{#1}{\protect\contentsline{#2}{#3}{#4}{}}}


%  #######   ##########  ##    ##   ##         #########  
% ##     ##      ##       ##  ##    ##         ##         
% ##             ##        ####     ##         ##         
%  #######       ##         ##      ##         ######     
%        ##      ##         ##      ##         ##         
% ##     ##      ##         ##      ##         ##         
%  #######       ##         ##      #########  #########  
%% fancy header & foot
\pagestyle{fancy}
\lhead{[ELEC-H-310] \fr{Électronique numérique}\en{Digital Electronics}\\ \tpnumber}
\rhead{\version\\ page \thepage}
\chead{\ifthenelse{\boolean{corrige}}{\fr{Corrigé}\en{Correction}}{}}
\cfoot{}
%%

\setlength{\parskip}{0.2cm plus2mm minus1mm} %espacement entre §
\setlength{\parindent}{0pt}

% ##########  ########   ##########  ##         #########  
%     ##         ##          ##      ##         ##         
%     ##         ##          ##      ##         ##         
%     ##         ##          ##      ##         ######     
%     ##         ##          ##      ##         ##         
%     ##         ##          ##      ##         ##         
%     ##      ########       ##      #########  ######### 
\date{\vspace{-1.7cm}\version}
\title{\vspace{-2cm} \tpnumber\\ \fr{Électronique numérique [ELEC-H-310] \ifthenelse{\boolean{corrige}}{~\\Corrigé}{}}%
\en{Digital Electronics [ELEC-H-310] \ifthenelse{\boolean{corrige}}{~\\Correction}{}}%
}




\begin{document}
\fr{\selectlanguage{french}}
\en{\selectlanguage{english}}

\maketitle
\vspace*{-1cm}

\begin{Q}
	\fr{Convertir dans les autres bases utiles les nombres suivants :}
	\en{Convert the following number in the relevant bases:}
	\begin{enumerate}
	\item $(82)_{10}$
	\item $(122)_{10}$
	\item $(1001110001)_2$
	\item $(F6D)_{16}$
	\item $(B65F)_{16}$
	\item $(0.625)_{10}$
	\end{enumerate}

	\reponse{
		\begin{center}
			\begin{tabular}{|l|l|l|l|}\hline
				Base 2 & Base 8 & Base 10 & Base 16 \\ \hline
				1010010 & 122 & \textbf{82} & 52 \\ \hline
				1111010 & 172 & \textbf{122} & 7A \\ \hline
				\textbf{1001110001} & 1161 & 625 & 271 \\ \hline
				111101101101 & 7555 & 3949 & \textbf{F6D} \\ \hline
				1011011001011111 & 133137 & 46687 & \textbf{B65F} \\ \hline
				0.101 & 0.5 & \textbf{0.625} & 0.A \\ \hline
			\end{tabular}
		\end{center}
	}
\end{Q}

\begin{Q}
\fr{Effectuer l’addition suivante dans toutes les bases utiles. Vérifier les résultats en
les convertissant en base 10 :}
\en{Compute this addition in all relevant bases. Check your results by converting them in base 10.}
$$(3633)_{10} + (254)_{10}$$

\reponse{~\\
	Base 2:
	\begin{tabular}{ccccccccccccc}
		& 1 & 1 & 1 & 0 & 0 & 0 & 1 & 1 & 0 & 0 & 0 & 1 \\
		+ & &   &   &   & 1 & 1 & 1 & 1 & 1 & 1 & 1 & 0\\ \hline
		& 1 & 1 & 1 & 1 & 0 & 0 & 1 & 0 & 1 & 1 & 1 & 1 \\
	\end{tabular}

	Base 8:
	\begin{tabular}{ccccc}
		  & 7 & 0 & 6 & 1 \\
		+ &   & 3 & 7 & 6 \\ \hline
		  & 7 & 4 & 5 & 7 \\
		\end{tabular}

	Base 16:
	\begin{tabular}{cccc}
		& E & 3 & 1 \\
		+ & & F & E \\ \hline
		& F & 2 & F\\
	\end{tabular}
}
\end{Q}


\begin{Q}
\fr{Représentation des nombres négatifs}
\en{Negative numbers representation}
\begin{enumerate}
	\item \fr{Représenter $(-14)_{10}$ sur 8 bits en base 2 dans les trois modes de représentation.}
	\en{Represent $(-14)_{10}$ on 8 bits in base 2 in all three representations.}
	\reponse{
		\begin{tabular}{|c|c|c|}\hline
			SVA & C1 & C2 \\ \hline
			10001110 & 11110001 & 11110010 \\ \hline
		\end{tabular}
	}
	\item \fr{Si on utilise 4 bits, quels sont, dans les 3 modes de représentation, les plus
	petites et les plus grandes valeurs représentables ? Comment se représente la
	valeur 0 ?}
	\en{If we use 4 bits, what are the highest and lowest values possible, in all three representations?
	How is represented the value 0?}
	\reponse{
		\begin{center}
			\begin{tabular}{|c|c|c|c|}\hline
				& min & 0 & max \\ \hline
				\multirow{2}{*}{SVA} & \multirow{2}{*}{1111} & 0000 & \multirow{2}{*}{0111} \\
				& & 1000 & \\ \hline
				\multirow{2}{*}{C1} & \multirow{2}{*}{10000} & 0000 & \multirow{2}{*}{0111} \\
				& & 1111 & \\ \hline
				C2 & 1000 & 0000 & 0111 \\ \hline
			\end{tabular}
		\end{center}

		\fr{À titre de bonus, ci-suit un tableau comparatif des différents modes de représentation (sur 4 bits):}
		\en{As a bonus, here is the full table comparing the three representations.}
		\begin{center}
			\begin{tabular}{|c|c|c|c|} \hline
				Base 10 & Signé & C1 & C2 \\ \hline
				7 & 0111 & 0111 & 0111 \\ \hline
				6 & 0110 & 0110 & 0110 \\ \hline
				5 & 0101 & 0101 & 0101 \\ \hline
				4 & 0100 & 0100 & 0100 \\ \hline
				3 & 0011 & 0011 & 0011 \\ \hline
				2 & 0010 & 0010 & 0010 \\ \hline
				1 & 0001 & 0001 & 0001 \\ \hline
				\multirow{2}{*}{0} & 0000 & 0000 & \multirow{2}{*}{0000} \\
				& 1000 & 1111 & \\ \hline
				-1 & 1001 & 1110 & 1111 \\ \hline
				-2 & 1010 & 1101 & 1110 \\ \hline
				-3 & 1011 & 1100 & 1101 \\ \hline
				-4 & 1100 & 1011 & 1100 \\ \hline
				-5 & 1101 & 1010 & 1011 \\ \hline
				-6 & 1110 & 1001 & 1010 \\ \hline
				-7 & 1111 & 1000 & 1001 \\ \hline
				-8 & N/A & N/A & 1000 \\ \hline
			\end{tabular}
		\end{center}
	}
\end{enumerate}
\end{Q}



\begin{Q}
\fr{En utilisant les axiomes prouver les théorèmes :}
\en{Proove the theorems using the axioms.}

\begin{minipage}[t]{0.5\linewidth}
	\centering Axiomes :
	$$A1a : x \in B, y \in B \rightarrow x+y \in B$$
	$$A1b : x \in B, y \in B \rightarrow x \cdot y \in B$$
	$$A2a : x+0 = x$$
	$$A2b : x \cdot 1 = x$$
	$$A3a : x \cdot (y+z) = x \cdot y+x \cdot z$$
	$$A3b : x+y \cdot z = (x+y)(x+z)$$
	$$A4a : x+y = y+x$$
	$$A4b : x \cdot y = y \cdot x$$
	$$A5a : x+\overline{x} = 1$$
	$$A5b : x \cdot \overline{x} = 0$$
	$$A6 : \exists x, y \in B : x \neq y$$
\end{minipage}
\begin{minipage}[t]{0.5\linewidth}
	\centering Théorèmes :
	$$T1a : x+x=x$$
	$$T1b : x \cdot x=x$$
	$$T2a : x+1=1$$
	$$T2b : x \cdot 0=0$$
	$$T3a : x+y \cdot x=x$$
	$$T3b : x \cdot (x+y)=x$$
	$$T4 : \overline{(\overline{x})}=x$$
\end{minipage}

\reponse{
\begin{itemize}
	\item $T1a : x+x=x$
	\begin{align*}
		x + x & = (x + x) \cdot 1&\mbox{, A2b}\\
		& = (x + x) \cdot (x + \overline{x})&\mbox{, A5a}\\
		& = x + x \cdot \overline{x}&\mbox{, A3b}\\
		& = x &\mbox{, A5b}
	\end{align*}

	\item $T1b : x \cdot x=x$
	\begin{align*}
		x \cdot x & = x \cdot x  + 0&\mbox{, A2a}\\
		& = x\cdot x + x \cdot \overline{x} &\mbox{, A5b}\\
		& = x \cdot (x + \overline{x}) &\mbox{, A3a} \\
		& = x \cdot 1&\mbox{, A5a}\\
		& = x&\mbox{, A2b}
	\end{align*}

	\item $T2a : x+1=1$
	\begin{align*}
		x + 1 & = (x+1) \cdot 1&\mbox{, A2b}\\
		& = (x+1) \cdot (x+\overline{x})&\mbox{, A5a}\\
		& = x + 1 \cdot \overline{x} &\mbox{, A3b}\\
		& = x + \overline{x} &\mbox{, A2b}\\
		& = 1 &\mbox{, A5a}
	\end{align*}

	\item $T2b : x \cdot 0=0$
	\begin{align*}
		x \cdot 0 & = x \cdot 0 + 0 &\mbox{, A2a} \\
		& = x \cdot 0 + x \cdot \overline{x}&\mbox{, A5b} \\
		& = x \cdot (0 + \overline{x}) &\mbox{, A3a} \\
		& = x \cdot (\overline{x}) &\mbox{, A2a} \\
		& = 0 &\mbox{, A5b}
	\end{align*}

	\item $T3a : x+y \cdot x=x$
	\begin{align*}
		x + x \cdot y & = x \cdot 1 + x \cdot y &\mbox{, A2b}\\
		& = x \cdot (1 + y)&\mbox{, A3a}\\
		& = x \cdot 1&\mbox{, T2a} \\
		& = x &\mbox{, A2b}
	\end{align*}

	\item $T3b : x \cdot (x+y)=x$
	\begin{align*}
		x \cdot (x+y) & = (x+0) \cdot (x+y) &\mbox{, A2a}\\
		& = x + 0 \cdot y &\mbox{, A3b}\\
		& = x + 0 &\mbox{, T2b} \\
		& = x &\mbox{, A2a}
	\end{align*}

	\item $T4 : \overline{(\overline{x})}=x$
	\begin{align*}
		\overline{(\overline{x})} & = \overline{(\overline{x})} \cdot 1 &\mbox{, A2b}\\
		& = \overline{(\overline{x})} \cdot (x + \overline{x}) &\mbox{, A5a}\\
		& = \overline{(\overline{x})} \cdot x + \overline{(\overline{x})} \cdot \overline{x} &\mbox{, A3a}\\
		& = \overline{(\overline{x})} \cdot x + 0 &\mbox{, A5b}\\
		& = \overline{(\overline{x})} \cdot x + x \cdot \overline{x} &\mbox{, A5b}\\
		& = x\cdot (\overline{(\overline{x})} + \overline{x})&\mbox{, A5a}\\
		& = x&\mbox{, A5a}\\
	\end{align*}
\end{itemize}
}

\end{Q}
%
% \begin{Q}
% 	L'approximation I(V) idéalisée avec seulement la tension de seuil est-elle une bonne approximation ?
% 	\label{Q:1}
% 	\reponse{Oui}%R
% \end{Q}


% \begin{figure}[H]
% 	\begin{center}
% 		\begin{circuitikz}\draw
% 			(0,0) node[anchor=east] {A} to [short,i>^=$I$] (1.5,0)
% 			(0,0) to [sDo, v=$V$] (2.5,0) node [anchor=west]{K}
% 		;\end{circuitikz}
% 	\end{center}
% \caption{Conventions électriques}
% \label{fig:zener_conv}
% \end{figure}

\end{document}
"
%
%%%%%%%%%%%%%%%%%%%%%%%%%%%%%%%%%%%%%%%%%%%%%%%%%%%%%%%%%%%%%%%%%%%%%%%%%%%%%%%%%%%%%%%%%%%

%Numero du TP :
\def \tpnumber {TP 1 }

\newcommand{\version}{v1.0.0}


% ########     #####      #####    ##         #########     ###     ##      ##  
% ##     ##  ##     ##  ##     ##  ##         ##           ## ##    ###     ##  
% ##     ##  ##     ##  ##     ##  ##         ##          ##   ##   ## ##   ##  
% ########   ##     ##  ##     ##  ##         ######     ##     ##  ##  ##  ##  
% ##     ##  ##     ##  ##     ##  ##         ##         #########  ##   ## ##  
% ##     ##  ##     ##  ##     ##  ##         ##         ##     ##  ##     ###  
% ########     #####      #####    #########  #########  ##     ##  ##      ##  
\newboolean{corrige}
\ifx\correction\undefined
\setboolean{corrige}{false}% pas de corrigé
\else
\setboolean{corrige}{true}%corrigé
\fi

\newboolean{fr}
\ifx\french\undefined
\setboolean{fr}{false}% pas de corrigé
\else
\setboolean{fr}{true}%corrigé
\fi

\newboolean{en}
\ifx\english\undefined
\setboolean{en}{false}% pas de corrigé
\else
\setboolean{en}{true}%corrigé
\fi

%\setboolean{corrige}{false}% pas de corrigé

\definecolor{darkblue}{rgb}{0,0,0.5}

\pdfinfo{
/Author (ULB -- BEAMS)
/Title (\tpnumber, ELEC-H-310)
/ModDate (D:\pdfdate)
}

\hypersetup{
pdftitle={\tpnumber [ELEC-H-310] Choucroute numérique},
pdfauthor={ULB -- BEAMS},
pdfsubject={}
}

\theoremstyle{definition}% questions pas en italique
\newtheorem{Q}{Question}[] % numéroter les questions [section] ou non []


%  ######     #####    ##       ##  ##       ##     ###     ##      ##  ######      #######   
% ##     ##  ##     ##  ###     ###  ###     ###    ## ##    ###     ##  ##    ##   ##     ##  
% ##         ##     ##  ## ## ## ##  ## ## ## ##   ##   ##   ## ##   ##  ##     ##  ##         
% ##         ##     ##  ##  ###  ##  ##  ###  ##  ##     ##  ##  ##  ##  ##     ##   #######   
% ##         ##     ##  ##       ##  ##       ##  #########  ##   ## ##  ##     ##         ##  
% ##     ##  ##     ##  ##       ##  ##       ##  ##     ##  ##     ###  ##    ##   ##     ##  
%  #######     #####    ##       ##  ##       ##  ##     ##  ##      ##  ######      #######   
\newcommand{\reponse}[1]{% pour intégrer une réponse : \reponse{texte} : sera inclus si \boolean{corrige}
	\ifthenelse {\boolean{corrige}} {\fr{\paragraph{Réponse :}}\en{\paragraph{Answer:}} \color{darkblue}   #1\color{black}} {}
 }

 \newcommand{\fr}[1]{
 	\ifthenelse {\boolean{fr}} {#1} {}
 }

 \newcommand{\en}[1]{
 	\ifthenelse {\boolean{en}} {#1} {}
 }

\newcommand{\addcontentslinenono}[4]{\addtocontents{#1}{\protect\contentsline{#2}{#3}{#4}{}}}


%  #######   ##########  ##    ##   ##         #########  
% ##     ##      ##       ##  ##    ##         ##         
% ##             ##        ####     ##         ##         
%  #######       ##         ##      ##         ######     
%        ##      ##         ##      ##         ##         
% ##     ##      ##         ##      ##         ##         
%  #######       ##         ##      #########  #########  
%% fancy header & foot
\pagestyle{fancy}
\lhead{[ELEC-H-310] \fr{Électronique numérique}\en{Digital Electronics}\\ \tpnumber}
\rhead{\version\\ page \thepage}
\chead{\ifthenelse{\boolean{corrige}}{\fr{Corrigé}\en{Correction}}{}}
\cfoot{}
%%

\setlength{\parskip}{0.2cm plus2mm minus1mm} %espacement entre §
\setlength{\parindent}{0pt}

% ##########  ########   ##########  ##         #########  
%     ##         ##          ##      ##         ##         
%     ##         ##          ##      ##         ##         
%     ##         ##          ##      ##         ######     
%     ##         ##          ##      ##         ##         
%     ##         ##          ##      ##         ##         
%     ##      ########       ##      #########  ######### 
\date{\vspace{-1.7cm}\version}
\title{\vspace{-2cm} \tpnumber\\ \fr{Électronique numérique [ELEC-H-310] \ifthenelse{\boolean{corrige}}{~\\Corrigé}{}}%
\en{Digital Electronics [ELEC-H-310] \ifthenelse{\boolean{corrige}}{~\\Correction}{}}%
}




\begin{document}
\fr{\selectlanguage{french}}
\en{\selectlanguage{english}}

\maketitle
\vspace*{-1cm}

\begin{Q}
	\fr{Convertir dans les autres bases utiles les nombres suivants :}
	\en{Convert the following number in the relevant bases:}
	\begin{enumerate}
	\item $(82)_{10}$
	\item $(122)_{10}$
	\item $(1001110001)_2$
	\item $(F6D)_{16}$
	\item $(B65F)_{16}$
	\item $(0.625)_{10}$
	\end{enumerate}

	\reponse{
		\begin{center}
			\begin{tabular}{|l|l|l|l|}\hline
				Base 2 & Base 8 & Base 10 & Base 16 \\ \hline
				1010010 & 122 & \textbf{82} & 52 \\ \hline
				1111010 & 172 & \textbf{122} & 7A \\ \hline
				\textbf{1001110001} & 1161 & 625 & 271 \\ \hline
				111101101101 & 7555 & 3949 & \textbf{F6D} \\ \hline
				1011011001011111 & 133137 & 46687 & \textbf{B65F} \\ \hline
				0.101 & 0.5 & \textbf{0.625} & 0.A \\ \hline
			\end{tabular}
		\end{center}
	}
\end{Q}

\begin{Q}
\fr{Effectuer l’addition suivante dans toutes les bases utiles. Vérifier les résultats en
les convertissant en base 10 :}
\en{Compute this addition in all relevant bases. Check your results by converting them in base 10.}
$$(3633)_{10} + (254)_{10}$$

\reponse{~\\
	Base 2:
	\begin{tabular}{ccccccccccccc}
		& 1 & 1 & 1 & 0 & 0 & 0 & 1 & 1 & 0 & 0 & 0 & 1 \\
		+ & &   &   &   & 1 & 1 & 1 & 1 & 1 & 1 & 1 & 0\\ \hline
		& 1 & 1 & 1 & 1 & 0 & 0 & 1 & 0 & 1 & 1 & 1 & 1 \\
	\end{tabular}

	Base 8:
	\begin{tabular}{ccccc}
		  & 7 & 0 & 6 & 1 \\
		+ &   & 3 & 7 & 6 \\ \hline
		  & 7 & 4 & 5 & 7 \\
		\end{tabular}

	Base 16:
	\begin{tabular}{cccc}
		& E & 3 & 1 \\
		+ & & F & E \\ \hline
		& F & 2 & F\\
	\end{tabular}
}
\end{Q}


\begin{Q}
\fr{Représentation des nombres négatifs}
\en{Negative numbers representation}
\begin{enumerate}
	\item \fr{Représenter $(-14)_{10}$ sur 8 bits en base 2 dans les trois modes de représentation.}
	\en{Represent $(-14)_{10}$ on 8 bits in base 2 in all three representations.}
	\reponse{
		\begin{tabular}{|c|c|c|}\hline
			SVA & C1 & C2 \\ \hline
			10001110 & 11110001 & 11110010 \\ \hline
		\end{tabular}
	}
	\item \fr{Si on utilise 4 bits, quels sont, dans les 3 modes de représentation, les plus
	petites et les plus grandes valeurs représentables ? Comment se représente la
	valeur 0 ?}
	\en{If we use 4 bits, what are the highest and lowest values possible, in all three representations?
	How is represented the value 0?}
	\reponse{
		\begin{center}
			\begin{tabular}{|c|c|c|c|}\hline
				& min & 0 & max \\ \hline
				\multirow{2}{*}{SVA} & \multirow{2}{*}{1111} & 0000 & \multirow{2}{*}{0111} \\
				& & 1000 & \\ \hline
				\multirow{2}{*}{C1} & \multirow{2}{*}{10000} & 0000 & \multirow{2}{*}{0111} \\
				& & 1111 & \\ \hline
				C2 & 1000 & 0000 & 0111 \\ \hline
			\end{tabular}
		\end{center}

		\fr{À titre de bonus, ci-suit un tableau comparatif des différents modes de représentation (sur 4 bits):}
		\en{As a bonus, here is the full table comparing the three representations.}
		\begin{center}
			\begin{tabular}{|c|c|c|c|} \hline
				Base 10 & Signé & C1 & C2 \\ \hline
				7 & 0111 & 0111 & 0111 \\ \hline
				6 & 0110 & 0110 & 0110 \\ \hline
				5 & 0101 & 0101 & 0101 \\ \hline
				4 & 0100 & 0100 & 0100 \\ \hline
				3 & 0011 & 0011 & 0011 \\ \hline
				2 & 0010 & 0010 & 0010 \\ \hline
				1 & 0001 & 0001 & 0001 \\ \hline
				\multirow{2}{*}{0} & 0000 & 0000 & \multirow{2}{*}{0000} \\
				& 1000 & 1111 & \\ \hline
				-1 & 1001 & 1110 & 1111 \\ \hline
				-2 & 1010 & 1101 & 1110 \\ \hline
				-3 & 1011 & 1100 & 1101 \\ \hline
				-4 & 1100 & 1011 & 1100 \\ \hline
				-5 & 1101 & 1010 & 1011 \\ \hline
				-6 & 1110 & 1001 & 1010 \\ \hline
				-7 & 1111 & 1000 & 1001 \\ \hline
				-8 & N/A & N/A & 1000 \\ \hline
			\end{tabular}
		\end{center}
	}
\end{enumerate}
\end{Q}



\begin{Q}
\fr{En utilisant les axiomes prouver les théorèmes :}
\en{Proove the theorems using the axioms.}

\begin{minipage}[t]{0.5\linewidth}
	\centering Axiomes :
	$$A1a : x \in B, y \in B \rightarrow x+y \in B$$
	$$A1b : x \in B, y \in B \rightarrow x \cdot y \in B$$
	$$A2a : x+0 = x$$
	$$A2b : x \cdot 1 = x$$
	$$A3a : x \cdot (y+z) = x \cdot y+x \cdot z$$
	$$A3b : x+y \cdot z = (x+y)(x+z)$$
	$$A4a : x+y = y+x$$
	$$A4b : x \cdot y = y \cdot x$$
	$$A5a : x+\overline{x} = 1$$
	$$A5b : x \cdot \overline{x} = 0$$
	$$A6 : \exists x, y \in B : x \neq y$$
\end{minipage}
\begin{minipage}[t]{0.5\linewidth}
	\centering Théorèmes :
	$$T1a : x+x=x$$
	$$T1b : x \cdot x=x$$
	$$T2a : x+1=1$$
	$$T2b : x \cdot 0=0$$
	$$T3a : x+y \cdot x=x$$
	$$T3b : x \cdot (x+y)=x$$
	$$T4 : \overline{(\overline{x})}=x$$
\end{minipage}

\reponse{
\begin{itemize}
	\item $T1a : x+x=x$
	\begin{align*}
		x + x & = (x + x) \cdot 1&\mbox{, A2b}\\
		& = (x + x) \cdot (x + \overline{x})&\mbox{, A5a}\\
		& = x + x \cdot \overline{x}&\mbox{, A3b}\\
		& = x &\mbox{, A5b}
	\end{align*}

	\item $T1b : x \cdot x=x$
	\begin{align*}
		x \cdot x & = x \cdot x  + 0&\mbox{, A2a}\\
		& = x\cdot x + x \cdot \overline{x} &\mbox{, A5b}\\
		& = x \cdot (x + \overline{x}) &\mbox{, A3a} \\
		& = x \cdot 1&\mbox{, A5a}\\
		& = x&\mbox{, A2b}
	\end{align*}

	\item $T2a : x+1=1$
	\begin{align*}
		x + 1 & = (x+1) \cdot 1&\mbox{, A2b}\\
		& = (x+1) \cdot (x+\overline{x})&\mbox{, A5a}\\
		& = x + 1 \cdot \overline{x} &\mbox{, A3b}\\
		& = x + \overline{x} &\mbox{, A2b}\\
		& = 1 &\mbox{, A5a}
	\end{align*}

	\item $T2b : x \cdot 0=0$
	\begin{align*}
		x \cdot 0 & = x \cdot 0 + 0 &\mbox{, A2a} \\
		& = x \cdot 0 + x \cdot \overline{x}&\mbox{, A5b} \\
		& = x \cdot (0 + \overline{x}) &\mbox{, A3a} \\
		& = x \cdot (\overline{x}) &\mbox{, A2a} \\
		& = 0 &\mbox{, A5b}
	\end{align*}

	\item $T3a : x+y \cdot x=x$
	\begin{align*}
		x + x \cdot y & = x \cdot 1 + x \cdot y &\mbox{, A2b}\\
		& = x \cdot (1 + y)&\mbox{, A3a}\\
		& = x \cdot 1&\mbox{, T2a} \\
		& = x &\mbox{, A2b}
	\end{align*}

	\item $T3b : x \cdot (x+y)=x$
	\begin{align*}
		x \cdot (x+y) & = (x+0) \cdot (x+y) &\mbox{, A2a}\\
		& = x + 0 \cdot y &\mbox{, A3b}\\
		& = x + 0 &\mbox{, T2b} \\
		& = x &\mbox{, A2a}
	\end{align*}

	\item $T4 : \overline{(\overline{x})}=x$
	\begin{align*}
		\overline{(\overline{x})} & = \overline{(\overline{x})} \cdot 1 &\mbox{, A2b}\\
		& = \overline{(\overline{x})} \cdot (x + \overline{x}) &\mbox{, A5a}\\
		& = \overline{(\overline{x})} \cdot x + \overline{(\overline{x})} \cdot \overline{x} &\mbox{, A3a}\\
		& = \overline{(\overline{x})} \cdot x + 0 &\mbox{, A5b}\\
		& = \overline{(\overline{x})} \cdot x + x \cdot \overline{x} &\mbox{, A5b}\\
		& = x\cdot (\overline{(\overline{x})} + \overline{x})&\mbox{, A5a}\\
		& = x&\mbox{, A5a}\\
	\end{align*}
\end{itemize}
}

\end{Q}
%
% \begin{Q}
% 	L'approximation I(V) idéalisée avec seulement la tension de seuil est-elle une bonne approximation ?
% 	\label{Q:1}
% 	\reponse{Oui}%R
% \end{Q}


% \begin{figure}[H]
% 	\begin{center}
% 		\begin{circuitikz}\draw
% 			(0,0) node[anchor=east] {A} to [short,i>^=$I$] (1.5,0)
% 			(0,0) to [sDo, v=$V$] (2.5,0) node [anchor=west]{K}
% 		;\end{circuitikz}
% 	\end{center}
% \caption{Conventions électriques}
% \label{fig:zener_conv}
% \end{figure}

\end{document}
"
%
%%%%%%%%%%%%%%%%%%%%%%%%%%%%%%%%%%%%%%%%%%%%%%%%%%%%%%%%%%%%%%%%%%%%%%%%%%%%%%%%%%%%%%%%%%%

%Numero du TP :
\def \tpnumber {TP 6 }

\newcommand{\version}{v1.0.0}


% ########     #####      #####    ##         #########     ###     ##      ##  
% ##     ##  ##     ##  ##     ##  ##         ##           ## ##    ###     ##  
% ##     ##  ##     ##  ##     ##  ##         ##          ##   ##   ## ##   ##  
% ########   ##     ##  ##     ##  ##         ######     ##     ##  ##  ##  ##  
% ##     ##  ##     ##  ##     ##  ##         ##         #########  ##   ## ##  
% ##     ##  ##     ##  ##     ##  ##         ##         ##     ##  ##     ###  
% ########     #####      #####    #########  #########  ##     ##  ##      ##  
\newboolean{corrige}
\ifx\correction\undefined
\setboolean{corrige}{false}% pas de corrigé
\else
\setboolean{corrige}{true}%corrigé
\fi

\newboolean{fr}
\ifx\french\undefined
\setboolean{fr}{false}% pas de corrigé
\else
\setboolean{fr}{true}%corrigé
\fi

\newboolean{en}
\ifx\english\undefined
\setboolean{en}{false}% pas de corrigé
\else
\setboolean{en}{true}%corrigé
\fi

%\setboolean{corrige}{false}% pas de corrigé

\definecolor{darkblue}{rgb}{0,0,0.5}

\pdfinfo{
/Author (ULB -- BEAMS)
/Title (\tpnumber, ELEC-H-310)
/ModDate (D:\pdfdate)
}

\hypersetup{
pdftitle={\tpnumber [ELEC-H-310] Choucroute numérique},
pdfauthor={ULB -- BEAMS},
pdfsubject={}
}

\theoremstyle{definition}% questions pas en italique
\newtheorem{Q}{Question}[] % numéroter les questions [section] ou non []


%  ######     #####    ##       ##  ##       ##     ###     ##      ##  ######      #######   
% ##     ##  ##     ##  ###     ###  ###     ###    ## ##    ###     ##  ##    ##   ##     ##  
% ##         ##     ##  ## ## ## ##  ## ## ## ##   ##   ##   ## ##   ##  ##     ##  ##         
% ##         ##     ##  ##  ###  ##  ##  ###  ##  ##     ##  ##  ##  ##  ##     ##   #######   
% ##         ##     ##  ##       ##  ##       ##  #########  ##   ## ##  ##     ##         ##  
% ##     ##  ##     ##  ##       ##  ##       ##  ##     ##  ##     ###  ##    ##   ##     ##  
%  #######     #####    ##       ##  ##       ##  ##     ##  ##      ##  ######      #######   
\newcommand{\reponse}[1]{% pour intégrer une réponse : \reponse{texte} : sera inclus si \boolean{corrige}
	\ifthenelse {\boolean{corrige}} {\fr{\paragraph{Réponse :}}\en{\paragraph{Answer:}} \color{darkblue}   #1\color{black}} {}
 }

 \newcommand{\fr}[1]{
 	\ifthenelse {\boolean{fr}} {#1} {}
 }

 \newcommand{\en}[1]{
 	\ifthenelse {\boolean{en}} {#1} {}
 }

\newcommand{\addcontentslinenono}[4]{\addtocontents{#1}{\protect\contentsline{#2}{#3}{#4}{}}}

\newlength{\gvs}% Gate Vertical Space
\gvs=6em
\newlength{\ghs}% Gate Horizontal Space
\ghs=10em

\newcommand\encircle[1]{%http://tex.stackexchange.com/questions/123924/indexed-letters-inside-circles
  \tikz[baseline=(X.base)] 
    \node (X) [draw, shape=circle, inner sep=0] {\strut #1};}


%  #######   ##########  ##    ##   ##         #########  
% ##     ##      ##       ##  ##    ##         ##         
% ##             ##        ####     ##         ##         
%  #######       ##         ##      ##         ######     
%        ##      ##         ##      ##         ##         
% ##     ##      ##         ##      ##         ##         
%  #######       ##         ##      #########  #########  
%% fancy header & foot
\pagestyle{fancy}
\lhead{[ELEC-H-310] \fr{Électronique numérique}\en{Digital Electronics}\\ \tpnumber}
\rhead{\version\\ page \thepage}
\chead{\ifthenelse{\boolean{corrige}}{\fr{Corrigé}\en{Correction}}{}}
\cfoot{}
%%

\setlength{\parskip}{0.2cm plus2mm minus1mm} %espacement entre §
\setlength{\parindent}{0pt}

% ##########  ########   ##########  ##         #########  
%     ##         ##          ##      ##         ##         
%     ##         ##          ##      ##         ##         
%     ##         ##          ##      ##         ######     
%     ##         ##          ##      ##         ##         
%     ##         ##          ##      ##         ##         
%     ##      ########       ##      #########  ######### 
\date{\vspace{-1.7cm}\version}
\title{\vspace{-2cm} \tpnumber\\ \fr{Électronique numérique [ELEC-H-310] \ifthenelse{\boolean{corrige}}{~\\Corrigé}{}}%
\en{Digital Electronics [ELEC-H-310] \ifthenelse{\boolean{corrige}}{~\\Correction}{}}%
}






\begin{document}
\maketitle
\vspace*{1cm}

\begin{Q}
	\fr{Construire la table des conditions d'équivalence pour la table de Huffman suivante (représentant une machine de Moore).}
	\en{Build the table of equivalence conditions for the following Huffman table (representing a Moore machine).}
	\begin{center}
		\begin{tabular}{|l|l|l|l|l|l|} \hline
			& \multicolumn{4}{c|}{$ab$} & $Z_1Z_2$ \\ \hline
			1 & \textbf{1} & 2 & 3 & 10 & 00 \\ \hline
			2 & 7 & \textbf{2} & 9 & 4 & 11 \\ \hline
			3 & 7 & 6 & \textbf{3} & 4 & 10 \\ \hline
			4 & 7 & 2 & 5 & \textbf{4} & 01 \\ \hline
			5 & 11 & 8 & \textbf{5} & 12 & 10 \\ \hline
			6 & 11 & \textbf{6} & 5 & 10 & 11 \\ \hline
			7 & \textbf{7} & 8 & 5 & 4 & 00 \\ \hline
			8 & 1 & \textbf{8} & 3 & 4 & 11 \\ \hline
			9 & 7 & 8 & \textbf{9} & 12 & 10 \\ \hline
			10 & 11 & 6 & 9 & \textbf{10} & 01 \\ \hline
			11 & \textbf{11} & 8 & 9 & 10 & 00 \\ \hline
			12 & 7 & 2 & 5 & \textbf{12} & 11 \\ \hline
		\end{tabular}
	\end{center}
	\reponse{
		\begin{center}
			\begin{tabular}{|c|c|c|c|c|c|c|c|c|c|c|c|} \hline%\cellcolor{green!25}
				2 & x & \cellcolor{gray!20} & \cellcolor{gray!20} & \cellcolor{gray!20} & \cellcolor{gray!20} & \cellcolor{gray!20} & \cellcolor{gray!20} & \cellcolor{gray!20} & \cellcolor{gray!20} & \cellcolor{gray!20} & \cellcolor{gray!20}\\ \hline
				3 & x & x & \cellcolor{gray!20} & \cellcolor{gray!20} & \cellcolor{gray!20} & \cellcolor{gray!20} & \cellcolor{gray!20} & \cellcolor{gray!20} & \cellcolor{gray!20} & \cellcolor{gray!20} & \cellcolor{gray!20}\\ \hline
				4 & x & x & x & \cellcolor{gray!20} & \cellcolor{gray!20} & \cellcolor{gray!20} & \cellcolor{gray!20} & \cellcolor{gray!20} & \cellcolor{gray!20} & \cellcolor{gray!20} & \cellcolor{gray!20}\\ \hline
				\multirow{3}{*}{5} & \multirow{3}{*}{x} & \multirow{3}{*}{x} & \cellcolor{red!25}7-11; & \multirow{3}{*}{x} & \cellcolor{gray!20} & \cellcolor{gray!20} & \cellcolor{gray!20} & \cellcolor{gray!20} & \cellcolor{gray!20} & \cellcolor{gray!20} & \cellcolor{gray!20}\\
				 &  &  & \cellcolor{red!25}6-8; &  & \cellcolor{gray!20} &  \cellcolor{gray!20}&  \cellcolor{gray!20}& \cellcolor{gray!20} & \cellcolor{gray!20} & \cellcolor{gray!20} & \cellcolor{gray!20}\\
				 &  &  & \cellcolor{red!25}4-12 &  & \cellcolor{gray!20} &  \cellcolor{gray!20}& \cellcolor{gray!20} & \cellcolor{gray!20} & \cellcolor{gray!20} & \cellcolor{gray!20} & \cellcolor{gray!20}\\ \hline
				% 5 & x & x & \cellcolor{red!25}7-11;\\6-8; & x &  &  &  &  &  &  & \\ \hline
				\multirow{3}{*}{6} & \multirow{3}{*}{x} & \cellcolor{green!25}7-11; & \multirow{3}{*}{x} & \multirow{3}{*}{x} & \multirow{3}{*}{x} & \cellcolor{gray!20} &\cellcolor{gray!20}  & \cellcolor{gray!20} & \cellcolor{gray!20} & \cellcolor{gray!20} & \cellcolor{gray!20}\\
				 &  & \cellcolor{green!25}5-9; &  &  &  & \cellcolor{gray!20} &  \cellcolor{gray!20}&  \cellcolor{gray!20}& \cellcolor{gray!20} & \cellcolor{gray!20} & \cellcolor{gray!20}\\
				 &  & \cellcolor{green!25}4-10 &  &  &  & \cellcolor{gray!20} & \cellcolor{gray!20} &\cellcolor{gray!20}  & \cellcolor{gray!20} & \cellcolor{gray!20} &\cellcolor{gray!20}\\ \hline
				\multirow{3}{*}{7} & \cellcolor{red!25}2-8; & \multirow{3}{*}{x} & \multirow{3}{*}{x} & \multirow{3}{*}{x} & \multirow{3}{*}{x} & \multirow{3}{*}{x} & \cellcolor{gray!20} &  \cellcolor{gray!20}&  \cellcolor{gray!20}& \cellcolor{gray!20} & \cellcolor{gray!20}\\
				 & \cellcolor{red!25}4-10 &  & & &  &  & \cellcolor{gray!20} & \cellcolor{gray!20} & \cellcolor{gray!20} & \cellcolor{gray!20} & \cellcolor{gray!20}\\
				 & \cellcolor{red!25}3-5 & & & &  &  & \cellcolor{gray!20} &\cellcolor{gray!20}  & \cellcolor{gray!20} & \cellcolor{gray!20} &\cellcolor{gray!20} \\ \hline
				\multirow{3}{*}{8} & \multirow{3}{*}{x}  & \cellcolor{red!25}1-7; & \multirow{3}{*}{x}  & \multirow{3}{*}{x}  & \multirow{3}{*}{x}  & \cellcolor{red!25}7-11; & \multirow{3}{*}{x}  & \cellcolor{gray!20} &  \cellcolor{gray!20}& \cellcolor{gray!20} & \cellcolor{gray!20}\\
				& x & \cellcolor{red!25}3-9 & & & & \cellcolor{red!25}3-5; & & \cellcolor{gray!20} &\cellcolor{gray!20}  & \cellcolor{gray!20} & \cellcolor{gray!20}\\
				&  & \cellcolor{red!25} & & & & \cellcolor{red!25}4-10 & &  \cellcolor{gray!20}& \cellcolor{gray!20} &  \cellcolor{gray!20}& \cellcolor{gray!20}\\ \hline
				\multirow{2}{*}{9} & \multirow{2}{*}{x} & \multirow{2}{*}{x} & \cellcolor{red!25}6-8; & \multirow{2}{*}{x} & \cellcolor{green!25}7-11; & \multirow{2}{*}{x} & \multirow{2}{*}{x} & \multirow{2}{*}{x} & \cellcolor{gray!20} &  \cellcolor{gray!20}& \cellcolor{gray!20}\\
				 &  &  & \cellcolor{red!25}4-12 &  & \cellcolor{green!25} &  &  &  &  \cellcolor{gray!20}&  \cellcolor{gray!20}&\cellcolor{gray!20}\\ \hline
				\multirow{3}{*}{10} & \multirow{3}{*}{x} & \multirow{3}{*}{x} & \multirow{3}{*}{x} & \cellcolor{green!25}7-11; & \multirow{3}{*}{x} & \multirow{3}{*}{x} & \multirow{3}{*}{x} & \multirow{3}{*}{x} & \multirow{3}{*}{x} & \cellcolor{gray!20} & \cellcolor{gray!20}\\
				 &  &  &  & \cellcolor{green!25}5-9; &  &  &  &  &  & \cellcolor{gray!20} & \cellcolor{gray!20}\\
				 &  &  &  & \cellcolor{green!25}2-6; &  &  &  &  &  & \cellcolor{gray!20} & \cellcolor{gray!20}\\ \hline
				\multirow{2}{*}{11} & \cellcolor{red!25}2-8; & \multirow{2}{*}{x} & \multirow{2}{*}{x} & \multirow{2}{*}{x} & \multirow{2}{*}{x} & \multirow{2}{*}{x} & \cellcolor{green!25}5-9; & \multirow{2}{*}{x} & \multirow{2}{*}{x} & \multirow{2}{*}{x} & \cellcolor{gray!20}\\
				 & \cellcolor{red!25}3-9 &  &  &  &  &  & \cellcolor{green!25}4-10; &  &  &  & \cellcolor{gray!20}\\ \hline
				12 & \multirow{4}{*}{x} & \cellcolor{red!25}5-9; & \multirow{4}{*}{x} & \multirow{4}{*}{x} & \multirow{4}{*}{x} & \cellcolor{red!25}11-7; & \multirow{4}{*}{x} & \cellcolor{red!25}1-7; & \multirow{4}{*}{x} & \multirow{4}{*}{x} & \multirow{4}{*}{x} \\
				 &  & \cellcolor{red!25}4-12 &  &  &  & \cellcolor{red!25}10-12; &  & \cellcolor{red!25}2-8; &  &  &  \\
				 &  & \cellcolor{red!25} &  &  &  & \cellcolor{red!25}2-6 &  & \cellcolor{red!25}3-5; &  &  &  \\
				 &  & \cellcolor{red!25} &  &  &  & \cellcolor{red!25} &  & \cellcolor{red!25}4-12 &  &  &  \\ \hline
				 & 1 & 2 & 3 & 4 & 5 & 6 & 7 & 8 & 9 & 10 & 11 \\ \hline
			\end{tabular}
		\end{center}
	}
\end{Q}













\begin{Q}
	\fr{Si on permet ou non l'équivalence de lignes dont la sortie est différente, montrer les deux automates différents (équations logiques et logigrammes) auxquels on parviendrait dans le cas de la table non complètement réduite suivante~:}
	\en{From this Huffman table, show the two possible automatons if we allow to merge lines for which the output is
	\begin{enumerate}
		\item different
		\item the same
	\end{enumerate}
	}

	\begin{center}

	\begin{tabular}{|l|l|l|l|l|l|l|}
	\hline
	$Y_1$ $Y_2$ & 00         & 01         & 11         & 10         & ab & Z \\ \hline
	1           & \textbf{1} & 2          & 3          & 4          &    & 0 \\ \cline{1-5} \cline{7-7}
	2           & -          & \textbf{2} & \textbf{2} & \textbf{2} &    & 1 \\ \cline{1-5} \cline{7-7}
	3           & 1          & 2          & \textbf{3} & -          &    & 1 \\ \cline{1-5} \cline{7-7}
	4           & \textbf{4} & -          & 2          & \textbf{4} &    & 1 \\ \cline{1-5} \cline{7-7}
	\end{tabular}

	\end{center}

	\fr{Comment appelle-t-on ces deux types de machines ?}
	\en{What are those two automatons called?}

\reponse{
	\begin{enumerate}
		\item \fr{Si on ne permet pas l'équivalence des états dont la sortie est différente, on est dans le cas d'une machine de Moore.

		La table d'équivalence correspondante est donc~:}
		\en{If we cannot merge states with different outputs, we talk about a Moore machine.

		The equivalence table thus is:}
		\begin{center}
			\begin{tabular}{cccc} \cline{1-2}
				\multicolumn{1}{|c|}{2}	& 						\multicolumn{1}{c|}{\cellcolor{red!25}x} 					& 																	& \\ \cline{1-3}
				\multicolumn{1}{|c|}{3}	& 						\multicolumn{1}{c|}{\cellcolor{red!25}x} 					& \multicolumn{1}{c|}{\cellcolor{green!25}OK}						& \\ \cline{1-4}
				\multicolumn{1}{|c|}{\multirow{2}{*}{4}} & 		\multicolumn{1}{c|}{\cellcolor{red!25}}						& \multicolumn{1}{c|}{\cellcolor{green!25}} 						& \multicolumn{1}{c|}{\cellcolor{red!25}2-3}\\
				\multicolumn{1}{|c|}{} & 						\multicolumn{1}{c|}{\multirow{-2}{*}{\cellcolor{red!25}x}} & \multicolumn{1}{c|}{\multirow{-2}{*}{\cellcolor{green!25}OK}} 	& \multicolumn{1}{c|}{\cellcolor{red!25}1-4} \\ \cline{1-4}
			 & 													\multicolumn{1}{|c|}{1} & 									  \multicolumn{1}{c|}{2} 											& \multicolumn{1}{c|}{3} \\ \cline{2-4}
			\end{tabular}
		\end{center}

		\fr{On peut choisir de fusionner les états 2 et 3, ou les états 2 et 4.
		Si on choisit de fusionner les états 2 et 3, on obtient la table de Huffman réduite suivante~:}
		\en{We can merge states 2 and 3, or 2 and 4.
		By merging 2 and 3, we get:}
		\begin{center}
			\begin{tabular}{|l|l|l|l|l|l|l|}
			\hline
			$Y_1$ $Y_2$ & 00         & 01         & 11         & 10         & ab & Z \\ \hline
			1           & \textbf{1} & 2          & 2          & 4          &    & 0 \\ \cline{1-5} \cline{7-7}
			2           & 1          & \textbf{2} & \textbf{2} & \textbf{2} &    & 1 \\ \cline{1-5} \cline{7-7}
			4           & \textbf{4} & -          & 2          & \textbf{4} &    & 1 \\ \cline{1-5} \cline{7-7}
			\end{tabular}
		\end{center}

		\fr{Il faut ensuite choisir un système de codage pour les états afin d'en déduire les fonctions logiques.
		Prenons $1$, $2$ et $4$ codés respectivement $00$, $01$ et $11$\footnote{Puisque le code $10$ n'est pas utilisé, cet état n'existe pas et toutes ses valeurs sont remplacées par des \textit{don't care}.}.
		On obtient alors la même table de Huffman de laquelle on peut déduire des k-maps et les fonctions logiques du système.}
		\en{The next step is to choose a coding for the states.
		Let's take $1$, $2$ and $4$ coded $00$, $01$ and $11$\footnote{Since the code $10$ is not used, this state does not exist and all it's values are replaced with \textit{don't care}.}.}

		\begin{center}
			\begin{tabular}{|l|l|l|l|l|l|l|}
			\hline
			$Y_1$ $Y_2$ & 00         & 01         & 11         & 10         & ab & Z \\ \hline
			00           & \textbf{00} & 01          & 01          & 11          &    & 0 \\ \cline{1-5} \cline{7-7}
			01           & 00          & \textbf{01} & \textbf{01} & \textbf{01} &    & 1 \\ \cline{1-5} \cline{7-7}
			11           & \textbf{11} & -          & 01          & \textbf{11} &    & 1 \\ \cline{1-5} \cline{7-7}
			10           & -           & -          & -           & -           &    & - \\ \cline{1-5} \cline{7-7}
			\end{tabular}

			\askmapiv{$Y_1 = y_1\overline{b} + \overline{y_2}a\overline{b}$}{a b $y_1$ $y_2$}{}{00-100--10-100-0}{%
			\color{red}\put(0.1,0.1){\dashbox{0.3}(0.8,1.8){}}%
			\color{red}\put(3.1,0.1){\dashbox{0.3}(0.8,1.8){}}%
			\color{blue}\put(3.2,3.2){\dashbox{0.1}(0.6,0.6){}}%
			\color{blue}\put(3.2,0.2){\dashbox{0.1}(0.6,0.6){}}%
			}

			\askmapiv{$Y_2 = y_1 + a + b$}{a b $y_1$ $y_2$}{}{00-111--11-111-1}{%
			\color{red}\put(0.1,0.1){\dashbox{0.3}(3.8,1.8){}}%
			\color{green}\put(1.1,0.2){\dashbox{0.2}(1.8,3.7){}}%
			\color{blue}\put(2.1,0.3){\dashbox{0.1}(1.7,3.5){}}%
			}

			\askmapiv{$Z = y_2$}{a b $y_1$ $y_2$}{}{0--1-1---1-1-1-1}{%
			\color{red}\put(0.1,1.1){\dashbox{0.1}(3.8,1.8){}}%
			}
		\end{center}

		\item 
		\fr{Si on permet l'équivalence des états dont la sortie est différente, on est dans le cas d'une machine de Mealy.
		Cependant, une restriction reste toujours de mise~: la fusion d'états futurs stables dont la sortie est différente est interdite.
		Considérons par exemple les états $1$ et $4$. Pour l'entrée $ab = 00$, les deux états futurs (respectivement $1$ et $4$) sont stables et leur sortie vaut respectivement $0$ et $1$, empêchant la fusion de ces deux états.
		Par contre, si l'on considère les états $2$ et $3$ pour les entrées $ab = 11$, les deux états futurs sont stables, mais ayant la même sortie, la fusion est potentiellement possible.

		Forts de ces considérations, nous pouvons dresser la table d'équivalence suivante~:}
		\en{If we allow merging states with different outputs, we are working with a Mealy machine.
		However, we need to keep in mind that with still cannot merge stable states with different outputs.
		For example, let's take a look at states $1$ and $4$. For the input $ab = 00$, both future states are stable and their output is $0$ and $1$. Since the output are different for stable future states, we cannot merge $1$ and $4$.
		On the other hand, if we consider states $2$ and $3$ for the input $ab = 11$, both futur states are stable, but with the same output.

		From this, we can build an equivalence table:}
		\begin{center}
			\begin{tabular}{cccc} \cline{1-2}
				\multicolumn{1}{|c|}{\multirow{2}{*}{2}}	&	\multicolumn{1}{c|}{\cellcolor{green!25}2-3} 					& 																	& \\
				\multicolumn{1}{|c|}{}						&	\multicolumn{1}{c|}{\cellcolor{green!25}2-4}					& 																	& \\ \cline{1-3}
				\multicolumn{1}{|c|}{3}	& 						\multicolumn{1}{c|}{\cellcolor{green!25}OK}						& \multicolumn{1}{c|}{\cellcolor{green!25} OK}						& \\ \cline{1-4}
				\multicolumn{1}{|c|}{\multirow{2}{*}{4}} & 		\multicolumn{1}{c|}{\cellcolor{red!25}}							& \multicolumn{1}{|c|}{\cellcolor{green!25}} 						& \multicolumn{1}{c|}{\cellcolor{red!25}2-3}\\
				\multicolumn{1}{|c|}{} & 						\multicolumn{1}{c|}{\multirow{-2}{*}{\cellcolor{red!25}x}} 	& \multicolumn{1}{|c|}{\multirow{-2}{*}{\cellcolor{green!25}OK}} 	& \multicolumn{1}{c|}{\cellcolor{red!25}1-4} \\ \cline{1-4}
			 & 													\multicolumn{1}{|c|}{1} & 										\multicolumn{1}{c|}{2} 											& \multicolumn{1}{c|}{3} \\ \cline{2-4}
			\end{tabular}
		\end{center}

		\fr{Fusionnons les états $1$ et $3$ dans l'état $1$, et les états $2$ et $4$ dans l'état $2$~:}
		\en{Let's merge states $1$ and $3$ into $1$, and $2$ and $4$ into $2$:}
		\begin{center}
		\begin{tabular}{|l|l|l|l|l|l|l|}
		\hline
		$Y$ & 00         & 01         & 11         & 10         & ab & Z \\ \hline
		1           & \textbf{1} & 2          & \textbf{1}          & 2          &    & ? \\ \cline{1-5} \cline{7-7}
		2           & \textbf{2}          & \textbf{2} & \textbf{2} & \textbf{2} &    & ? \\ \cline{1-5} \cline{7-7}
		\end{tabular}
		\end{center}

		\fr{La sortie dépendant maintenant de l'état ainsi que des entrées, définir une sortie pour l'état tout entier n'a plus de sens.
		Nous allons donc assigner une sortie pour chaque état futur.
		Les états futurs fusionnés comportant au moins un état stable récupèrent la valeur de la sortie de l'état stable en question\footnote{On voit ici pourquoi fusionner deux états futurs stables ayant des sorties différentes est interdit~; on ne saurait pas la sortie de quel état stable récupérer.}.
		Par exemple, lors de la fusion des états $1$ et $3$ pour les entrées $ab = 00$, nous avons fusionné un état futur «~$1$~» stable dont la sortie valait $1$ avec un état instable «~$1$~» dont la sortie valait $0$.
		La sortie de l'état futur fusionné vaut donc $1$.

		La sortie des états futurs résultant de la fusion d'états futurs instables est déterminée de sorte à éviter les aléas, comme nous l'avons déjà fait dans les séances précédentes.
		Par exemple, la transition de l'état $1$ vers l'état $2$ lorsque les entrées $ab$ passent de $11$ à $01$ conserve la même sortie~: $1$.
		La sortie de la transition est donc aussi fixée à $1$ afin d'éviter un \textit{glitch}. La table~\ref{tab:sortie-transition} reprend les règles à observer dans de pareilles situations.}

		\en{Now, the output depends on the input and the current state.
		We thus need to define an output for each future state.
		Merged future states inheritate the output of their original state.

		Output of merged states that were originally unstable is set so that we avoid glitches.
		The table~\ref{tab:sortie-transition} summarizes the rules to set the outputof ``unstable" states.}
		\begin{table}[H]
			\centering
			\begin{tabular}{ccc}
			$Z_p$ & $Z_f$ & $Z_t$ \\ \hline
			0 & 0 & 0 \\
			0 & 1 & - \\
			1 & 0 & - \\
			1 & 1 & 1 \\
			\end{tabular}
			\fr{\caption{Valeur de la sortie lors de la transition entre états, avec $Z_p$ la sortie de l'état présent, $Z_f$ la sortie de l'état futur et $Z_t$ la sortie de la transition.}}
			\en{\caption{Output of the transition state $Z_t$ when we start at $Z_p$ and end up at $Z_f$.}}
			\label{tab:sortie-transition}
		\end{table}


		\fr{On obtient alors la table de Huffman réduite suivante pour la machine de Mealy~:}
		\en{This way, we obtain this Huffman table:}

		\begin{center}
		\begin{tabular}{|l|l|l|l|l|l}
		\hline
		$Y$ & 00         & 01         & 11         & 10         & \multicolumn{1}{l|}{ab} \\ \hline
		1           & \textbf{1/0} & 2/1          & \textbf{1/1}          & 2/1          &  \\ \cline{1-5}
		2           & \textbf{2/1}          & \textbf{2/1} & \textbf{2/1} & \textbf{2/1} &  \\ \cline{1-5}
		\end{tabular}
		\end{center}

		\fr{En codant l'état $1$ par $0$ et l'état $2$ par $1$, on peut obtenir les k-maps et fonctions logiques suivantes~:}
		\en{By coding $1$ with $0$ and $2$ with $1$, we obtain the following:}
		\begin{center}
		\askmapiii{$Y = y + \overline{a}b + a\overline{b}$}{a b y}{}{01111101}{%
			\color{red}\put(0.1,0.1){\dashbox{0.2}(3.8,0.8){}}%
			\color{blue}\put(1.1,0.2){\dashbox{0.1}(0.8,1.7){}}%
			\color{green}\put(3.1,0.2){\dashbox{0.1}(0.7,1.7){}}%
			}

		\askmapiii{$Z = y + a + b$}{a b y}{}{01111111}{%
			\color{red}\put(0.1,0.1){\dashbox{0.2}(3.8,0.8){}}%
			\color{blue}\put(1.1,0.2){\dashbox{0.1}(1.8,1.7){}}%
			\color{green}\put(2.1,0.3){\dashbox{0.1}(1.6,1.5){}}%
			}
		\end{center}



	\end{enumerate}
}
\end{Q}












\begin{Q}
	\fr{Dans la table des états d'Huffmann suivante:}
	\en{In the following Huffman table:}
	\begin{center}
		\begin{tabular}{|l|l|l|l|l|l}
		\hline
		 $Y_1$ $Y_2$ & 00         & 01         & 11         & 10         & \multicolumn{1}{l|}{ab} \\ \hline
		1 & \textbf{1} & \textbf{1} & 3          & -          &    \\ \cline{1-5}
		2 & \textbf{2} & \textbf{2} & 3          & 4          &    \\ \cline{1-5}
		3 & 2          & 1          & \textbf{3} & \textbf{3} &    \\ \cline{1-5}
		4 & 1          & -          & \textbf{4} & \textbf{4} &    \\ \cline{1-5}
		\end{tabular}

	\end{center}

	\begin{enumerate}
		\item \fr{Trouver un codage n'offrant pas de problèmes de course. Déterminer les fonctions de
		rétroaction $Y_1$ et $Y_2$.}
		\en{Find a coding without race problems. Write the retroaction functions $Y_1$ and $Y_2$.}


		\reponse{
			\fr{Un problème de course survient lorsque la machine doit passer dans un nouvel état dans lequel plus d'une fonction de rétroaction change.
			Par exemple, lorsque l'on passe de l'état $00$ à l'état $11$, $y_1$ et $y_2$ changent de valeur.
			Or, il se peut que l'un des deux signaux soit plus lent que l'autre, entraînant une transition vers un état non désiré.

			Pour éviter ces problèmes, il suffit parfois de choisir un codage des états approprié.
			Dans notre cas, si l'on choisit de coder l'état $1$ par $00$, il faut obligatoirement coder l'état $3$ par un codage ayant une distance de Hamming de 1 à cause de la transition $1 \rightarrow 3$.
			Choisissons de coder $3$ par $01$.

			En suivant la même réflexion pour les deux autres états, on peut trouver la table de Huffman codée suivante~:}

			\en{A race problem happens when the machine needs to change more than one retroaction variable to transition.
			For example, when we go from $00$ to $11$, both $y_1$ and $y_2$ change.
			But if both signals do not travel at the same speed, we could end up in the wrong state.

			To avoid this, it may be enough to choose an appropriate coding.
			For example,  if we code $1$ by $00$, we need to code the state $3$ so that it is at a Hamming distance of 1 from $1$. Let's say we replace $3$ by $01$.

			In the end, we can get this coded Huffman table:}
			\begin{center}
				\begin{tabular}{|l|l|l|l|l|l}
				\hline
				 $Y_1$ $Y_2$ & 00         & 01         & 11         & 10         & \multicolumn{1}{l|}{ab} \\ \hline
				00 & \textbf{00} & \textbf{00} & 01          & -          &    \\ \cline{1-5}
				11 & \textbf{11} & \textbf{11} & 01          & 10          &    \\ \cline{1-5}
				01 & 11          & 00          & \textbf{01} & \textbf{01} &    \\ \cline{1-5}
				10 & 00         & -          & \textbf{10} & \textbf{10} &    \\ \cline{1-5}
				\end{tabular}
			\end{center}

			\fr{Attention~: lors de l'établissement des k-maps, il faut veiller à réordonner la table pour respecter une distance de Hamming de 1 entre les cases voisines.}
			\en{Beware: when extracting the K-map from the Huffman table, don't forget to sort the states so that all neighbouring lines are at a Hamming distance of 1 from each other.}

			\begin{center}
				\askmapiv{$Y_1 = y_2\overline{ab} + y_1\overline{a}b + y_1\overline{y_2}a + y_1a\overline{b}$}{a b $y_1$ $y_2$}{}{010100-1-0110010}{%
				\color{red}\put(0.1,1.1){\dashbox{0.1}(0.8,1.8){}}%
				\color{green}\put(1.1,0.1){\dashbox{0.1}(0.8,1.8){}}%
				\color{blue}\put(2.1,0.1){\dashbox{0.1}(1.8,0.8){}}%
				\color{orange}\put(3.1,0.2){\dashbox{0.2}(0.7,1.7){}}%
				}

				\askmapiv{$Y_2 = y_2\overline{ab} + y_1\overline{a}b + y_2ab + \overline{y_1}a$}{a b $y_1$ $y_2$}{}{010100-1-1001101}{%
				\color{red}\put(0.1,1.1){\dashbox{0.1}(0.8,1.8){}}%
				\color{green}\put(1.1,0.1){\dashbox{0.1}(0.8,1.8){}}%
				\color{blue}\put(2.1,1.1){\dashbox{0.1}(0.8,1.8){}}%
				\color{orange}\put(2.2,2.1){\dashbox{0.2}(1.7,1.8){}}%
				}
			\end{center}


		}
		\item \fr{Résoudre les problèmes de course critique si le codage : $1 = 00$; $2 = 01$; $3 = 11$; $4 = 10$; est imposé. Donner les expressions des nouveaux $Y_1$ et $Y_2$.}
		\en{Solve the race problems with this imposed coding: $1 = 00$; $2 = 01$; $3 = 11$; $4 = 10$.}

		\reponse{
			\fr{En appliquant le codage imposé, on obtient la table de Huffman suivante~:}
			\en{By applying the coding, we get this Huffman table:}
			\begin{center}
				\begin{tabular}{|l|l|l|l|l|l}
				\hline
				 $Y_1$ $Y_2$ & 00         & 01         & 11         & 10         & \multicolumn{1}{l|}{ab} \\ \hline
				00 & \textbf{00} & \textbf{00} & 11          & -          &    \\ \cline{1-5}
				01 & \textbf{01} & \textbf{01} & 11          & 10          &    \\ \cline{1-5}
				11 & 01          & 00          & \textbf{11} & \textbf{11} &    \\ \cline{1-5}
				10 & 00          & -          & \textbf{10} & \textbf{10} &    \\ \cline{1-5}
				\end{tabular}
			\end{center}

			\fr{Nous avons maintenant plusieurs problèmes de course, par exemple lorsque l'on se trouve dans l'état $00$ avec les entrées $01$ et que ces dernières passent à $11$, la machine nous envoie à l'état $11$, faisant donc changer les deux variables de rétroaction simultanément.
			Nous pouvons répertorier tous les problèmes de course~:}
			\en{We have several race problems.
			For example, when inputs change from $01$ to $11$ in state $00$, the machine sends us to $11$ where both retroaction variables changed.
			Here are all the race problems:}
			\fr{%
			\begin{enumerate}
				\item Transition de l'état $00$ à l'état $11$ lorsque les entrées passent de $01$ à $11$.
				\item Transition de l'état $01$ à l'état $10$ lorsque les entrées passent de $00$ à $10$.
				\item Transition de l'état $11$ à l'état $00$ lorsque les entrées passent de $11$ à $01$.
			\end{enumerate}
			}
			\en{%
			\begin{enumerate}
				\item Transition from state $00$ to $11$ when inputs change from $01$ to $11$.
				\item Transition from state $01$ to $10$ when inputs change from $00$ to $10$.
				\item Transition from state $11$ to $00$ when inputs change from $11$ to $01$.
			\end{enumerate}
			}

			\fr{Puisque le codage est imposé, nous allons utiliser les \textit{don't care} et les états instables pour régler ces problèmes en effectuant des transitions intermédiaires.
			En illustrant avec des tables de Huffman partielles~:}
			\en{Since the coding is imposed, we need to use \textit{don't care} and unstable states to solve the race problems, by addind intermediary transitions.
			Let's illustrate with partial Huffman tables:}
			\begin{enumerate}
				\item 
				% \begin{center}
					\begin{tabular}{|c|c|c|}\hline
					$Y_1Y_2$ & 01 & 11 \\ \hline
					00 & \textbf{00} & 11 \\ \hline
					01 & \textbf{01} & 11 \\ \hline
					11 & 00 & \textbf{11} \\ \hline
					\end{tabular}
					\fr{devient}
					\en{becomes}
					\begin{tabular}{|c|c|c|}\hline
					$Y_1Y_2$ & 01 & 11 \\ \hline
					00 & \textbf{00} & \color{red}{01} \\ \hline
					01 & \textbf{01} & 11 \\ \hline
					11 & 00 & \textbf{11} \\ \hline
					\end{tabular}
				% \end{center}

				\item
				\begin{tabular}{|c|c|c|}\hline
					$Y_1Y_2$ & 00 & 10 \\ \hline
					00 & \textbf{00} & - \\ \hline
					01 & \textbf{01} & 10 \\ \hline
					10 & 00 & \textbf{10} \\ \hline
				\end{tabular}
					\fr{devient}
					\en{becomes}
				\begin{tabular}{|c|c|c|}\hline
					$Y_1Y_2$ & 00 & 10 \\ \hline
					00 & \textbf{00} & \color{red}{10} \\ \hline
					01 & \textbf{01} & \color{red}{00} \\ \hline
					10 & 00 & \textbf{10} \\ \hline
				\end{tabular}

				\item
				\begin{tabular}{|c|c|c|}\hline
					$Y_1Y_2$ & 01 & 11 \\ \hline
					00 & \textbf{00} & 01 \\ \hline
					11 & 00 & \textbf{11} \\ \hline
					10 & - & \textbf{10} \\ \hline
				\end{tabular}
					\fr{devient}
					\en{becomes}
				\begin{tabular}{|c|c|c|}\hline
					$Y_1Y_2$ & 01 & 11 \\ \hline
					00 & \textbf{00} & 01 \\ \hline
					11 & \color{red}{10} & \textbf{11} \\ \hline
					10 & \color{red}{00} & \textbf{10} \\ \hline
				\end{tabular}
			\end{enumerate}

			\fr{On obtient donc une nouvelle table de Huffman sans condition de course~:}
			\en{We end up with a new Huffman table without race problems:}
			\begin{center}
				\begin{tabular}{|l|l|l|l|l|l}
				\hline
				 $Y_1$ $Y_2$ & 00         & 01         & 11         & 10         & \multicolumn{1}{l|}{ab} \\ \hline
				00 & \textbf{00} & \textbf{00} & 01          & 10          &    \\ \cline{1-5}
				01 & \textbf{01} & \textbf{01} & 11          & 00          &    \\ \cline{1-5}
				11 & 01          & 10          & \textbf{11} & \textbf{11} &    \\ \cline{1-5}
				10 & 00          & 00          & \textbf{10} & \textbf{10} &    \\ \cline{1-5}
				\end{tabular}
			\end{center}
		}
	\end{enumerate}

\reponse{}
\end{Q}












\begin{Q}
	\fr{En codant les états 1, 2, 3 et 4 par $y_1y_2$ = 00, 01, 11 et 10 respectivement, calculer les fonctions d'excitation des organes de mémoire pour l'automatisme suivant :}
	\en{By coding states 1, 2, 3 and 4 by $y_1y_2$ = 00, 01, 11 and 10, compute the excitation functions or memory modules for this automaton:}


	\begin{center}
		\begin{tabular}{|l|l|l|l|l|l}
		\hline
		$Y_1$ $Y_2$ & 00         & 01         & 11         & 10         & \multicolumn{1}{l|}{ab} \\ \hline
		1           & \textbf{1} & \textbf{1} & 2          & -          &                         \\ \cline{1-5}
		2           & \textbf{2} & 3          & \textbf{2} & \textbf{2} &                         \\ \cline{1-5}
		3           & 4          & \textbf{3} & 2          & -          &                         \\ \cline{1-5}
		4           & \textbf{4} & 1          & 2          & \textbf{-} &                         \\ \cline{1-5}
		\end{tabular}
	\end{center}
	\fr{Comme organes de mémoire on considérera des flip-flops D puis des flip-flops SRc.}
	\en{Consider bistables D and SRc.}

	\reponse{
		\fr{Commençons par coder la table de Huffman~:}
		\en{First, coding the Huffman table:}
		\begin{center}
			\begin{tabular}{|l|l|l|l|l|l}
			\hline
			$Y_1$ $Y_2$ & 00         & 01         & 11         & 10         & \multicolumn{1}{l|}{ab} \\ \hline
			00           & \textbf{00} & \textbf{00} & 01          & -          &                         \\ \cline{1-5}
			01           & \textbf{01} & 11          & \textbf{01} & \textbf{01} &                         \\ \cline{1-5}
			11           & 10          & \textbf{11} & 01          & -          &                         \\ \cline{1-5}
			10           & \textbf{10} & 00          & 01          & \textbf{-} &                         \\ \cline{1-5}
			\end{tabular}
		\end{center}

		\fr{À titre de rappel, les termes d'excitation sont repris dans la table~\ref{tab:fct-exc} et les tables d'excitation dans les tables~\ref{tab:exc-flip-flop}.}
		\en{The excitation terms are in table~\ref{tab:fct-exc} and the excitation tables in~\ref{tab:exc-flip-flop}.}

		\begin{table}[H]
		\centering
			\begin{tabular}{cccl}
			$Q$ & $Q^+$ & & \\ \hline
			0 & 0 & $\mu_0$ & Maintien à 0 \\
			0 & 1 & $\varepsilon$ & Enclenchement \\
			1 & 0 & $\delta$ & Désenclenchement \\
			1 & 1 & $\mu_1$ & Maintien à 1\\
			\end{tabular}
		\fr{\caption{Termes d'excitation, $Q$ étant l'état courant et $Q^+$ l'état futur.}}
		\en{\caption{Excitation terms, $Q$ is the current state, $Q^+$ the future state.}}
		\label{tab:fct-exc}
		\end{table}
		\begin{table}[ht]
			\center
			\subfloat[\label{tab:exc-D}Flip-flop D]{%
				\parbox{.3\linewidth}{% Embed the content of the subfloat into a parbox to make it wider. Otherwise, the width of the subfloat is set by the width of the table, and so is the caption.
				\center
					\begin{tabular}{c|c}
					& D \\ \hline
					$\mu_0$ & 0 \\
					$\varepsilon$ & 1 \\
					$\delta$ & 0 \\
					$\mu_1$ & 1
					\end{tabular}
				}
			}
			\subfloat[\label{tab:exc-SR}Flip-flop SR]{%
				\parbox{.3\linewidth}{%
					\center
					\begin{tabular}{c|cc}
					& S & R \\ \hline
					$\mu_0$ & 0 & - \\
					$\varepsilon$ & 1 & 0 \\
					$\delta$ & 0 & 1 \\
					$\mu_1$ & - & 0
					\end{tabular}
				}
			}
			\fr{\caption{Tables d'excitation des flip-flops (a) D et (b) SR.}}
			\en{\caption{Excitation tables of (a) D and (b) SR.}}
			\label{tab:exc-flip-flop}
		\end{table}

		\fr{La première étape consiste à dresser la table d'états à partir de la table de Huffman.
		Pour cela, on remplace les termes des états stables par leur terme de maintien correspondant~: $00$ devient $\mu_0\mu_0$, $01$ devient $\mu_0\mu_1$, etc.
		Quant aux termes de transition, ils dépendent des changements d'état. Par exemple, si l'on passe de l'état $00$ à l'état $01$, le premier terme de l'état futur est maintenu à zéro, tandis que le second passe de $0$ à $1$ (il est enclenché), résultant en une transition $\mu_0\varepsilon$.
		On obtient alors la table d'état suivante~:}

		\en{The first step is to build the state table from the Huffman table.
		We simply need to replace the stable states with their proper holding term.
		$00$ becomes $\mu_0\mu_0$, $01$ becomes $\mu_0\mu_1$, etc.
		As for the transition states, it depends on the variable that is changed.
		For example, if we go from state $00$ to $01$, the first term of the future state is held to $0$, whilst the second changes from $0$ to $1$, hence becoming $\mu_0\varepsilon$.

		We finally get this state table:}
		\begin{center}
			\begin{tabular}{|l|l|l|l|l|l} \hline
			$Q^+_1$ $Q^+_2$ & 00         & 01         & 11         & 10         & \multicolumn{1}{l|}{ab} \\ \hline
			00           & $\mu_0\mu_0$ & $\mu_0\mu_0$ & $\mu_0\varepsilon$ & - & \\ \cline{1-5}
			01           & $\mu_0\mu_1$ & $\varepsilon\mu_1$ & $\mu_0\mu_1$ & $\mu_0\mu_1$ & \\ \cline{1-5}
			11           & $\mu_1\delta$ & $\mu_1\mu_1$ & $\delta\mu_1$ & - & \\ \cline{1-5}
			10           & $\mu_1\mu_0$ & $\delta\mu_0$ & $\delta\varepsilon$ & - & \\ \cline{1-5}
			\end{tabular}
		\end{center}

		\fr{Étant donné que chaque état futur est composé de deux termes, deux flip-flops seront nécessaires, un pour chacune des deux «~sous-tables d'états~» suivantes~:}
		\en{Since all future states are composed of two variables, we will need two bistables, one for each ``sub-state-table":}
		\begin{center}
			\begin{tabular}{|l|l|l|l|l|l} \hline
			$Q^+_1$ & 00         & 01         & 11         & 10         & \multicolumn{1}{l|}{ab} \\ \hline
			00           & $\mu_0$ & $\mu_0$ & $\mu_0$ & - & \\ \cline{1-5}
			01           & $\mu_0$ & $\varepsilon$ & $\mu_0$ & $\mu_0$ & \\ \cline{1-5}
			11           & $\mu_1$ & $\mu_1$ & $\delta$ & - & \\ \cline{1-5}
			10           & $\mu_1$ & $\delta$ & $\delta$ & - & \\ \cline{1-5}
			\end{tabular}
			\begin{tabular}{|l|l|l|l|l|l} \hline
			$Q^+_2$ & 00         & 01         & 11         & 10         & \multicolumn{1}{l|}{ab} \\ \hline
			00           & $\mu_0$ & $\mu_0$ & $\varepsilon$ & - & \\ \cline{1-5}
			01           & $\mu_1$ & $\mu_1$ & $\mu_1$ & $\mu_1$ & \\ \cline{1-5}
			11           & $\delta$ & $\mu_1$ & $\mu_1$ & - & \\ \cline{1-5}
			10           & $\mu_0$ & $\mu_0$ & $\varepsilon$ & - & \\ \cline{1-5}
			\end{tabular}
		\end{center}


		\begin{enumerate}
			\item \fr{Considérons d'abord un flip-flop D.
			D'après la table d'excitation de l'organe table~\ref{tab:exc-D}, on peut établir les deux tables d'état des deux flip-flops $D_1$ et $D_2$~:}
			\en{Let's first work with a bistable D.
			Using the excitation table~\ref{tab:exc-D}, we can build two state tables for the two bistables.}
			\begin{center}
				\begin{tabular}{|l|l|l|l|l|l} \hline
				$D_1$ & 00         & 01         & 11         & 10         & \multicolumn{1}{l|}{ab} \\ \hline
				00           & 0 & 0 & 0 & - & \\ \cline{1-5}
				01           & 0 & 1 & 0 & 0 & \\ \cline{1-5}
				11           & 1 & 1 & 0 & - & \\ \cline{1-5}
				10           & 1 & 0 & 0 & - & \\ \cline{1-5}
				\end{tabular}
				\begin{tabular}{|l|l|l|l|l|l} \hline
				$D_2$ & 00         & 01         & 11         & 10         & \multicolumn{1}{l|}{ab} \\ \hline
				00           & 0 & 0 & 1 & - & \\ \cline{1-5}
				01           & 1 & 1 & 1 & 1 & \\ \cline{1-5}
				11           & 0 & 1 & 1 & - & \\ \cline{1-5}
				10           & 0 & 0 & 1 & - & \\ \cline{1-5}
				\end{tabular}
			\end{center}

			\fr{Les k-maps et fonctions d'excitation sont maintenant immédiates~:}
			\en{Hence the K-maps and excitation functions:}
			\begin{center}
			\askmapiv{$D_1 = y_1\overline{a} + y_2\overline{a}b$}{a b $y_1$ $y_2$}{}{00110101-0--0000}{%
			\color{red}\put(0.1,0.1){\dashbox{0.1}(0.8,1.8){}}%
			\color{red}\put(3.1,0.1){\dashbox{0.1}(0.8,1.8){}}%
			\color{green}\put(1.1,1.1){\dashbox{0.2}(0.8,1.8){}}%
			}

			\askmapiv{$D_2 = y_2b + \overline{y_1}y_2 + a$}{a b $y_1$ $y_2$}{}{01000101-1--1111}{%
			\color{red}\put(1.1,1.1){\dashbox{0.1}(1.8,1.7){}}%
			\color{blue}\put(0.1,2.1){\dashbox{0.1}(3.8,0.8){}}%
			\color{green}\put(2.1,0.1){\dashbox{0.2}(1.8,3.8){}}%
			}
			\end{center}


			\item \fr{Passons ensuite au flip-flop SR.
			En utilisant les mêmes sous-tables et la table d'excitation~\ref{tab:exc-SR}, on peut obtenir de nouvelles tables d'état~:}
			\en{We then do the same for the SR by using table~~\ref{tab:exc-SR}.}
			\begin{center}
				\begin{tabular}{|l|l|l|l|l|l} \hline
				$S_1R_1$ & 00         & 01         & 11         & 10         & \multicolumn{1}{l|}{ab} \\ \hline
				00           & 0- & 0- & 0- & -- & \\ \cline{1-5}
				01           & 0- & 10 & 0- & 0- & \\ \cline{1-5}
				11           & -0 & -0 & 01 & -- & \\ \cline{1-5}
				10           & -0 & 01 & 01 & -- & \\ \cline{1-5}
				\end{tabular}
				\begin{tabular}{|l|l|l|l|l|l} \hline
				$S_2R_2$ & 00         & 01         & 11         & 10         & \multicolumn{1}{l|}{ab} \\ \hline
				00           & 0- & 0- & 10 & -- & \\ \cline{1-5}
				01           & -0 & -0 & -0 & -0 & \\ \cline{1-5}
				11           & 01 & -0 & -0 & -- & \\ \cline{1-5}
				10           & 0- & 0- & 10 & -- & \\ \cline{1-5}
				\end{tabular}
			\end{center}

			\fr{De ces tables, on peut déduire des tables de Karnaugh ainsi que des fonctions d'excitation.}
			\en{From which we find K-maps and excitation functions.}
			\begin{center}
			\askmapiv{$S_1 = y_2\overline{a}b$}{a b $y_1$ $y_2$}{}{00--010--0--0000}{%
			\color{red}\put(1.1,1.1){\dashbox{0.1}(0.8,1.8){}}%
			}

			\askmapiv{$R_1 = a + \overline{y_2}b$}{a b $y_1$ $y_2$}{}{--00-010------11}{%
			\color{red}\put(1.1,0.1){\dashbox{0.1}(1.8,0.8){}}%
			\color{red}\put(1.1,3.1){\dashbox{0.1}(1.8,0.8){}}%
			\color{blue}\put(2.1,0.2){\dashbox{0.2}(1.8,3.6){}}%
			}

			\askmapiv{$S_2 = a$}{a b $y_1$ $y_2$}{}{0-000-0-----1-1-}{%
			\color{green}\put(2.1,0.1){\dashbox{0.2}(1.8,3.8){}}%
			}

			\askmapiv{$R_2 = y_1 \overline{b}$}{a b $y_1$ $y_2$}{}{-0-1-0-0-0--0000}{%
			\color{red}\put(0.1,0.1){\dashbox{0.1}(0.8,1.8){}}%
			\color{red}\put(3.1,0.1){\dashbox{0.1}(0.8,1.8){}}%
			}
			\end{center}

		\end{enumerate}
	}

\end{Q}


\end{document}
