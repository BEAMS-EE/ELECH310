\documentclass[11pt,a4paper,dvipsnames]{article}
\usepackage[utf8]{inputenc}
\usepackage[T1]{fontenc}
\usepackage{amsthm} %numéroter les questions
\usepackage[frenchb]{babel}
\usepackage{datetime}
\usepackage{xspace} % typographie IN
\usepackage{hyperref}% hyperliensti
\usepackage[all]{hypcap} %lien pointe en haut des figures
\usepackage[french]{varioref} %voir x p y
\usepackage{fancyhdr}% en têtes
%\input cyracc.def
\usepackage{graphicx} %include pictures
\usepackage{pgfplots}

\usepackage{tikz}
\usetikzlibrary{calc}
\usetikzlibrary{babel}
% \usetikzlibrary{circuits.logic.US,circuits.logic.IEC}
\usepackage{circuitikz}
% \usepackage{gnuplottex}
\usepackage{float}
\usepackage{ifthen}

\usepackage[top=1.3 in, bottom=1.3 in, left=1.3 in, right=1.3 in]{geometry} % Yeah, that's bad to play with margins
\usepackage[]{pdfpages}

\usepackage[]{attachfile}

\usepackage{amsmath}

\usepackage{askmaps}
% \usepackage[usenames,dvipsnames,svgnames,table]{xcolor}
\usepackage[]{xcolor}
\usepackage{colortbl}

\usepackage{multirow}

\newcommand\encircle[1]{%http://tex.stackexchange.com/questions/123924/indexed-letters-inside-circles
  \tikz[baseline=(X.base)] 
    \node (X) [draw, shape=circle, inner sep=0] {\strut #1};}

\newdateformat{mydate}{2016--2017}%hack pour remplacer \THEYEAR

%cyr
\newcommand\textcyr[1]{{\fontencoding{OT2}\fontfamily{wncyr}\selectfont #1}}


\newboolean{corrige}
\ifx\correction\undefined
\setboolean{corrige}{false}% pas de corrigé
\else
\setboolean{corrige}{true}%corrigé
\fi

%\setboolean{corrige}{false}% pas de corrigé

\newboolean{annexes}
\setboolean{annexes}{true}%annexes
%\setboolean{annexes}{false}% pas de annexes

\definecolor{darkblue}{rgb}{0,0,0.5}

\newboolean{mos}
%\setboolean{mos}{true}%annexes
\setboolean{mos}{false}% pas de annexes

\usepackage{aeguill} %guillemets

%% fancy header & foot
\pagestyle{fancy}
\lhead{[ELEC-H-310] Électronique numérique\\ TP 5}
\rhead{\mydate\today\\ page \thepage}
\chead{\ifthenelse{\boolean{corrige}}{Corrigé}{}}
\cfoot{}
%%

\pdfinfo{
/Author (Quentin Delhaye, Ken Hasselmann, ULB -- BEAMS)
/Title (TP 5, ELEC-H-310)
/ModDate (D:\pdfdate)
}

\hypersetup{
pdftitle={TP 5 [ELEC-H-310] Choucroute numérique},
pdfauthor={Quentin Delhaye, Ken Hasselmann, ©2014 ULB -- BEAMS  },
pdfsubject={}
}

\theoremstyle{definition}% questions pas en italique
\newtheorem{Q}{Question}[] % numéroter les questions [section] ou non []

\newcommand{\reponse}[1]{% pour intégrer une réponse : \reponse{texte} : sera inclus si \boolean{corrige}
	\ifthenelse {\boolean{corrige}} {\paragraph{Réponse :} \color{darkblue}   #1\color{black}} {}
 }

\newcommand{\addcontentslinenono}[4]{\addtocontents{#1}{\protect\contentsline{#2}{#3}{#4}{}}}

\date{\vspace{-1.7cm}\mydate\today}
\title{\vspace{-2cm} TP 5\\ Électronique numérique [ELEC-H-310] \ifthenelse{\boolean{corrige}}{~\\Corrigé}{}}

\setlength{\parskip}{0.2cm plus2mm minus1mm} %espacement entre §
\setlength{\parindent}{0pt}

% \renewcommand{\theenumi}{\alph{enumi}} % Change the 'enumerate' format to use letters.
\usepackage{enumitem}
\setlist[enumerate]{label=\alph*)}% If you want only the x-th level to use this format, use '[enumerate,x]'

\newlength{\gvs}% Gate Vertical Space
\gvs=6em
\newlength{\ghs}% Gate Horizontal Space
\ghs=10em

\begin{document}
\pagestyle{empty}
\maketitle
\vspace*{1cm}

\begin{Q}
Un système de traitement d'image fournit des points (pixels) à quatre niveaux de gris~: $ab = 00, 01, 10, 11$ (respectivement noir, foncé, clair et blanc).
On veut construire un automate qui reçoit les valeurs de $ab$ en séquence, ligne après ligne, et met à $1$ une sortie $Z$ lorsqu'il détecte la fin d'une rampe montante ou descendante complète~: $00-01-10-11$ ou $11-10-01-00$.
On fera l'hypothèse simplificatrice que deux pixels successifs n'auront jamais la même valeur.

Dresser la table de Huffman de ce problème.

\reponse{
	\begin{center}
		\begin{tabular}{|l|l|l|l|l|l|} \hline
			& \multicolumn{4}{c}{$ab$} & \\ \hline
			& 00 & 01 & 11 & 10 & $Z$ \\ \hline
			1 & 2 & \textbf{1} & 6 & \textbf{1} & 0 \\ \hline
			2 & \textbf{2} & 3 & 1 & 1 & 0 \\ \hline
			3 & 1 & \textbf{3} & 1 & 4 & 0 \\ \hline
			4 & 1 & 1 & 5 & \textbf{4} & 0 \\ \hline
			5 & 1 & 1 & \textbf{5} & 7 & 1 \\ \hline
			6 & 1 & 1 & \textbf{6} & 7 & 0 \\ \hline
			7 & 1 & 8 & 1 & \textbf{7} & 0 \\ \hline
			8 & 9 & \textbf{8} & 1 & 1 & 0 \\ \hline
			9 & \textbf{9} & 3 & 1 & 1 & 1 \\ \hline
		\end{tabular}
	\end{center}
	%TODO Table des conditions d'équivalence
}
\end{Q}

% \begin{Q}
% Dans la table des états de Huffman suivante~:
% \begin{center}
% 	\begin{tabular}{|l|l|l|l|l|l|} \hline
% 		& \multicolumn{4}{c}{$ab$} & \\ \hline
% 		& 00 & 01 & 11 & 10 & $Z$ \\ \hline
% 		1 & \textbf{1} & \textbf{1} & 3 & - & 0 \\ \hline
% 		2 & \textbf{2} & \textbf{2} & 3 & 4 & 1 \\ \hline
% 		3 & 2 & 1 & \textbf{3} & \textbf{3} & 1 \\ \hline
% 		4 & 1 & - & \textbf{4} & \textbf{4} & 0 \\ \hline
% 	\end{tabular}
% \end{center}
% \begin{enumerate}
% 	\item Trouver un codage n'offrant pas de problème de course.
% 	Déterminer les fonctions de rétroactions $Y_1$ et $Y_2$.

% 	\item Résoudre les problèmes de course critique si le codage~: $1=00$, $2=01$, $3=11$ et $4=10$ est imposé.
% 	Donner les expressions pour les nouveaux $Y_1$ et $Y_2$.
% \end{enumerate}
% \end{Q}


\begin{Q}
	Deux loutres mutantes attaquent Metropolis.
	Superman est le seul à pouvoir sauver la ville.
	Les terribles mustélidés sont cependant particulièrement résistants et Superman décide de quérir l'aide du Dr Hamilton pour les vaincre.
	Ce dernier lui donne alors une table de Huffman sur la sequence de pouvoirs à utiliser pour triompher ($Z_1$).
	Cependant, il y a aussi introduit une séquence qui rendrait les loutres immortelles et leur permettrait de cracher du feu ($Z_2$).
	La notice suivante est jointe~:
	\begin{verse}
		$a$~: vision laser~; $b$~: souffle glacé~; $Z_1$~: loutres vaincues~; $Z_2$~: loutres invulnérables.
	\end{verse}

	Que doit faire Superman pour sauver Metropolis des loutres infernales~?

	\begin{center}
		\begin{tabular}{|c|c|c|c|c|c|c|} \hline
			& \multicolumn{4}{c|}{$ab$} & & \\ \hline
			$Y_1Y_2$& 00 & 01 & 11 & 10 & $Z_1$ & $Z_2$\\ \hline
			1 & \textbf{1} & 2 & - & 5 & 0 & 0 \\ \hline
			2 & 1 & \textbf{2} & - & 3 & 0 & 0 \\ \hline
			3 & 1 & 4 & - & \textbf{3} & 0 & 0 \\ \hline
			4 & \textbf{4} & \textbf{4} & \textbf{4} & \textbf{4} & 1 & 0 \\ \hline
			5 & 1 & 6 & - & \textbf{5} & 0 & 0\\ \hline
			6 & 1 & \textbf{6} & - & 7 & 0 & 0 \\ \hline
			7 & \textbf{7} & \textbf{7} & \textbf{7} & \textbf{7} & 0 & 1 \\ \hline
		\end{tabular}
	\end{center}

	\reponse{
		Si Superman veut éliminer ces perfides créatures, il devra d'abord les geler à l'aide de son souffle, puis les brûler avec son regard de Kryptonien et finalement les geler à nouveau.
	}
\end{Q}

\begin{Q}
	Selon une légende créée spécialement pour ce TP, jaloux du succès du code Konami\footnote{$\uparrow \uparrow \downarrow \downarrow \leftarrow \rightarrow \leftarrow \rightarrow$ \encircle{B} \encircle{A}}, Nintendo aurait implémenté un code similaire dans ses jeux développés sur NES.
	Le «~code Nintendo~» serait cependant plus simple~: pour activer une fonctionnalité cachée du jeu, il suffirait d'appuyer sur le bouton \encircle{A}, puis d'appuyer sur \encircle{B} en maintenant \encircle{A} enfoncé et enfin d'appuyer sur \framebox{\textsc{select}} toujours en maintenant les deux premiers boutons enfoncés.
	Si l'un des boutons est relâché ou enfoncé dans le mauvais ordre, la séquence est annulée.

	Dresser le circuit logique du code permettant d'activer le bonus.

	\reponse{
		Afin de pouvoir créer le circuit logique, il convient d'abord de remplir la table de Huffman décrivant la séquence de boutons, et en dériver les équations logiques.

		En considérant que la touche \framebox{select} est représentée par la variable $c$, on obtient~:
		\begin{center}
			\begin{tabular}{|c|c|c|c|c|c|c|c|c|c|}\hline
			& \multicolumn{8}{c|}{$abc$} & \\ \hline
			$Y_1Y_2$ & 000 & 001 & 011 & 010 & 100 & 101 & 111 & 110 & Z \\ \hline
			00 & \textbf{00} & \textbf{00} & \textbf{00} & \textbf{00} & 01 & \textbf{00} & \textbf{00} & \textbf{00} & 0 \\ \hline
			01 & 00 & 00 & 00 & 00 & \textbf{01} & 00 & 00 & 11 & 0 \\ \hline
			11 & 00 & 00 & 00 & 00 & 00 & 00 & 10 & \textbf{11} & 0 \\ \hline
			01 & 00 & 00 & 00 & 00 & 00 & 00 & \textbf{10} & 00 & 1 \\ \hline
			\end{tabular}
		\end{center}

		On en déduit les k-maps suivantes~:
		\begin{center}
			\askmapv{$Y_1 = abcy_1 + ab\overline{c}y_2$}{a b c $y_1$ $y_2$}{}{00000000000000000000000001010011}{%
			\color{red}\put(6.1,0.1){\dashbox{0.1}(0.8,1.8){}}%
			\color{green}\put(7.1,1.1){\dashbox{0.2}(0.8,1.8){}}%
			}

			\askmapv{$Y_2 = a\overline{bcy_1} + ab\overline{c}y_2$}{a b c $y_1$ $y_2$}{}{00000000000000001100000001010000}{%
			\color{red}\put(4.1,2.1){\dashbox{0.1}(0.8,1.8){}}%
			\color{green}\put(7.1,1.1){\dashbox{0.2}(0.8,1.8){}}%
			}

			\askmapv{$Z = y_1\overline{y_2}$}{a b c $y_1$ $y_2$}{}{00-000-000-000-000-000-000-0001-}{%
			\color{red}\put(0.1,0.1){\dashbox{0.1}(7.8,0.8){}}%
			}
		\end{center}
			\begin{circuitikz}[scale=0.8, every node/.style={scale=0.8}]
			% \begin{circuitikz}
				\draw
				(0.5\ghs,0.5\gvs) node[not port, rotate=90] (mynotb) {}
				(1\ghs,0.5\gvs) node[not port, rotate=90] (mynotc) {}
				(1.5\ghs,0.5\gvs) node[not port, rotate=90] (mynoty1) {}
				(2\ghs,0.5\gvs) node[not port, rotate=90] (mynoty2) {}
				(2.5\ghs,1.5\gvs) node[and port] (myand11) {}%bottom left
				(2.5\ghs,2.3\gvs) node[and port] (myand12) {}
				(2.5\ghs,3.8\gvs) node[and port] (myand13) {}
				(2.5\ghs,4.6\gvs) node[and port] (myand14) {}
				(2.5\ghs,5.4\gvs) node[and port] (myand15) {}%top left

				(3.5\ghs,1.9\gvs) node[and port] (myand21) {}
				(3.5\ghs,2.7\gvs) node[and port] (myand22) {}
				(3.5\ghs,3.8\gvs) node[and port] (myand23) {}
				(3.5\ghs,5.0\gvs) node[and port] (myand24) {}

				(4.2\ghs,2.3\gvs) node[or port] (myory2) {}% Bottom OR, Y2
				(4.2\ghs,4.4\gvs) node[or port] (myory1) {}% Y1

				(2.5\ghs,6.5\gvs) node[and port] (myandz) {}
				(0,7\gvs) node[shape=coordinate] (top) {}
				(0.0,0.0) node[shape=coordinate] (bottoma) {}%Bottom of line A

				(bottoma |- myand12.in 1) node[shape=coordinate] (dot11) {}
				(bottoma |- myand14.in 1) node[shape=coordinate] (dot12) {}
				($(mynotb.in)-(0.2\ghs,0)$) node[shape=coordinate] (dot21) {}
				(dot21 |- myand14.in 2) node[shape=coordinate] (dot22) {}
				(mynotb |- myand12.in 2) node[shape=coordinate] (dot31) {}
				($(mynotc.in)-(0.2\ghs,0)$) node[shape=coordinate] (dot41) {}
				(dot41 |- myand15.in 2) node[shape=coordinate] (dot42) {}
				(mynotc |- myand11.in 2) node[shape=coordinate] (dot51) {}
				(dot51 |- myand13.in 2) node[shape=coordinate] (dot52) {}
				($(mynoty1.in)-(0.2\ghs,0)$) node[shape=coordinate] (dot61) {}
				(dot61 |- myand15.in 1) node[shape=coordinate] (dot62) {}
				(dot61 |- myandz.in 2) node[shape=coordinate] (dot63) {}
				(mynoty1 |- myand11.in 1) node[shape=coordinate] (dot71) {}
				($(mynoty2.in)-(0.2\ghs,0)$) node[shape=coordinate] (dot81) {}
				(dot81 |- myand13.in 1) node[shape=coordinate] (dot82) {}
				(mynoty2 |- myandz.in 1) node[shape=coordinate] (dot91) {}


				% A
				(bottoma) to[short, -*] (dot11) to[short, -*] (dot12) {} -- (bottoma |- top)
				(bottoma) -- +(0, -0.2\gvs)
				(dot11) -- (myand12.in 1)
				(dot12) -- (myand14.in 1)

				% B
				(dot21) -- +(0,-0.4\gvs)
				(dot21) to[short, *-*] (dot22) -- (dot21 |- top)
				(dot22) -- (myand14.in 2)

				% B'
				(mynotb.out) to[short, -*] (dot31) -- (mynotb |- top)
				(mynotb.in) -- (dot21)
				(dot31) -- (myand12.in 2)

				% C
				(dot41) -- +(0,-0.4\gvs)
				(dot41) to[short, *-*] (dot42) -- (dot41 |- top)
				(dot42) -- (myand15.in 2)

				% C'
				(mynotc.out) to[short, -*] (dot51) to[short, -*] (dot52) -- (mynotc |- top)
				(mynotc.in) -- (dot41)
				(dot51) -- (myand11.in 2)
				(dot52) -- (myand13.in 2)

				% Y1
				(dot61) -- +(0,-0.4\gvs)
				(dot61) to[short, *-*] (dot62) to[short, -*] (dot63) -- (dot61 |- top)
				(dot62) -- (myand15.in 1)
				(dot63) -- (myandz.in 2)

				% Y1'
				(mynoty1.out) to[short, -*] (dot71) -- (mynoty1 |- top)
				(mynoty1.in) -- (dot61)
				(dot71) -- (myand11.in 1)

				% Y2
				(dot81) -- +(0,-0.2\gvs)
				(dot81) to[short, *-*] (dot82) -- (dot81 |- top)
				(dot82) -- (myand13.in 1)

				% Y2'
				(mynoty2.out) to[short, -*] (dot91) -- (mynoty2 |- top)
				(mynoty2.in) -- (dot81)
				(dot91) -- (myandz.in 1)
				
				% AND21
				(myand11.out) -| (myand21.in 2)
				(myand12.out) -| (myand21.in 1)

				% AND22
				(myand23.in 2) to[short, *-] (myand22.in 1)
				(myand14.out) to[short, *-] (myand14.out |- myand23.in 1) to[short, *-] (myand14.out |- myand22.in 2) -- (myand22.in 2)

				% AND23
				(myand14.out |- myand23.in 1) -- (myand23.in 1)
				(myand13.out) -| (myand23.in 2)

				% AND24
				(myand14.out) -| (myand24.in 2)
				(myand15.out) -| (myand24.in 1)

				% OR Y2
				(myand21.out) -| (myory2.in 2)
				(myand22.out) -| (myory2.in 1)
				(myory2.out) |- ($(dot81)-(0,0.2\gvs)$)

				% OR Y1
				(myand23.out) -| (myory1.in 2)
				(myand24.out) -| (myory1.in 1)
				(myory1.out) -- +(0.2\gvs,0) |- ($(dot61)-(0,0.4\gvs)$)

				% Z
				(myandz.out) -- +(0.2\ghs,0)
				(myandz.out) node[anchor=south west] {\Large $Z$}
		
				% Labels
				($(bottoma)+(0,0.15\gvs)$) node[anchor=north east] {\Large $a$}
				(dot21) node[anchor=north east] {\Large $b$}
				(dot41) node[anchor=north east] {\Large $c$}
				(dot61) node[anchor=north east] {\Large $y_1$}
				(dot81) node[anchor=north east] {\Large $y_2$}
				(myory1.out) node[anchor=south west] {\Large $Y_1$}
				($(myory2.out)-(0.1\ghs,0)$) node[anchor=south west] {\Large $Y_2$}
				;
			\end{circuitikz}
	}

\end{Q}	


\end{document}
