\documentclass[11pt,a4paper,dvipsnames]{article}
\usepackage[utf8]{inputenc}
\usepackage[T1]{fontenc}
\usepackage{amsthm} %numéroter les questions
\usepackage[frenchb,english]{babel}
\usepackage{datetime}
\usepackage{xspace} % typographie IN
\usepackage{hyperref}% hyperliens
\usepackage[all]{hypcap} %lien pointe en haut des figures
\usepackage[french]{varioref} %voir x p y
\usepackage{fancyhdr}% en têtes
\usepackage{graphicx} %include pictures
\usepackage{pgfplots}

\usepackage{tikz}
\usetikzlibrary{calc,arrows,automata}
\usetikzlibrary{babel}
\usepackage{circuitikz}
% \usepackage{gnuplottex}
\usepackage{float}
\usepackage{ifthen}

\usepackage[top=1.3 in, bottom=1.3 in, left=1.3 in, right=1.3 in]{geometry}
\usepackage[]{pdfpages}
\usepackage[]{attachfile}

\usepackage{amsmath}
\usepackage{amsfonts}
\usepackage{amssymb}
\usepackage{enumitem}
\setlist[enumerate]{label=\alph*)}% If you want only the x-th level to use this format, use '[enumerate,x]'
\usepackage{multirow}
\usepackage{bigdelim}%Braces in tabular

\usepackage{aeguill} %guillemets
\usepackage{askmaps}
\usepackage{colortbl}


%%%%%%%%%%%%%%%%%%%%%%%%%%%%%%%%%%%%%%%%%%%%%%%%%%%%%%%%%%%%%%%%%%%%%%%%%%%%%%%%%%%%%%%%%%%
% READ THIS BEFORE CHANGING ANYTHING
%
%
% - TP number: can be changed using the command
%		\def \tpnumber {TP 1 }
%
% - Version: controlled by a new command:
%		\newcommand{\version}{v1.0.0}
%
% - Booleans: there are three booleans used in this document:
%	- 'corrige', controlled by defining the variable 'correction'
%	- 'fr', controlled by defining the variable 'french'
%	- 'en', controlled by defining the variable 'english'
% You can define those variables in a makefile using such a command:
% pdflatex -shell-escape -jobname="elech310_tp1_fr" "\def\french{} \documentclass[11pt,a4paper]{article}
\usepackage[utf8]{inputenc}
\usepackage[T1]{fontenc}
\usepackage{amsthm} %numéroter les questions
\usepackage[frenchb,english]{babel}
\usepackage{datetime}
\usepackage{xspace} % typographie IN
\usepackage{hyperref}% hyperliens
\usepackage[all]{hypcap} %lien pointe en haut des figures
\usepackage[french]{varioref} %voir x p y
\usepackage{fancyhdr}% en têtes
\usepackage{graphicx} %include pictures
\usepackage{pgfplots}

\usepackage{tikz}
\usetikzlibrary{calc}
\usetikzlibrary{babel}
\usepackage{circuitikz}
% \usepackage{gnuplottex}
\usepackage{float}
\usepackage{ifthen}

\usepackage[top=1.3 in, bottom=1.3 in, left=1.3 in, right=1.3 in]{geometry}
\usepackage[]{pdfpages}
\usepackage[]{attachfile}

\usepackage{amsmath}
\usepackage{enumitem}
\setlist[enumerate]{label=\alph*)}% If you want only the x-th level to use this format, use '[enumerate,x]'
\usepackage{multirow}

\usepackage{aeguill} %guillemets


%%%%%%%%%%%%%%%%%%%%%%%%%%%%%%%%%%%%%%%%%%%%%%%%%%%%%%%%%%%%%%%%%%%%%%%%%%%%%%%%%%%%%%%%%%%
% READ THIS BEFORE CHANGING ANYTHING
%
%
% - TP number: can be changed using the command
%		\def \tpnumber {TP 1 }
%
% - Version: controlled by a new command:
%		\newcommand{\version}{v1.0.0}
%
% - Booleans: there are three booleans used in this document:
%	- 'corrige', controlled by defining the variable 'correction'
%	- 'fr', controlled by defining the variable 'french'
%	- 'en', controlled by defining the variable 'english'
% You can define those variables in a makefile using such a command:
% pdflatex -shell-escape -jobname="elech310_tp1_fr" "\def\french{} \documentclass[11pt,a4paper]{article}
\usepackage[utf8]{inputenc}
\usepackage[T1]{fontenc}
\usepackage{amsthm} %numéroter les questions
\usepackage[frenchb,english]{babel}
\usepackage{datetime}
\usepackage{xspace} % typographie IN
\usepackage{hyperref}% hyperliens
\usepackage[all]{hypcap} %lien pointe en haut des figures
\usepackage[french]{varioref} %voir x p y
\usepackage{fancyhdr}% en têtes
\usepackage{graphicx} %include pictures
\usepackage{pgfplots}

\usepackage{tikz}
\usetikzlibrary{calc}
\usetikzlibrary{babel}
\usepackage{circuitikz}
% \usepackage{gnuplottex}
\usepackage{float}
\usepackage{ifthen}

\usepackage[top=1.3 in, bottom=1.3 in, left=1.3 in, right=1.3 in]{geometry}
\usepackage[]{pdfpages}
\usepackage[]{attachfile}

\usepackage{amsmath}
\usepackage{enumitem}
\setlist[enumerate]{label=\alph*)}% If you want only the x-th level to use this format, use '[enumerate,x]'
\usepackage{multirow}

\usepackage{aeguill} %guillemets


%%%%%%%%%%%%%%%%%%%%%%%%%%%%%%%%%%%%%%%%%%%%%%%%%%%%%%%%%%%%%%%%%%%%%%%%%%%%%%%%%%%%%%%%%%%
% READ THIS BEFORE CHANGING ANYTHING
%
%
% - TP number: can be changed using the command
%		\def \tpnumber {TP 1 }
%
% - Version: controlled by a new command:
%		\newcommand{\version}{v1.0.0}
%
% - Booleans: there are three booleans used in this document:
%	- 'corrige', controlled by defining the variable 'correction'
%	- 'fr', controlled by defining the variable 'french'
%	- 'en', controlled by defining the variable 'english'
% You can define those variables in a makefile using such a command:
% pdflatex -shell-escape -jobname="elech310_tp1_fr" "\def\french{} \documentclass[11pt,a4paper]{article}
\usepackage[utf8]{inputenc}
\usepackage[T1]{fontenc}
\usepackage{amsthm} %numéroter les questions
\usepackage[frenchb,english]{babel}
\usepackage{datetime}
\usepackage{xspace} % typographie IN
\usepackage{hyperref}% hyperliens
\usepackage[all]{hypcap} %lien pointe en haut des figures
\usepackage[french]{varioref} %voir x p y
\usepackage{fancyhdr}% en têtes
\usepackage{graphicx} %include pictures
\usepackage{pgfplots}

\usepackage{tikz}
\usetikzlibrary{calc}
\usetikzlibrary{babel}
\usepackage{circuitikz}
% \usepackage{gnuplottex}
\usepackage{float}
\usepackage{ifthen}

\usepackage[top=1.3 in, bottom=1.3 in, left=1.3 in, right=1.3 in]{geometry}
\usepackage[]{pdfpages}
\usepackage[]{attachfile}

\usepackage{amsmath}
\usepackage{enumitem}
\setlist[enumerate]{label=\alph*)}% If you want only the x-th level to use this format, use '[enumerate,x]'
\usepackage{multirow}

\usepackage{aeguill} %guillemets


%%%%%%%%%%%%%%%%%%%%%%%%%%%%%%%%%%%%%%%%%%%%%%%%%%%%%%%%%%%%%%%%%%%%%%%%%%%%%%%%%%%%%%%%%%%
% READ THIS BEFORE CHANGING ANYTHING
%
%
% - TP number: can be changed using the command
%		\def \tpnumber {TP 1 }
%
% - Version: controlled by a new command:
%		\newcommand{\version}{v1.0.0}
%
% - Booleans: there are three booleans used in this document:
%	- 'corrige', controlled by defining the variable 'correction'
%	- 'fr', controlled by defining the variable 'french'
%	- 'en', controlled by defining the variable 'english'
% You can define those variables in a makefile using such a command:
% pdflatex -shell-escape -jobname="elech310_tp1_fr" "\def\french{} \input{elech310_tp1.tex}"
%
%%%%%%%%%%%%%%%%%%%%%%%%%%%%%%%%%%%%%%%%%%%%%%%%%%%%%%%%%%%%%%%%%%%%%%%%%%%%%%%%%%%%%%%%%%%

%Numero du TP :
\def \tpnumber {TP 1 }

\newcommand{\version}{v1.0.0}


% ########     #####      #####    ##         #########     ###     ##      ##  
% ##     ##  ##     ##  ##     ##  ##         ##           ## ##    ###     ##  
% ##     ##  ##     ##  ##     ##  ##         ##          ##   ##   ## ##   ##  
% ########   ##     ##  ##     ##  ##         ######     ##     ##  ##  ##  ##  
% ##     ##  ##     ##  ##     ##  ##         ##         #########  ##   ## ##  
% ##     ##  ##     ##  ##     ##  ##         ##         ##     ##  ##     ###  
% ########     #####      #####    #########  #########  ##     ##  ##      ##  
\newboolean{corrige}
\ifx\correction\undefined
\setboolean{corrige}{false}% pas de corrigé
\else
\setboolean{corrige}{true}%corrigé
\fi

\newboolean{fr}
\ifx\french\undefined
\setboolean{fr}{false}% pas de corrigé
\else
\setboolean{fr}{true}%corrigé
\fi

\newboolean{en}
\ifx\english\undefined
\setboolean{en}{false}% pas de corrigé
\else
\setboolean{en}{true}%corrigé
\fi

%\setboolean{corrige}{false}% pas de corrigé

\definecolor{darkblue}{rgb}{0,0,0.5}

\pdfinfo{
/Author (ULB -- BEAMS)
/Title (\tpnumber, ELEC-H-310)
/ModDate (D:\pdfdate)
}

\hypersetup{
pdftitle={\tpnumber [ELEC-H-310] Choucroute numérique},
pdfauthor={ULB -- BEAMS},
pdfsubject={}
}

\theoremstyle{definition}% questions pas en italique
\newtheorem{Q}{Question}[] % numéroter les questions [section] ou non []


%  ######     #####    ##       ##  ##       ##     ###     ##      ##  ######      #######   
% ##     ##  ##     ##  ###     ###  ###     ###    ## ##    ###     ##  ##    ##   ##     ##  
% ##         ##     ##  ## ## ## ##  ## ## ## ##   ##   ##   ## ##   ##  ##     ##  ##         
% ##         ##     ##  ##  ###  ##  ##  ###  ##  ##     ##  ##  ##  ##  ##     ##   #######   
% ##         ##     ##  ##       ##  ##       ##  #########  ##   ## ##  ##     ##         ##  
% ##     ##  ##     ##  ##       ##  ##       ##  ##     ##  ##     ###  ##    ##   ##     ##  
%  #######     #####    ##       ##  ##       ##  ##     ##  ##      ##  ######      #######   
\newcommand{\reponse}[1]{% pour intégrer une réponse : \reponse{texte} : sera inclus si \boolean{corrige}
	\ifthenelse {\boolean{corrige}} {\fr{\paragraph{Réponse :}}\en{\paragraph{Answer:}} \color{darkblue}   #1\color{black}} {}
 }

 \newcommand{\fr}[1]{
 	\ifthenelse {\boolean{fr}} {#1} {}
 }

 \newcommand{\en}[1]{
 	\ifthenelse {\boolean{en}} {#1} {}
 }

\newcommand{\addcontentslinenono}[4]{\addtocontents{#1}{\protect\contentsline{#2}{#3}{#4}{}}}


%  #######   ##########  ##    ##   ##         #########  
% ##     ##      ##       ##  ##    ##         ##         
% ##             ##        ####     ##         ##         
%  #######       ##         ##      ##         ######     
%        ##      ##         ##      ##         ##         
% ##     ##      ##         ##      ##         ##         
%  #######       ##         ##      #########  #########  
%% fancy header & foot
\pagestyle{fancy}
\lhead{[ELEC-H-310] \fr{Électronique numérique}\en{Digital Electronics}\\ \tpnumber}
\rhead{\version\\ page \thepage}
\chead{\ifthenelse{\boolean{corrige}}{\fr{Corrigé}\en{Correction}}{}}
\cfoot{}
%%

\setlength{\parskip}{0.2cm plus2mm minus1mm} %espacement entre §
\setlength{\parindent}{0pt}

% ##########  ########   ##########  ##         #########  
%     ##         ##          ##      ##         ##         
%     ##         ##          ##      ##         ##         
%     ##         ##          ##      ##         ######     
%     ##         ##          ##      ##         ##         
%     ##         ##          ##      ##         ##         
%     ##      ########       ##      #########  ######### 
\date{\vspace{-1.7cm}\version}
\title{\vspace{-2cm} \tpnumber\\ \fr{Électronique numérique [ELEC-H-310] \ifthenelse{\boolean{corrige}}{~\\Corrigé}{}}%
\en{Digital Electronics [ELEC-H-310] \ifthenelse{\boolean{corrige}}{~\\Correction}{}}%
}




\begin{document}
\fr{\selectlanguage{french}}
\en{\selectlanguage{english}}

\maketitle
\vspace*{-1cm}

\begin{Q}
	\fr{Convertir dans les autres bases utiles les nombres suivants :}
	\en{Convert the following number in the relevant bases:}
	\begin{enumerate}
	\item $(82)_{10}$
	\item $(122)_{10}$
	\item $(1001110001)_2$
	\item $(F6D)_{16}$
	\item $(B65F)_{16}$
	\item $(0.625)_{10}$
	\end{enumerate}

	\reponse{
		\begin{center}
			\begin{tabular}{|l|l|l|l|}\hline
				Base 2 & Base 8 & Base 10 & Base 16 \\ \hline
				1010010 & 122 & \textbf{82} & 52 \\ \hline
				1111010 & 172 & \textbf{122} & 7A \\ \hline
				\textbf{1001110001} & 1161 & 625 & 271 \\ \hline
				111101101101 & 7555 & 3949 & \textbf{F6D} \\ \hline
				1011011001011111 & 133137 & 46687 & \textbf{B65F} \\ \hline
				0.101 & 0.5 & \textbf{0.625} & 0.A \\ \hline
			\end{tabular}
		\end{center}
	}
\end{Q}

\begin{Q}
\fr{Effectuer l’addition suivante dans toutes les bases utiles. Vérifier les résultats en
les convertissant en base 10 :}
\en{Compute this addition in all relevant bases. Check your results by converting them in base 10.}
$$(3633)_{10} + (254)_{10}$$

\reponse{~\\
	Base 2:
	\begin{tabular}{ccccccccccccc}
		& 1 & 1 & 1 & 0 & 0 & 0 & 1 & 1 & 0 & 0 & 0 & 1 \\
		+ & &   &   &   & 1 & 1 & 1 & 1 & 1 & 1 & 1 & 0\\ \hline
		& 1 & 1 & 1 & 1 & 0 & 0 & 1 & 0 & 1 & 1 & 1 & 1 \\
	\end{tabular}

	Base 8:
	\begin{tabular}{ccccc}
		  & 7 & 0 & 6 & 1 \\
		+ &   & 3 & 7 & 6 \\ \hline
		  & 7 & 4 & 5 & 7 \\
		\end{tabular}

	Base 16:
	\begin{tabular}{cccc}
		& E & 3 & 1 \\
		+ & & F & E \\ \hline
		& F & 2 & F\\
	\end{tabular}
}
\end{Q}


\begin{Q}
\fr{Représentation des nombres négatifs}
\en{Negative numbers representation}
\begin{enumerate}
	\item \fr{Représenter $(-14)_{10}$ sur 8 bits en base 2 dans les trois modes de représentation.}
	\en{Represent $(-14)_{10}$ on 8 bits in base 2 in all three representations.}
	\reponse{
		\begin{tabular}{|c|c|c|}\hline
			SVA & C1 & C2 \\ \hline
			10001110 & 11110001 & 11110010 \\ \hline
		\end{tabular}
	}
	\item \fr{Si on utilise 4 bits, quels sont, dans les 3 modes de représentation, les plus
	petites et les plus grandes valeurs représentables ? Comment se représente la
	valeur 0 ?}
	\en{If we use 4 bits, what are the highest and lowest values possible, in all three representations?
	How is represented the value 0?}
	\reponse{
		\begin{center}
			\begin{tabular}{|c|c|c|c|}\hline
				& min & 0 & max \\ \hline
				\multirow{2}{*}{SVA} & \multirow{2}{*}{1111} & 0000 & \multirow{2}{*}{0111} \\
				& & 1000 & \\ \hline
				\multirow{2}{*}{C1} & \multirow{2}{*}{10000} & 0000 & \multirow{2}{*}{0111} \\
				& & 1111 & \\ \hline
				C2 & 1000 & 0000 & 0111 \\ \hline
			\end{tabular}
		\end{center}

		\fr{À titre de bonus, ci-suit un tableau comparatif des différents modes de représentation (sur 4 bits):}
		\en{As a bonus, here is the full table comparing the three representations.}
		\begin{center}
			\begin{tabular}{|c|c|c|c|} \hline
				Base 10 & Signé & C1 & C2 \\ \hline
				7 & 0111 & 0111 & 0111 \\ \hline
				6 & 0110 & 0110 & 0110 \\ \hline
				5 & 0101 & 0101 & 0101 \\ \hline
				4 & 0100 & 0100 & 0100 \\ \hline
				3 & 0011 & 0011 & 0011 \\ \hline
				2 & 0010 & 0010 & 0010 \\ \hline
				1 & 0001 & 0001 & 0001 \\ \hline
				\multirow{2}{*}{0} & 0000 & 0000 & \multirow{2}{*}{0000} \\
				& 1000 & 1111 & \\ \hline
				-1 & 1001 & 1110 & 1111 \\ \hline
				-2 & 1010 & 1101 & 1110 \\ \hline
				-3 & 1011 & 1100 & 1101 \\ \hline
				-4 & 1100 & 1011 & 1100 \\ \hline
				-5 & 1101 & 1010 & 1011 \\ \hline
				-6 & 1110 & 1001 & 1010 \\ \hline
				-7 & 1111 & 1000 & 1001 \\ \hline
				-8 & N/A & N/A & 1000 \\ \hline
			\end{tabular}
		\end{center}
	}
\end{enumerate}
\end{Q}



\begin{Q}
\fr{En utilisant les axiomes prouver les théorèmes :}
\en{Proove the theorems using the axioms.}

\begin{minipage}[t]{0.5\linewidth}
	\centering Axiomes :
	$$A1a : x \in B, y \in B \rightarrow x+y \in B$$
	$$A1b : x \in B, y \in B \rightarrow x \cdot y \in B$$
	$$A2a : x+0 = x$$
	$$A2b : x \cdot 1 = x$$
	$$A3a : x \cdot (y+z) = x \cdot y+x \cdot z$$
	$$A3b : x+y \cdot z = (x+y)(x+z)$$
	$$A4a : x+y = y+x$$
	$$A4b : x \cdot y = y \cdot x$$
	$$A5a : x+\overline{x} = 1$$
	$$A5b : x \cdot \overline{x} = 0$$
	$$A6 : \exists x, y \in B : x \neq y$$
\end{minipage}
\begin{minipage}[t]{0.5\linewidth}
	\centering Théorèmes :
	$$T1a : x+x=x$$
	$$T1b : x \cdot x=x$$
	$$T2a : x+1=1$$
	$$T2b : x \cdot 0=0$$
	$$T3a : x+y \cdot x=x$$
	$$T3b : x \cdot (x+y)=x$$
	$$T4 : \overline{(\overline{x})}=x$$
\end{minipage}

\reponse{
\begin{itemize}
	\item $T1a : x+x=x$
	\begin{align*}
		x + x & = (x + x) \cdot 1&\mbox{, A2b}\\
		& = (x + x) \cdot (x + \overline{x})&\mbox{, A5a}\\
		& = x + x \cdot \overline{x}&\mbox{, A3b}\\
		& = x &\mbox{, A5b}
	\end{align*}

	\item $T1b : x \cdot x=x$
	\begin{align*}
		x \cdot x & = x \cdot x  + 0&\mbox{, A2a}\\
		& = x\cdot x + x \cdot \overline{x} &\mbox{, A5b}\\
		& = x \cdot (x + \overline{x}) &\mbox{, A3a} \\
		& = x \cdot 1&\mbox{, A5a}\\
		& = x&\mbox{, A2b}
	\end{align*}

	\item $T2a : x+1=1$
	\begin{align*}
		x + 1 & = (x+1) \cdot 1&\mbox{, A2b}\\
		& = (x+1) \cdot (x+\overline{x})&\mbox{, A5a}\\
		& = x + 1 \cdot \overline{x} &\mbox{, A3b}\\
		& = x + \overline{x} &\mbox{, A2b}\\
		& = 1 &\mbox{, A5a}
	\end{align*}

	\item $T2b : x \cdot 0=0$
	\begin{align*}
		x \cdot 0 & = x \cdot 0 + 0 &\mbox{, A2a} \\
		& = x \cdot 0 + x \cdot \overline{x}&\mbox{, A5b} \\
		& = x \cdot (0 + \overline{x}) &\mbox{, A3a} \\
		& = x \cdot (\overline{x}) &\mbox{, A2a} \\
		& = 0 &\mbox{, A5b}
	\end{align*}

	\item $T3a : x+y \cdot x=x$
	\begin{align*}
		x + x \cdot y & = x \cdot 1 + x \cdot y &\mbox{, A2b}\\
		& = x \cdot (1 + y)&\mbox{, A3a}\\
		& = x \cdot 1&\mbox{, T2a} \\
		& = x &\mbox{, A2b}
	\end{align*}

	\item $T3b : x \cdot (x+y)=x$
	\begin{align*}
		x \cdot (x+y) & = (x+0) \cdot (x+y) &\mbox{, A2a}\\
		& = x + 0 \cdot y &\mbox{, A3b}\\
		& = x + 0 &\mbox{, T2b} \\
		& = x &\mbox{, A2a}
	\end{align*}

	\item $T4 : \overline{(\overline{x})}=x$
	\begin{align*}
		\overline{(\overline{x})} & = \overline{(\overline{x})} \cdot 1 &\mbox{, A2b}\\
		& = \overline{(\overline{x})} \cdot (x + \overline{x}) &\mbox{, A5a}\\
		& = \overline{(\overline{x})} \cdot x + \overline{(\overline{x})} \cdot \overline{x} &\mbox{, A3a}\\
		& = \overline{(\overline{x})} \cdot x + 0 &\mbox{, A5b}\\
		& = \overline{(\overline{x})} \cdot x + x \cdot \overline{x} &\mbox{, A5b}\\
		& = x\cdot (\overline{(\overline{x})} + \overline{x})&\mbox{, A5a}\\
		& = x&\mbox{, A5a}\\
	\end{align*}
\end{itemize}
}

\end{Q}
%
% \begin{Q}
% 	L'approximation I(V) idéalisée avec seulement la tension de seuil est-elle une bonne approximation ?
% 	\label{Q:1}
% 	\reponse{Oui}%R
% \end{Q}


% \begin{figure}[H]
% 	\begin{center}
% 		\begin{circuitikz}\draw
% 			(0,0) node[anchor=east] {A} to [short,i>^=$I$] (1.5,0)
% 			(0,0) to [sDo, v=$V$] (2.5,0) node [anchor=west]{K}
% 		;\end{circuitikz}
% 	\end{center}
% \caption{Conventions électriques}
% \label{fig:zener_conv}
% \end{figure}

\end{document}
"
%
%%%%%%%%%%%%%%%%%%%%%%%%%%%%%%%%%%%%%%%%%%%%%%%%%%%%%%%%%%%%%%%%%%%%%%%%%%%%%%%%%%%%%%%%%%%

%Numero du TP :
\def \tpnumber {TP 1 }

\newcommand{\version}{v1.0.0}


% ########     #####      #####    ##         #########     ###     ##      ##  
% ##     ##  ##     ##  ##     ##  ##         ##           ## ##    ###     ##  
% ##     ##  ##     ##  ##     ##  ##         ##          ##   ##   ## ##   ##  
% ########   ##     ##  ##     ##  ##         ######     ##     ##  ##  ##  ##  
% ##     ##  ##     ##  ##     ##  ##         ##         #########  ##   ## ##  
% ##     ##  ##     ##  ##     ##  ##         ##         ##     ##  ##     ###  
% ########     #####      #####    #########  #########  ##     ##  ##      ##  
\newboolean{corrige}
\ifx\correction\undefined
\setboolean{corrige}{false}% pas de corrigé
\else
\setboolean{corrige}{true}%corrigé
\fi

\newboolean{fr}
\ifx\french\undefined
\setboolean{fr}{false}% pas de corrigé
\else
\setboolean{fr}{true}%corrigé
\fi

\newboolean{en}
\ifx\english\undefined
\setboolean{en}{false}% pas de corrigé
\else
\setboolean{en}{true}%corrigé
\fi

%\setboolean{corrige}{false}% pas de corrigé

\definecolor{darkblue}{rgb}{0,0,0.5}

\pdfinfo{
/Author (ULB -- BEAMS)
/Title (\tpnumber, ELEC-H-310)
/ModDate (D:\pdfdate)
}

\hypersetup{
pdftitle={\tpnumber [ELEC-H-310] Choucroute numérique},
pdfauthor={ULB -- BEAMS},
pdfsubject={}
}

\theoremstyle{definition}% questions pas en italique
\newtheorem{Q}{Question}[] % numéroter les questions [section] ou non []


%  ######     #####    ##       ##  ##       ##     ###     ##      ##  ######      #######   
% ##     ##  ##     ##  ###     ###  ###     ###    ## ##    ###     ##  ##    ##   ##     ##  
% ##         ##     ##  ## ## ## ##  ## ## ## ##   ##   ##   ## ##   ##  ##     ##  ##         
% ##         ##     ##  ##  ###  ##  ##  ###  ##  ##     ##  ##  ##  ##  ##     ##   #######   
% ##         ##     ##  ##       ##  ##       ##  #########  ##   ## ##  ##     ##         ##  
% ##     ##  ##     ##  ##       ##  ##       ##  ##     ##  ##     ###  ##    ##   ##     ##  
%  #######     #####    ##       ##  ##       ##  ##     ##  ##      ##  ######      #######   
\newcommand{\reponse}[1]{% pour intégrer une réponse : \reponse{texte} : sera inclus si \boolean{corrige}
	\ifthenelse {\boolean{corrige}} {\fr{\paragraph{Réponse :}}\en{\paragraph{Answer:}} \color{darkblue}   #1\color{black}} {}
 }

 \newcommand{\fr}[1]{
 	\ifthenelse {\boolean{fr}} {#1} {}
 }

 \newcommand{\en}[1]{
 	\ifthenelse {\boolean{en}} {#1} {}
 }

\newcommand{\addcontentslinenono}[4]{\addtocontents{#1}{\protect\contentsline{#2}{#3}{#4}{}}}


%  #######   ##########  ##    ##   ##         #########  
% ##     ##      ##       ##  ##    ##         ##         
% ##             ##        ####     ##         ##         
%  #######       ##         ##      ##         ######     
%        ##      ##         ##      ##         ##         
% ##     ##      ##         ##      ##         ##         
%  #######       ##         ##      #########  #########  
%% fancy header & foot
\pagestyle{fancy}
\lhead{[ELEC-H-310] \fr{Électronique numérique}\en{Digital Electronics}\\ \tpnumber}
\rhead{\version\\ page \thepage}
\chead{\ifthenelse{\boolean{corrige}}{\fr{Corrigé}\en{Correction}}{}}
\cfoot{}
%%

\setlength{\parskip}{0.2cm plus2mm minus1mm} %espacement entre §
\setlength{\parindent}{0pt}

% ##########  ########   ##########  ##         #########  
%     ##         ##          ##      ##         ##         
%     ##         ##          ##      ##         ##         
%     ##         ##          ##      ##         ######     
%     ##         ##          ##      ##         ##         
%     ##         ##          ##      ##         ##         
%     ##      ########       ##      #########  ######### 
\date{\vspace{-1.7cm}\version}
\title{\vspace{-2cm} \tpnumber\\ \fr{Électronique numérique [ELEC-H-310] \ifthenelse{\boolean{corrige}}{~\\Corrigé}{}}%
\en{Digital Electronics [ELEC-H-310] \ifthenelse{\boolean{corrige}}{~\\Correction}{}}%
}




\begin{document}
\fr{\selectlanguage{french}}
\en{\selectlanguage{english}}

\maketitle
\vspace*{-1cm}

\begin{Q}
	\fr{Convertir dans les autres bases utiles les nombres suivants :}
	\en{Convert the following number in the relevant bases:}
	\begin{enumerate}
	\item $(82)_{10}$
	\item $(122)_{10}$
	\item $(1001110001)_2$
	\item $(F6D)_{16}$
	\item $(B65F)_{16}$
	\item $(0.625)_{10}$
	\end{enumerate}

	\reponse{
		\begin{center}
			\begin{tabular}{|l|l|l|l|}\hline
				Base 2 & Base 8 & Base 10 & Base 16 \\ \hline
				1010010 & 122 & \textbf{82} & 52 \\ \hline
				1111010 & 172 & \textbf{122} & 7A \\ \hline
				\textbf{1001110001} & 1161 & 625 & 271 \\ \hline
				111101101101 & 7555 & 3949 & \textbf{F6D} \\ \hline
				1011011001011111 & 133137 & 46687 & \textbf{B65F} \\ \hline
				0.101 & 0.5 & \textbf{0.625} & 0.A \\ \hline
			\end{tabular}
		\end{center}
	}
\end{Q}

\begin{Q}
\fr{Effectuer l’addition suivante dans toutes les bases utiles. Vérifier les résultats en
les convertissant en base 10 :}
\en{Compute this addition in all relevant bases. Check your results by converting them in base 10.}
$$(3633)_{10} + (254)_{10}$$

\reponse{~\\
	Base 2:
	\begin{tabular}{ccccccccccccc}
		& 1 & 1 & 1 & 0 & 0 & 0 & 1 & 1 & 0 & 0 & 0 & 1 \\
		+ & &   &   &   & 1 & 1 & 1 & 1 & 1 & 1 & 1 & 0\\ \hline
		& 1 & 1 & 1 & 1 & 0 & 0 & 1 & 0 & 1 & 1 & 1 & 1 \\
	\end{tabular}

	Base 8:
	\begin{tabular}{ccccc}
		  & 7 & 0 & 6 & 1 \\
		+ &   & 3 & 7 & 6 \\ \hline
		  & 7 & 4 & 5 & 7 \\
		\end{tabular}

	Base 16:
	\begin{tabular}{cccc}
		& E & 3 & 1 \\
		+ & & F & E \\ \hline
		& F & 2 & F\\
	\end{tabular}
}
\end{Q}


\begin{Q}
\fr{Représentation des nombres négatifs}
\en{Negative numbers representation}
\begin{enumerate}
	\item \fr{Représenter $(-14)_{10}$ sur 8 bits en base 2 dans les trois modes de représentation.}
	\en{Represent $(-14)_{10}$ on 8 bits in base 2 in all three representations.}
	\reponse{
		\begin{tabular}{|c|c|c|}\hline
			SVA & C1 & C2 \\ \hline
			10001110 & 11110001 & 11110010 \\ \hline
		\end{tabular}
	}
	\item \fr{Si on utilise 4 bits, quels sont, dans les 3 modes de représentation, les plus
	petites et les plus grandes valeurs représentables ? Comment se représente la
	valeur 0 ?}
	\en{If we use 4 bits, what are the highest and lowest values possible, in all three representations?
	How is represented the value 0?}
	\reponse{
		\begin{center}
			\begin{tabular}{|c|c|c|c|}\hline
				& min & 0 & max \\ \hline
				\multirow{2}{*}{SVA} & \multirow{2}{*}{1111} & 0000 & \multirow{2}{*}{0111} \\
				& & 1000 & \\ \hline
				\multirow{2}{*}{C1} & \multirow{2}{*}{10000} & 0000 & \multirow{2}{*}{0111} \\
				& & 1111 & \\ \hline
				C2 & 1000 & 0000 & 0111 \\ \hline
			\end{tabular}
		\end{center}

		\fr{À titre de bonus, ci-suit un tableau comparatif des différents modes de représentation (sur 4 bits):}
		\en{As a bonus, here is the full table comparing the three representations.}
		\begin{center}
			\begin{tabular}{|c|c|c|c|} \hline
				Base 10 & Signé & C1 & C2 \\ \hline
				7 & 0111 & 0111 & 0111 \\ \hline
				6 & 0110 & 0110 & 0110 \\ \hline
				5 & 0101 & 0101 & 0101 \\ \hline
				4 & 0100 & 0100 & 0100 \\ \hline
				3 & 0011 & 0011 & 0011 \\ \hline
				2 & 0010 & 0010 & 0010 \\ \hline
				1 & 0001 & 0001 & 0001 \\ \hline
				\multirow{2}{*}{0} & 0000 & 0000 & \multirow{2}{*}{0000} \\
				& 1000 & 1111 & \\ \hline
				-1 & 1001 & 1110 & 1111 \\ \hline
				-2 & 1010 & 1101 & 1110 \\ \hline
				-3 & 1011 & 1100 & 1101 \\ \hline
				-4 & 1100 & 1011 & 1100 \\ \hline
				-5 & 1101 & 1010 & 1011 \\ \hline
				-6 & 1110 & 1001 & 1010 \\ \hline
				-7 & 1111 & 1000 & 1001 \\ \hline
				-8 & N/A & N/A & 1000 \\ \hline
			\end{tabular}
		\end{center}
	}
\end{enumerate}
\end{Q}



\begin{Q}
\fr{En utilisant les axiomes prouver les théorèmes :}
\en{Proove the theorems using the axioms.}

\begin{minipage}[t]{0.5\linewidth}
	\centering Axiomes :
	$$A1a : x \in B, y \in B \rightarrow x+y \in B$$
	$$A1b : x \in B, y \in B \rightarrow x \cdot y \in B$$
	$$A2a : x+0 = x$$
	$$A2b : x \cdot 1 = x$$
	$$A3a : x \cdot (y+z) = x \cdot y+x \cdot z$$
	$$A3b : x+y \cdot z = (x+y)(x+z)$$
	$$A4a : x+y = y+x$$
	$$A4b : x \cdot y = y \cdot x$$
	$$A5a : x+\overline{x} = 1$$
	$$A5b : x \cdot \overline{x} = 0$$
	$$A6 : \exists x, y \in B : x \neq y$$
\end{minipage}
\begin{minipage}[t]{0.5\linewidth}
	\centering Théorèmes :
	$$T1a : x+x=x$$
	$$T1b : x \cdot x=x$$
	$$T2a : x+1=1$$
	$$T2b : x \cdot 0=0$$
	$$T3a : x+y \cdot x=x$$
	$$T3b : x \cdot (x+y)=x$$
	$$T4 : \overline{(\overline{x})}=x$$
\end{minipage}

\reponse{
\begin{itemize}
	\item $T1a : x+x=x$
	\begin{align*}
		x + x & = (x + x) \cdot 1&\mbox{, A2b}\\
		& = (x + x) \cdot (x + \overline{x})&\mbox{, A5a}\\
		& = x + x \cdot \overline{x}&\mbox{, A3b}\\
		& = x &\mbox{, A5b}
	\end{align*}

	\item $T1b : x \cdot x=x$
	\begin{align*}
		x \cdot x & = x \cdot x  + 0&\mbox{, A2a}\\
		& = x\cdot x + x \cdot \overline{x} &\mbox{, A5b}\\
		& = x \cdot (x + \overline{x}) &\mbox{, A3a} \\
		& = x \cdot 1&\mbox{, A5a}\\
		& = x&\mbox{, A2b}
	\end{align*}

	\item $T2a : x+1=1$
	\begin{align*}
		x + 1 & = (x+1) \cdot 1&\mbox{, A2b}\\
		& = (x+1) \cdot (x+\overline{x})&\mbox{, A5a}\\
		& = x + 1 \cdot \overline{x} &\mbox{, A3b}\\
		& = x + \overline{x} &\mbox{, A2b}\\
		& = 1 &\mbox{, A5a}
	\end{align*}

	\item $T2b : x \cdot 0=0$
	\begin{align*}
		x \cdot 0 & = x \cdot 0 + 0 &\mbox{, A2a} \\
		& = x \cdot 0 + x \cdot \overline{x}&\mbox{, A5b} \\
		& = x \cdot (0 + \overline{x}) &\mbox{, A3a} \\
		& = x \cdot (\overline{x}) &\mbox{, A2a} \\
		& = 0 &\mbox{, A5b}
	\end{align*}

	\item $T3a : x+y \cdot x=x$
	\begin{align*}
		x + x \cdot y & = x \cdot 1 + x \cdot y &\mbox{, A2b}\\
		& = x \cdot (1 + y)&\mbox{, A3a}\\
		& = x \cdot 1&\mbox{, T2a} \\
		& = x &\mbox{, A2b}
	\end{align*}

	\item $T3b : x \cdot (x+y)=x$
	\begin{align*}
		x \cdot (x+y) & = (x+0) \cdot (x+y) &\mbox{, A2a}\\
		& = x + 0 \cdot y &\mbox{, A3b}\\
		& = x + 0 &\mbox{, T2b} \\
		& = x &\mbox{, A2a}
	\end{align*}

	\item $T4 : \overline{(\overline{x})}=x$
	\begin{align*}
		\overline{(\overline{x})} & = \overline{(\overline{x})} \cdot 1 &\mbox{, A2b}\\
		& = \overline{(\overline{x})} \cdot (x + \overline{x}) &\mbox{, A5a}\\
		& = \overline{(\overline{x})} \cdot x + \overline{(\overline{x})} \cdot \overline{x} &\mbox{, A3a}\\
		& = \overline{(\overline{x})} \cdot x + 0 &\mbox{, A5b}\\
		& = \overline{(\overline{x})} \cdot x + x \cdot \overline{x} &\mbox{, A5b}\\
		& = x\cdot (\overline{(\overline{x})} + \overline{x})&\mbox{, A5a}\\
		& = x&\mbox{, A5a}\\
	\end{align*}
\end{itemize}
}

\end{Q}
%
% \begin{Q}
% 	L'approximation I(V) idéalisée avec seulement la tension de seuil est-elle une bonne approximation ?
% 	\label{Q:1}
% 	\reponse{Oui}%R
% \end{Q}


% \begin{figure}[H]
% 	\begin{center}
% 		\begin{circuitikz}\draw
% 			(0,0) node[anchor=east] {A} to [short,i>^=$I$] (1.5,0)
% 			(0,0) to [sDo, v=$V$] (2.5,0) node [anchor=west]{K}
% 		;\end{circuitikz}
% 	\end{center}
% \caption{Conventions électriques}
% \label{fig:zener_conv}
% \end{figure}

\end{document}
"
%
%%%%%%%%%%%%%%%%%%%%%%%%%%%%%%%%%%%%%%%%%%%%%%%%%%%%%%%%%%%%%%%%%%%%%%%%%%%%%%%%%%%%%%%%%%%

%Numero du TP :
\def \tpnumber {TP 1 }

\newcommand{\version}{v1.0.0}


% ########     #####      #####    ##         #########     ###     ##      ##  
% ##     ##  ##     ##  ##     ##  ##         ##           ## ##    ###     ##  
% ##     ##  ##     ##  ##     ##  ##         ##          ##   ##   ## ##   ##  
% ########   ##     ##  ##     ##  ##         ######     ##     ##  ##  ##  ##  
% ##     ##  ##     ##  ##     ##  ##         ##         #########  ##   ## ##  
% ##     ##  ##     ##  ##     ##  ##         ##         ##     ##  ##     ###  
% ########     #####      #####    #########  #########  ##     ##  ##      ##  
\newboolean{corrige}
\ifx\correction\undefined
\setboolean{corrige}{false}% pas de corrigé
\else
\setboolean{corrige}{true}%corrigé
\fi

\newboolean{fr}
\ifx\french\undefined
\setboolean{fr}{false}% pas de corrigé
\else
\setboolean{fr}{true}%corrigé
\fi

\newboolean{en}
\ifx\english\undefined
\setboolean{en}{false}% pas de corrigé
\else
\setboolean{en}{true}%corrigé
\fi

%\setboolean{corrige}{false}% pas de corrigé

\definecolor{darkblue}{rgb}{0,0,0.5}

\pdfinfo{
/Author (ULB -- BEAMS)
/Title (\tpnumber, ELEC-H-310)
/ModDate (D:\pdfdate)
}

\hypersetup{
pdftitle={\tpnumber [ELEC-H-310] Choucroute numérique},
pdfauthor={ULB -- BEAMS},
pdfsubject={}
}

\theoremstyle{definition}% questions pas en italique
\newtheorem{Q}{Question}[] % numéroter les questions [section] ou non []


%  ######     #####    ##       ##  ##       ##     ###     ##      ##  ######      #######   
% ##     ##  ##     ##  ###     ###  ###     ###    ## ##    ###     ##  ##    ##   ##     ##  
% ##         ##     ##  ## ## ## ##  ## ## ## ##   ##   ##   ## ##   ##  ##     ##  ##         
% ##         ##     ##  ##  ###  ##  ##  ###  ##  ##     ##  ##  ##  ##  ##     ##   #######   
% ##         ##     ##  ##       ##  ##       ##  #########  ##   ## ##  ##     ##         ##  
% ##     ##  ##     ##  ##       ##  ##       ##  ##     ##  ##     ###  ##    ##   ##     ##  
%  #######     #####    ##       ##  ##       ##  ##     ##  ##      ##  ######      #######   
\newcommand{\reponse}[1]{% pour intégrer une réponse : \reponse{texte} : sera inclus si \boolean{corrige}
	\ifthenelse {\boolean{corrige}} {\fr{\paragraph{Réponse :}}\en{\paragraph{Answer:}} \color{darkblue}   #1\color{black}} {}
 }

 \newcommand{\fr}[1]{
 	\ifthenelse {\boolean{fr}} {#1} {}
 }

 \newcommand{\en}[1]{
 	\ifthenelse {\boolean{en}} {#1} {}
 }

\newcommand{\addcontentslinenono}[4]{\addtocontents{#1}{\protect\contentsline{#2}{#3}{#4}{}}}


%  #######   ##########  ##    ##   ##         #########  
% ##     ##      ##       ##  ##    ##         ##         
% ##             ##        ####     ##         ##         
%  #######       ##         ##      ##         ######     
%        ##      ##         ##      ##         ##         
% ##     ##      ##         ##      ##         ##         
%  #######       ##         ##      #########  #########  
%% fancy header & foot
\pagestyle{fancy}
\lhead{[ELEC-H-310] \fr{Électronique numérique}\en{Digital Electronics}\\ \tpnumber}
\rhead{\version\\ page \thepage}
\chead{\ifthenelse{\boolean{corrige}}{\fr{Corrigé}\en{Correction}}{}}
\cfoot{}
%%

\setlength{\parskip}{0.2cm plus2mm minus1mm} %espacement entre §
\setlength{\parindent}{0pt}

% ##########  ########   ##########  ##         #########  
%     ##         ##          ##      ##         ##         
%     ##         ##          ##      ##         ##         
%     ##         ##          ##      ##         ######     
%     ##         ##          ##      ##         ##         
%     ##         ##          ##      ##         ##         
%     ##      ########       ##      #########  ######### 
\date{\vspace{-1.7cm}\version}
\title{\vspace{-2cm} \tpnumber\\ \fr{Électronique numérique [ELEC-H-310] \ifthenelse{\boolean{corrige}}{~\\Corrigé}{}}%
\en{Digital Electronics [ELEC-H-310] \ifthenelse{\boolean{corrige}}{~\\Correction}{}}%
}




\begin{document}
\fr{\selectlanguage{french}}
\en{\selectlanguage{english}}

\maketitle
\vspace*{-1cm}

\begin{Q}
	\fr{Convertir dans les autres bases utiles les nombres suivants :}
	\en{Convert the following number in the relevant bases:}
	\begin{enumerate}
	\item $(82)_{10}$
	\item $(122)_{10}$
	\item $(1001110001)_2$
	\item $(F6D)_{16}$
	\item $(B65F)_{16}$
	\item $(0.625)_{10}$
	\end{enumerate}

	\reponse{
		\begin{center}
			\begin{tabular}{|l|l|l|l|}\hline
				Base 2 & Base 8 & Base 10 & Base 16 \\ \hline
				1010010 & 122 & \textbf{82} & 52 \\ \hline
				1111010 & 172 & \textbf{122} & 7A \\ \hline
				\textbf{1001110001} & 1161 & 625 & 271 \\ \hline
				111101101101 & 7555 & 3949 & \textbf{F6D} \\ \hline
				1011011001011111 & 133137 & 46687 & \textbf{B65F} \\ \hline
				0.101 & 0.5 & \textbf{0.625} & 0.A \\ \hline
			\end{tabular}
		\end{center}
	}
\end{Q}

\begin{Q}
\fr{Effectuer l’addition suivante dans toutes les bases utiles. Vérifier les résultats en
les convertissant en base 10 :}
\en{Compute this addition in all relevant bases. Check your results by converting them in base 10.}
$$(3633)_{10} + (254)_{10}$$

\reponse{~\\
	Base 2:
	\begin{tabular}{ccccccccccccc}
		& 1 & 1 & 1 & 0 & 0 & 0 & 1 & 1 & 0 & 0 & 0 & 1 \\
		+ & &   &   &   & 1 & 1 & 1 & 1 & 1 & 1 & 1 & 0\\ \hline
		& 1 & 1 & 1 & 1 & 0 & 0 & 1 & 0 & 1 & 1 & 1 & 1 \\
	\end{tabular}

	Base 8:
	\begin{tabular}{ccccc}
		  & 7 & 0 & 6 & 1 \\
		+ &   & 3 & 7 & 6 \\ \hline
		  & 7 & 4 & 5 & 7 \\
		\end{tabular}

	Base 16:
	\begin{tabular}{cccc}
		& E & 3 & 1 \\
		+ & & F & E \\ \hline
		& F & 2 & F\\
	\end{tabular}
}
\end{Q}


\begin{Q}
\fr{Représentation des nombres négatifs}
\en{Negative numbers representation}
\begin{enumerate}
	\item \fr{Représenter $(-14)_{10}$ sur 8 bits en base 2 dans les trois modes de représentation.}
	\en{Represent $(-14)_{10}$ on 8 bits in base 2 in all three representations.}
	\reponse{
		\begin{tabular}{|c|c|c|}\hline
			SVA & C1 & C2 \\ \hline
			10001110 & 11110001 & 11110010 \\ \hline
		\end{tabular}
	}
	\item \fr{Si on utilise 4 bits, quels sont, dans les 3 modes de représentation, les plus
	petites et les plus grandes valeurs représentables ? Comment se représente la
	valeur 0 ?}
	\en{If we use 4 bits, what are the highest and lowest values possible, in all three representations?
	How is represented the value 0?}
	\reponse{
		\begin{center}
			\begin{tabular}{|c|c|c|c|}\hline
				& min & 0 & max \\ \hline
				\multirow{2}{*}{SVA} & \multirow{2}{*}{1111} & 0000 & \multirow{2}{*}{0111} \\
				& & 1000 & \\ \hline
				\multirow{2}{*}{C1} & \multirow{2}{*}{10000} & 0000 & \multirow{2}{*}{0111} \\
				& & 1111 & \\ \hline
				C2 & 1000 & 0000 & 0111 \\ \hline
			\end{tabular}
		\end{center}

		\fr{À titre de bonus, ci-suit un tableau comparatif des différents modes de représentation (sur 4 bits):}
		\en{As a bonus, here is the full table comparing the three representations.}
		\begin{center}
			\begin{tabular}{|c|c|c|c|} \hline
				Base 10 & Signé & C1 & C2 \\ \hline
				7 & 0111 & 0111 & 0111 \\ \hline
				6 & 0110 & 0110 & 0110 \\ \hline
				5 & 0101 & 0101 & 0101 \\ \hline
				4 & 0100 & 0100 & 0100 \\ \hline
				3 & 0011 & 0011 & 0011 \\ \hline
				2 & 0010 & 0010 & 0010 \\ \hline
				1 & 0001 & 0001 & 0001 \\ \hline
				\multirow{2}{*}{0} & 0000 & 0000 & \multirow{2}{*}{0000} \\
				& 1000 & 1111 & \\ \hline
				-1 & 1001 & 1110 & 1111 \\ \hline
				-2 & 1010 & 1101 & 1110 \\ \hline
				-3 & 1011 & 1100 & 1101 \\ \hline
				-4 & 1100 & 1011 & 1100 \\ \hline
				-5 & 1101 & 1010 & 1011 \\ \hline
				-6 & 1110 & 1001 & 1010 \\ \hline
				-7 & 1111 & 1000 & 1001 \\ \hline
				-8 & N/A & N/A & 1000 \\ \hline
			\end{tabular}
		\end{center}
	}
\end{enumerate}
\end{Q}



\begin{Q}
\fr{En utilisant les axiomes prouver les théorèmes :}
\en{Proove the theorems using the axioms.}

\begin{minipage}[t]{0.5\linewidth}
	\centering Axiomes :
	$$A1a : x \in B, y \in B \rightarrow x+y \in B$$
	$$A1b : x \in B, y \in B \rightarrow x \cdot y \in B$$
	$$A2a : x+0 = x$$
	$$A2b : x \cdot 1 = x$$
	$$A3a : x \cdot (y+z) = x \cdot y+x \cdot z$$
	$$A3b : x+y \cdot z = (x+y)(x+z)$$
	$$A4a : x+y = y+x$$
	$$A4b : x \cdot y = y \cdot x$$
	$$A5a : x+\overline{x} = 1$$
	$$A5b : x \cdot \overline{x} = 0$$
	$$A6 : \exists x, y \in B : x \neq y$$
\end{minipage}
\begin{minipage}[t]{0.5\linewidth}
	\centering Théorèmes :
	$$T1a : x+x=x$$
	$$T1b : x \cdot x=x$$
	$$T2a : x+1=1$$
	$$T2b : x \cdot 0=0$$
	$$T3a : x+y \cdot x=x$$
	$$T3b : x \cdot (x+y)=x$$
	$$T4 : \overline{(\overline{x})}=x$$
\end{minipage}

\reponse{
\begin{itemize}
	\item $T1a : x+x=x$
	\begin{align*}
		x + x & = (x + x) \cdot 1&\mbox{, A2b}\\
		& = (x + x) \cdot (x + \overline{x})&\mbox{, A5a}\\
		& = x + x \cdot \overline{x}&\mbox{, A3b}\\
		& = x &\mbox{, A5b}
	\end{align*}

	\item $T1b : x \cdot x=x$
	\begin{align*}
		x \cdot x & = x \cdot x  + 0&\mbox{, A2a}\\
		& = x\cdot x + x \cdot \overline{x} &\mbox{, A5b}\\
		& = x \cdot (x + \overline{x}) &\mbox{, A3a} \\
		& = x \cdot 1&\mbox{, A5a}\\
		& = x&\mbox{, A2b}
	\end{align*}

	\item $T2a : x+1=1$
	\begin{align*}
		x + 1 & = (x+1) \cdot 1&\mbox{, A2b}\\
		& = (x+1) \cdot (x+\overline{x})&\mbox{, A5a}\\
		& = x + 1 \cdot \overline{x} &\mbox{, A3b}\\
		& = x + \overline{x} &\mbox{, A2b}\\
		& = 1 &\mbox{, A5a}
	\end{align*}

	\item $T2b : x \cdot 0=0$
	\begin{align*}
		x \cdot 0 & = x \cdot 0 + 0 &\mbox{, A2a} \\
		& = x \cdot 0 + x \cdot \overline{x}&\mbox{, A5b} \\
		& = x \cdot (0 + \overline{x}) &\mbox{, A3a} \\
		& = x \cdot (\overline{x}) &\mbox{, A2a} \\
		& = 0 &\mbox{, A5b}
	\end{align*}

	\item $T3a : x+y \cdot x=x$
	\begin{align*}
		x + x \cdot y & = x \cdot 1 + x \cdot y &\mbox{, A2b}\\
		& = x \cdot (1 + y)&\mbox{, A3a}\\
		& = x \cdot 1&\mbox{, T2a} \\
		& = x &\mbox{, A2b}
	\end{align*}

	\item $T3b : x \cdot (x+y)=x$
	\begin{align*}
		x \cdot (x+y) & = (x+0) \cdot (x+y) &\mbox{, A2a}\\
		& = x + 0 \cdot y &\mbox{, A3b}\\
		& = x + 0 &\mbox{, T2b} \\
		& = x &\mbox{, A2a}
	\end{align*}

	\item $T4 : \overline{(\overline{x})}=x$
	\begin{align*}
		\overline{(\overline{x})} & = \overline{(\overline{x})} \cdot 1 &\mbox{, A2b}\\
		& = \overline{(\overline{x})} \cdot (x + \overline{x}) &\mbox{, A5a}\\
		& = \overline{(\overline{x})} \cdot x + \overline{(\overline{x})} \cdot \overline{x} &\mbox{, A3a}\\
		& = \overline{(\overline{x})} \cdot x + 0 &\mbox{, A5b}\\
		& = \overline{(\overline{x})} \cdot x + x \cdot \overline{x} &\mbox{, A5b}\\
		& = x\cdot (\overline{(\overline{x})} + \overline{x})&\mbox{, A5a}\\
		& = x&\mbox{, A5a}\\
	\end{align*}
\end{itemize}
}

\end{Q}
%
% \begin{Q}
% 	L'approximation I(V) idéalisée avec seulement la tension de seuil est-elle une bonne approximation ?
% 	\label{Q:1}
% 	\reponse{Oui}%R
% \end{Q}


% \begin{figure}[H]
% 	\begin{center}
% 		\begin{circuitikz}\draw
% 			(0,0) node[anchor=east] {A} to [short,i>^=$I$] (1.5,0)
% 			(0,0) to [sDo, v=$V$] (2.5,0) node [anchor=west]{K}
% 		;\end{circuitikz}
% 	\end{center}
% \caption{Conventions électriques}
% \label{fig:zener_conv}
% \end{figure}

\end{document}
"
%
%%%%%%%%%%%%%%%%%%%%%%%%%%%%%%%%%%%%%%%%%%%%%%%%%%%%%%%%%%%%%%%%%%%%%%%%%%%%%%%%%%%%%%%%%%%

%Numero du TP :
\def \tpnumber {TP 5 }

\newcommand{\version}{v1.0.0}


% ########     #####      #####    ##         #########     ###     ##      ##  
% ##     ##  ##     ##  ##     ##  ##         ##           ## ##    ###     ##  
% ##     ##  ##     ##  ##     ##  ##         ##          ##   ##   ## ##   ##  
% ########   ##     ##  ##     ##  ##         ######     ##     ##  ##  ##  ##  
% ##     ##  ##     ##  ##     ##  ##         ##         #########  ##   ## ##  
% ##     ##  ##     ##  ##     ##  ##         ##         ##     ##  ##     ###  
% ########     #####      #####    #########  #########  ##     ##  ##      ##  
\newboolean{corrige}
\ifx\correction\undefined
\setboolean{corrige}{false}% pas de corrigé
\else
\setboolean{corrige}{true}%corrigé
\fi

\newboolean{fr}
\ifx\french\undefined
\setboolean{fr}{false}% pas de corrigé
\else
\setboolean{fr}{true}%corrigé
\fi

\newboolean{en}
\ifx\english\undefined
\setboolean{en}{false}% pas de corrigé
\else
\setboolean{en}{true}%corrigé
\fi

%\setboolean{corrige}{false}% pas de corrigé

\definecolor{darkblue}{rgb}{0,0,0.5}

\pdfinfo{
/Author (ULB -- BEAMS)
/Title (\tpnumber, ELEC-H-310)
/ModDate (D:\pdfdate)
}

\hypersetup{
pdftitle={\tpnumber [ELEC-H-310] Choucroute numérique},
pdfauthor={ULB -- BEAMS},
pdfsubject={}
}

\theoremstyle{definition}% questions pas en italique
\newtheorem{Q}{Question}[] % numéroter les questions [section] ou non []


%  ######     #####    ##       ##  ##       ##     ###     ##      ##  ######      #######   
% ##     ##  ##     ##  ###     ###  ###     ###    ## ##    ###     ##  ##    ##   ##     ##  
% ##         ##     ##  ## ## ## ##  ## ## ## ##   ##   ##   ## ##   ##  ##     ##  ##         
% ##         ##     ##  ##  ###  ##  ##  ###  ##  ##     ##  ##  ##  ##  ##     ##   #######   
% ##         ##     ##  ##       ##  ##       ##  #########  ##   ## ##  ##     ##         ##  
% ##     ##  ##     ##  ##       ##  ##       ##  ##     ##  ##     ###  ##    ##   ##     ##  
%  #######     #####    ##       ##  ##       ##  ##     ##  ##      ##  ######      #######   
\newcommand{\reponse}[1]{% pour intégrer une réponse : \reponse{texte} : sera inclus si \boolean{corrige}
	\ifthenelse {\boolean{corrige}} {\fr{\paragraph{Réponse :}}\en{\paragraph{Answer:}} \color{darkblue}   #1\color{black}} {}
 }

 \newcommand{\fr}[1]{
 	\ifthenelse {\boolean{fr}} {#1} {}
 }

 \newcommand{\en}[1]{
 	\ifthenelse {\boolean{en}} {#1} {}
 }

\newcommand{\addcontentslinenono}[4]{\addtocontents{#1}{\protect\contentsline{#2}{#3}{#4}{}}}

\newlength{\gvs}% Gate Vertical Space
\gvs=6em
\newlength{\ghs}% Gate Horizontal Space
\ghs=10em

\newcommand\encircle[1]{%http://tex.stackexchange.com/questions/123924/indexed-letters-inside-circles
  \tikz[baseline=(X.base)] 
    \node (X) [draw, shape=circle, inner sep=0] {\strut #1};}


%  #######   ##########  ##    ##   ##         #########  
% ##     ##      ##       ##  ##    ##         ##         
% ##             ##        ####     ##         ##         
%  #######       ##         ##      ##         ######     
%        ##      ##         ##      ##         ##         
% ##     ##      ##         ##      ##         ##         
%  #######       ##         ##      #########  #########  
%% fancy header & foot
\pagestyle{fancy}
\lhead{[ELEC-H-310] \fr{Électronique numérique}\en{Digital Electronics}\\ \tpnumber}
\rhead{\version\\ page \thepage}
\chead{\ifthenelse{\boolean{corrige}}{\fr{Corrigé}\en{Correction}}{}}
\cfoot{}
%%

\setlength{\parskip}{0.2cm plus2mm minus1mm} %espacement entre §
\setlength{\parindent}{0pt}

% ##########  ########   ##########  ##         #########  
%     ##         ##          ##      ##         ##         
%     ##         ##          ##      ##         ##         
%     ##         ##          ##      ##         ######     
%     ##         ##          ##      ##         ##         
%     ##         ##          ##      ##         ##         
%     ##      ########       ##      #########  ######### 
\date{\vspace{-1.7cm}\version}
\title{\vspace{-2cm} \tpnumber\\ \fr{Électronique numérique [ELEC-H-310] \ifthenelse{\boolean{corrige}}{~\\Corrigé}{}}%
\en{Digital Electronics [ELEC-H-310] \ifthenelse{\boolean{corrige}}{~\\Correction}{}}%
}





\begin{document}
\maketitle
\vspace*{1cm}

\begin{Q}
\fr{Un système de traitement d'image fournit des points (pixels) à quatre niveaux de gris~: $ab = 00, 01, 10, 11$ (respectivement noir, foncé, clair et blanc).
On veut construire un automate qui reçoit les valeurs de $ab$ en séquence, ligne après ligne, et met à $1$ une sortie $Z$ lorsqu'il détecte la fin d'une rampe montante ou descendante complète~: $00-01-10-11$ ou $11-10-01-00$.
On fera l'hypothèse simplificatrice que deux pixels successifs n'auront jamais la même valeur.

Dresser la table de Huffman de ce problème.}

\en{An image processing system gives us pixels with four levels of grey: $ab = 00, 01, 10, 11$ (respectively black, dark, light and white).
We want to build a machine that gets those $ab$ values and sets its output $Z$ to 1 whenever it detects the end of a full ascending or descending sequence: $00-01-10-11$ or $11-10-01-00$.
We can simplify this problem by assuming that two successive pixels will never have the same value.

Build the Huffman table of this machine.}

\reponse{
	\begin{center}
		\begin{tabular}{|l|l|l|l|l|l|} \hline
			& \multicolumn{4}{c}{$ab$} & \\ \hline
			& 00 & 01 & 11 & 10 & $Z$ \\ \hline
			1 & 2 & \textbf{1} & 6 & \textbf{1} & 0 \\ \hline
			2 & \textbf{2} & 3 & 1 & 1 & 0 \\ \hline
			3 & 1 & \textbf{3} & 1 & 4 & 0 \\ \hline
			4 & 1 & 1 & 5 & \textbf{4} & 0 \\ \hline
			5 & 1 & 1 & \textbf{5} & 7 & 1 \\ \hline
			6 & 1 & 1 & \textbf{6} & 7 & 0 \\ \hline
			7 & 1 & 8 & 1 & \textbf{7} & 0 \\ \hline
			8 & 9 & \textbf{8} & 1 & 1 & 0 \\ \hline
			9 & \textbf{9} & 3 & 1 & 1 & 1 \\ \hline
		\end{tabular}
	\end{center}
	%TODO Table des conditions d'équivalence
}
\end{Q}


\begin{Q}
	\fr{Deux loutres mutantes attaquent Metropolis.
	Superman est le seul à pouvoir sauver la ville.
	Les terribles mustélidés sont cependant particulièrement résistants et Superman décide de quérir l'aide du Dr Hamilton pour les vaincre.
	Ce dernier lui donne alors une table de Huffman sur la sequence de pouvoirs à utiliser pour triompher ($Z_1$).
	Cependant, il y a aussi introduit une séquence qui rendrait les loutres immortelles et leur permettrait de cracher du feu ($Z_2$).
	La notice suivante est jointe~:
	\begin{verse}
		$a$~: vision laser~; $b$~: souffle glacé~; $Z_1$~: loutres vaincues~; $Z_2$~: loutres invulnérables.
	\end{verse}

	Que doit faire Superman pour sauver Metropolis des loutres infernales~?}

	\en{Two evil weasels are attacking Metropolis.
	Superman is the only one who can save the town.
	However, the dreadful Mustelida are tough and Superman decides to ask Dr Hamilton for help.
	The professor thus gives him a Huffman table explaining the sequence of power required to triumph.
	Deceitfully, he also includes a sequence that would turn the weasels into immortal fire breathing monsters.
	The following instructions are added:
	\begin{verse}
		$a$: laser vision; $b$: ice breath; $Z_1$: weasels defeated; $Z_2$: invulnerable weasels.
	\end{verse}

	What does Superman need to do in order to save Metropolis?}

	\begin{center}
		\begin{tabular}{|c|c|c|c|c|c|c|} \hline
			& \multicolumn{4}{c|}{$ab$} & & \\ \hline
			$Y_1Y_2$& 00 & 01 & 11 & 10 & $Z_1$ & $Z_2$\\ \hline
			1 & \textbf{1} & 2 & - & 5 & 0 & 0 \\ \hline
			2 & 1 & \textbf{2} & - & 3 & 0 & 0 \\ \hline
			3 & 1 & 4 & - & \textbf{3} & 0 & 0 \\ \hline
			4 & \textbf{4} & \textbf{4} & \textbf{4} & \textbf{4} & 1 & 0 \\ \hline
			5 & 1 & 6 & - & \textbf{5} & 0 & 0\\ \hline
			6 & 1 & \textbf{6} & - & 7 & 0 & 0 \\ \hline
			7 & \textbf{7} & \textbf{7} & \textbf{7} & \textbf{7} & 0 & 1 \\ \hline
		\end{tabular}
	\end{center}

	\reponse{
		\fr{Si Superman veut éliminer ces perfides créatures, il devra d'abord les geler à l'aide de son souffle, puis les brûler avec son regard de Kryptonien et finalement les geler à nouveau.}
		\en{If Superman aims at defeating the creatures, he will first need to freeze them using his breath, then burn them with his Kryptonian glare and finaly freeze them back.}
	}
\end{Q}

\begin{Q}
	\fr{Selon une légende créée spécialement pour ce TP, jaloux du succès du code Konami\footnote{$\uparrow \uparrow \downarrow \downarrow \leftarrow \rightarrow \leftarrow \rightarrow$ \encircle{B} \encircle{A}}, Nintendo aurait implémenté un code similaire dans ses jeux développés sur NES.
	Le «~code Nintendo~» serait cependant plus simple~: pour activer une fonctionnalité cachée du jeu, il suffirait d'appuyer sur le bouton \encircle{A}, puis d'appuyer sur \encircle{B} en maintenant \encircle{A} enfoncé et enfin d'appuyer sur \framebox{\textsc{select}} toujours en maintenant les deux premiers boutons enfoncés.
	Si l'un des boutons est relâché ou enfoncé dans le mauvais ordre, la séquence est annulée.

	Dressez le circuit logique du code permettant d'activer le bonus.}

	\en{According to a legend created for this session, envious of the Konami code\footnote{$\uparrow \uparrow \downarrow \downarrow \leftarrow \rightarrow \leftarrow \rightarrow$ \encircle{B} \encircle{A}} success, Nintendo would have implemented a similar feature in his NES games.
	The ``Nintendo code" is simpler though: in order to activate a hidden bonus, the player simply needs to press \encircle{A}, then \encircle{B} while maintaining \encircle{A} and finaly to press \framebox{\textsc{select}} while still holding the first two buttons down.
	If one of the buttons is released, or if they are pressed in the wrong order, the sequence is canceled.

	Build a digital circuit implementing this feature.}

	\reponse{
		\fr{Afin de pouvoir créer le circuit logique, il convient d'abord de remplir la table de Huffman décrivant la séquence de boutons, et en dériver les équations logiques.

		En considérant que la touche \framebox{select} est représentée par la variable $c$, on obtient~:}

		\en{To build the circuit, we first need to fill a Huffman table describing the code sequence, and to find the corresponding logical equations.

		If we replace the button \framebox{select} by the variable $c$, we get:}

		\begin{center}
			\begin{tabular}{|c|c|c|c|c|c|c|c|c|c|}\hline
			& \multicolumn{8}{c|}{$abc$} & \\ \hline
			$Y_1Y_2$ & 000 & 001 & 011 & 010 & 100 & 101 & 111 & 110 & Z \\ \hline
			00 & \textbf{00} & \textbf{00} & \textbf{00} & \textbf{00} & 01 & \textbf{00} & \textbf{00} & \textbf{00} & 0 \\ \hline
			01 & 00 & 00 & 00 & 00 & \textbf{01} & 00 & 00 & 11 & 0 \\ \hline
			11 & 00 & 00 & 00 & 00 & 00 & 00 & 10 & \textbf{11} & 0 \\ \hline
			01 & 00 & 00 & 00 & 00 & 00 & 00 & \textbf{10} & 00 & 1 \\ \hline
			\end{tabular}
		\end{center}

		\fr{On en déduit les k-maps suivantes~:}
		\en{Which gives us the following K-maps:}
		\begin{center}
			\askmapv{$Y_1 = abcy_1 + ab\overline{c}y_2$}{a b c $y_1$ $y_2$}{}{00000000000000000000000001010011}{%
			\color{red}\put(6.1,0.1){\dashbox{0.1}(0.8,1.8){}}%
			\color{green}\put(7.1,1.1){\dashbox{0.2}(0.8,1.8){}}%
			}

			\askmapv{$Y_2 = a\overline{bcy_1} + ab\overline{c}y_2$}{a b c $y_1$ $y_2$}{}{00000000000000001100000001010000}{%
			\color{red}\put(4.1,2.1){\dashbox{0.1}(0.8,1.8){}}%
			\color{green}\put(7.1,1.1){\dashbox{0.2}(0.8,1.8){}}%
			}

			\askmapv{$Z = y_1\overline{y_2}$}{a b c $y_1$ $y_2$}{}{00-000-000-000-000-000-000-0001-}{%
			\color{red}\put(0.1,0.1){\dashbox{0.1}(7.8,0.8){}}%
			}
		\end{center}
			\begin{circuitikz}[scale=0.8, every node/.style={scale=0.8}]
			% \begin{circuitikz}
				\draw
				(0.5\ghs,0.5\gvs) node[not port, rotate=90] (mynotb) {}
				(1\ghs,0.5\gvs) node[not port, rotate=90] (mynotc) {}
				(1.5\ghs,0.5\gvs) node[not port, rotate=90] (mynoty1) {}
				(2\ghs,0.5\gvs) node[not port, rotate=90] (mynoty2) {}
				(2.5\ghs,1.5\gvs) node[and port] (myand11) {}%bottom left
				(2.5\ghs,2.3\gvs) node[and port] (myand12) {}
				(2.5\ghs,3.8\gvs) node[and port] (myand13) {}
				(2.5\ghs,4.6\gvs) node[and port] (myand14) {}
				(2.5\ghs,5.4\gvs) node[and port] (myand15) {}%top left

				(3.5\ghs,1.9\gvs) node[and port] (myand21) {}
				(3.5\ghs,2.7\gvs) node[and port] (myand22) {}
				(3.5\ghs,3.8\gvs) node[and port] (myand23) {}
				(3.5\ghs,5.0\gvs) node[and port] (myand24) {}

				(4.2\ghs,2.3\gvs) node[or port] (myory2) {}% Bottom OR, Y2
				(4.2\ghs,4.4\gvs) node[or port] (myory1) {}% Y1

				(2.5\ghs,6.5\gvs) node[and port] (myandz) {}
				(0,7\gvs) node[shape=coordinate] (top) {}
				(0.0,0.0) node[shape=coordinate] (bottoma) {}%Bottom of line A

				(bottoma |- myand12.in 1) node[shape=coordinate] (dot11) {}
				(bottoma |- myand14.in 1) node[shape=coordinate] (dot12) {}
				($(mynotb.in)-(0.2\ghs,0)$) node[shape=coordinate] (dot21) {}
				(dot21 |- myand14.in 2) node[shape=coordinate] (dot22) {}
				(mynotb |- myand12.in 2) node[shape=coordinate] (dot31) {}
				($(mynotc.in)-(0.2\ghs,0)$) node[shape=coordinate] (dot41) {}
				(dot41 |- myand15.in 2) node[shape=coordinate] (dot42) {}
				(mynotc |- myand11.in 2) node[shape=coordinate] (dot51) {}
				(dot51 |- myand13.in 2) node[shape=coordinate] (dot52) {}
				($(mynoty1.in)-(0.2\ghs,0)$) node[shape=coordinate] (dot61) {}
				(dot61 |- myand15.in 1) node[shape=coordinate] (dot62) {}
				(dot61 |- myandz.in 2) node[shape=coordinate] (dot63) {}
				(mynoty1 |- myand11.in 1) node[shape=coordinate] (dot71) {}
				($(mynoty2.in)-(0.2\ghs,0)$) node[shape=coordinate] (dot81) {}
				(dot81 |- myand13.in 1) node[shape=coordinate] (dot82) {}
				(mynoty2 |- myandz.in 1) node[shape=coordinate] (dot91) {}


				% A
				(bottoma) to[short, -*] (dot11) to[short, -*] (dot12) {} -- (bottoma |- top)
				(bottoma) -- +(0, -0.2\gvs)
				(dot11) -- (myand12.in 1)
				(dot12) -- (myand14.in 1)

				% B
				(dot21) -- +(0,-0.4\gvs)
				(dot21) to[short, *-*] (dot22) -- (dot21 |- top)
				(dot22) -- (myand14.in 2)

				% B'
				(mynotb.out) to[short, -*] (dot31) -- (mynotb |- top)
				(mynotb.in) -- (dot21)
				(dot31) -- (myand12.in 2)

				% C
				(dot41) -- +(0,-0.4\gvs)
				(dot41) to[short, *-*] (dot42) -- (dot41 |- top)
				(dot42) -- (myand15.in 2)

				% C'
				(mynotc.out) to[short, -*] (dot51) to[short, -*] (dot52) -- (mynotc |- top)
				(mynotc.in) -- (dot41)
				(dot51) -- (myand11.in 2)
				(dot52) -- (myand13.in 2)

				% Y1
				(dot61) -- +(0,-0.4\gvs)
				(dot61) to[short, *-*] (dot62) to[short, -*] (dot63) -- (dot61 |- top)
				(dot62) -- (myand15.in 1)
				(dot63) -- (myandz.in 2)

				% Y1'
				(mynoty1.out) to[short, -*] (dot71) -- (mynoty1 |- top)
				(mynoty1.in) -- (dot61)
				(dot71) -- (myand11.in 1)

				% Y2
				(dot81) -- +(0,-0.2\gvs)
				(dot81) to[short, *-*] (dot82) -- (dot81 |- top)
				(dot82) -- (myand13.in 1)

				% Y2'
				(mynoty2.out) to[short, -*] (dot91) -- (mynoty2 |- top)
				(mynoty2.in) -- (dot81)
				(dot91) -- (myandz.in 1)
				
				% AND21
				(myand11.out) -| (myand21.in 2)
				(myand12.out) -| (myand21.in 1)

				% AND22
				(myand23.in 2) to[short, *-] (myand22.in 1)
				(myand14.out) to[short, *-] (myand14.out |- myand23.in 1) to[short, *-] (myand14.out |- myand22.in 2) -- (myand22.in 2)

				% AND23
				(myand14.out |- myand23.in 1) -- (myand23.in 1)
				(myand13.out) -| (myand23.in 2)

				% AND24
				(myand14.out) -| (myand24.in 2)
				(myand15.out) -| (myand24.in 1)

				% OR Y2
				(myand21.out) -| (myory2.in 2)
				(myand22.out) -| (myory2.in 1)
				(myory2.out) |- ($(dot81)-(0,0.2\gvs)$)

				% OR Y1
				(myand23.out) -| (myory1.in 2)
				(myand24.out) -| (myory1.in 1)
				(myory1.out) -- +(0.2\gvs,0) |- ($(dot61)-(0,0.4\gvs)$)

				% Z
				(myandz.out) -- +(0.2\ghs,0)
				(myandz.out) node[anchor=south west] {\Large $Z$}
		
				% Labels
				($(bottoma)+(0,0.15\gvs)$) node[anchor=north east] {\Large $a$}
				(dot21) node[anchor=north east] {\Large $b$}
				(dot41) node[anchor=north east] {\Large $c$}
				(dot61) node[anchor=north east] {\Large $y_1$}
				(dot81) node[anchor=north east] {\Large $y_2$}
				(myory1.out) node[anchor=south west] {\Large $Y_1$}
				($(myory2.out)-(0.1\ghs,0)$) node[anchor=south west] {\Large $Y_2$}
				;
			\end{circuitikz}
	}

\end{Q}	


\vfill
\footnotesize{
\en{Found an error? Let us know: \url{https://github.com/BEAMS-EE/ELECH310/issues}}
\fr{Une erreur ? Dites-le-nous : \url{https://github.com/BEAMS-EE/ELECH310/issues}}
}

\end{document}
