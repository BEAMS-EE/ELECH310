\documentclass[11pt,a4paper,dvipsnames,]{article}
\usepackage[utf8]{inputenc}
\usepackage[T1]{fontenc}
\usepackage{amsthm} %numéroter les questions
\usepackage[frenchb]{babel}
\usepackage{datetime}
\usepackage{xspace} % typographie IN
\usepackage{hyperref}% hyperliens
\usepackage[all]{hypcap} %lien pointe en haut des figures
\usepackage[french]{varioref} %voir x p y
\usepackage{fancyhdr}% en têtes
%\input cyracc.def
\usepackage{graphicx} %include pictures
\usepackage{pgfplots}

\usepackage{tikz}
\usetikzlibrary{calc,arrows,automata}
\usetikzlibrary{babel}
% \usetikzlibrary{circuits.logic.US,circuits.logic.IEC}
\usepackage{circuitikz}
% \usepackage{gnuplottex}
\usepackage{float}
\usepackage{ifthen}

\usepackage[top=1.3 in, bottom=1.3 in, left=1.3 in, right=1.3 in]{geometry} % Yeah, that's bad to play with margins
\usepackage[]{pdfpages}

\usepackage[]{attachfile}

\usepackage{amsmath}

\usepackage{askmaps}
% \usepackage[usenames,dvipsnames,svgnames,table]{xcolor}
\usepackage[]{xcolor}
\usepackage{colortbl}

\usepackage{multirow}

\newdateformat{mydate}{2016--2017}%hack pour remplacer \THEYEAR

%cyr
\newcommand\textcyr[1]{{\fontencoding{OT2}\fontfamily{wncyr}\selectfont #1}}


\newboolean{corrige}
\ifx\correction\undefined
\setboolean{corrige}{false}% pas de corrigé
\else
\setboolean{corrige}{true}%corrigé
\fi

%\setboolean{corrige}{false}% pas de corrigé

\newboolean{annexes}
\setboolean{annexes}{true}%annexes
%\setboolean{annexes}{false}% pas de annexes

\definecolor{darkblue}{rgb}{0,0,0.5}

\newboolean{mos}
%\setboolean{mos}{true}%annexes
\setboolean{mos}{false}% pas de annexes

\usepackage{aeguill} %guillemets

%% fancy header & foot
\pagestyle{fancy}
\lhead{[ELEC-H-310] Électronique numérique\\ TP 4}
\rhead{\mydate\today\\ page \thepage}
\chead{\ifthenelse{\boolean{corrige}}{Corrigé}{}}
\cfoot{}
%%

\pdfinfo{
/Author (Quentin Delhaye, Ken Hasselmann, ULB -- BEAMS)
/Title (TP 4, ELEC-H-310)
/ModDate (D:\pdfdate)
}

\hypersetup{
pdftitle={TP 4 [ELEC-H-310] Choucroute numérique},
pdfauthor={Quentin Delhaye, Ken Hasselmann, ©2014 ULB -- BEAMS  },
pdfsubject={}
}

\theoremstyle{definition}% questions pas en italique
\newtheorem{Q}{Question}[] % numéroter les questions [section] ou non []

\newcommand{\reponse}[1]{% pour intégrer une réponse : \reponse{texte} : sera inclus si \boolean{corrige}
	\ifthenelse {\boolean{corrige}} {\paragraph{Réponse :} \color{darkblue}   #1\color{black}} {}
 }

\newcommand{\addcontentslinenono}[4]{\addtocontents{#1}{\protect\contentsline{#2}{#3}{#4}{}}}

\date{\vspace{-1.7cm}\mydate\today}
\title{\vspace{-2cm} TP 4\\ Électronique numérique [ELEC-H-310] \ifthenelse{\boolean{corrige}}{~\\Corrigé}{}}

\setlength{\parskip}{0.2cm plus2mm minus1mm} %espacement entre §
\setlength{\parindent}{0pt}

% \renewcommand{\theenumi}{\alph{enumi}} % Change the 'enumerate' format to use letters.
\usepackage{enumitem}
\setlist[enumerate]{label=\alph*)}% If you want only the x-th level to use this format, use '[enumerate,x]'

\newlength{\gvs}% Gate Vertical Space
\gvs=6em
\newlength{\ghs}% Gate Horizontal Space
\ghs=10em

\begin{document}
\pagestyle{empty}
\maketitle
\vspace*{1cm}

\begin{Q}
	Quelles sont les équations logiques correspondant à ce schéma~?
	Dresser la table de Huffman et en déduire les graphes d'état.
	\begin{enumerate}
		\item ~\\
		\begin{center}
			\begin{circuitikz}[scale=0.7, every node/.style={scale=0.7}]
			% \begin{circuitikz}
				\draw
				(0\ghs,0.5\gvs) node[not port] (mynot1) {}
				(0\ghs,1.5\gvs) node[not port] (mynot2) {}
				(0\ghs,2.5\gvs) node[not port] (mynot3) {}
				(\ghs,0\gvs) node[and port] (myand1) {}
				(\ghs,1\gvs) node[and port] (myand2) {}
				(\ghs,2\gvs) node[and port] (myand3) {}
				(\ghs,3\gvs) node[and port] (myand4) {}
				(2\ghs,2.5\gvs) node[or port] (myor) {}

				% NOT gates outputs
				let
				\p1=(mynot1.out),
				\p2=(mynot2.out),
				\p3=(mynot3.out),
				\p4=(myand4.in 2),
				\p5=(myand4.in 1),
				\p6=(myand4.out)
				in

				(mynot1.out) -- (0.5\ghs,\y1) |- (myand3.in 2)
				(mynot2.out) -- (0.45\ghs,\y2) |- ($(\x4,\y6)+(0.6em,0)$)% Size of a AND in/out pin: 0.6em.
				(mynot3.out) -- (0.35\ghs,\y3) |- (myand2.in 1)

				% NOT gates inputs
				let
				\p1=(mynot1.in),
				\p2=(mynot2.in),
				\p3=(mynot3.in),
				\p4=(myand1.in 2),
				\p5=(mynot2.in),
				\p6=(mynot3.in)
				in

				% mynot1 input
				(mynot1.in) |- (myand2.in 2)
				(mynot1.in) -- +(-0.1\ghs,0)% Draw a 0.1\ghs line going left from the input of mynot1 (aesthetics)
				(mynot1.in) to[short,-*] (\x1,\y4) to (myand1.in 2)
				(myand2.out) -- +(0.3\ghs,0) |- ($(\x1,\y4)-(0,0.5\gvs)$) to (mynot1.in)

				% mynot2 input
				(mynot2.in) -- +(-0.2\ghs,0)
				(mynot2.in) |- +(0.7\ghs,0.45\gvs) |- (myand3.in 1)

				% mynot3 input
				(mynot3.in) -- +(-0.2\ghs,0)
				(mynot3.in) |- (myand4.in 1)

				% AND gates outputs
				let
				\p1=(myand1.out),
				\p2=(myor.out),
				\p3=(mynot1.in),
				\p4=(myand1.in 2),
				\p5=(myand2.out)
				in

				(myand4.out) |- (myor.in 1)
				(myand3.out) |- (myor.in 2)
				(myand1.out) -- (\x2,\y1) node[anchor=south west] {\Large $Z$} -- +(0.2\ghs,0)

				% Other
				(myand4.in 2) -- +(-0.05\ghs,0) |- (myand1.in 1)
				(myor.out) |- ($(\x3,\y4)-(0.2\ghs,0.7\gvs)$) |- ($(\x3,\y5)-(0.2\ghs,0)$) node[anchor=south west] {\Large $y_1$} -- ($(\x4,\y5)-(0.05\ghs,0)$) to[short, *-] ($(\x3,\y5)-(0.2\ghs,0)$) -- ($(\x4,\y5)+(0.5em,0)$)
				;
				% Labels
				\draw (mynot3.in) node[anchor=south east] {\Large $a$};
				\draw (mynot2.in) node[anchor=south east] {\Large $b$};
				\draw (mynot1.in) node[anchor=north east] {\Large $y_2$};
				\draw (myand2.out) node[anchor=south west] {\Large $Y_2$};
				\draw (myor.out) node[anchor=south west] {\Large $Y_1$};

				\fill (mynot1.in) circle [radius=2pt];
				\fill (mynot2.in) circle [radius=2pt];
				\fill (mynot3.in) circle [radius=2pt];
			\end{circuitikz}
		\end{center}
		\reponse{
			\begin{align*}
				Y_1 & = a\overline{b}y_1 + b\overline{y_2} \\
				Y_2 & = \overline{a}y_1y_2 \\
				Z & = y_1y_2
			\end{align*}

			\begin{center}
				\begin{tabular}{|c|c|c|c|c|c|} \hline
				& \multicolumn{4}{c|}{$ab$} & \\ \hline
				$Y_1Y_2$ & 00 & 01 & 11 & 10 & $Z$ \\ \hline
				00 & 00 & 10 & 10 & 00 & 0 \\ \hline
				01 & 00 & 00 & 00 & 00 & 0 \\ \hline
				11 & 01 & 01 & 00 & 10 & 1 \\ \hline
				10 & 00 & 10 & 10 & 10 & 0 \\ \hline
				\end{tabular}
			\end{center}

			\begin{center}
			\begin{tikzpicture}[->,>=stealth',shorten >=1pt,auto,node distance=4.5cm,semithick,align=center]
			  \tikzstyle{every state}=[fill=white,text=black]

			  \node[state]         (A)              {$00$\\ Z=0};
			  \node[state]         (B) [right of=A] {$01$\\ Z=0};
			  \node[state]         (D) [below of=A] {$10$\\ Z=0};
			  \node[state]         (C) [below of=B] {$11$\\ Z=1};

			  \path (A) edge 			  node {01,11} (D)
						edge [loop left]  node {00,10} (A)
			        (B) edge [bend right] node {} (A)
			        (C) edge [bend left]  node {10} (D)
			        	edge [bend right] node {00, 01} (B)
			        	edge 			  node {11} (A)
			        (D) edge [loop left]  node {11,01, 10} (D)
			            edge [bend left]  node {00} (A);

			   \draw node[align=center] at (2.25,1) {00, 01, 11, 10};
			\end{tikzpicture}
			\end{center}
		}

		\item ~\\
		\begin{center}
			\begin{circuitikz}[scale=0.7, every node/.style={scale=0.7}]
			% \begin{circuitikz}
				\draw
				(0.5\ghs,0.5\gvs) node[not port, rotate=90] (mynot1) {}
				(1\ghs,0.5\gvs) node[not port, rotate=90] (mynot2) {}
				(1.5\ghs,0.5\gvs) node[not port, rotate=90] (mynot3) {}
				(2.3\ghs,2\gvs) node[and port] (myand1) {}
				(2.3\ghs,2.8\gvs) node[and port] (myand2) {}
				(2.3\ghs,3.6\gvs) node[and port] (myand3) {}
				(2.3\ghs,4.4\gvs) node[and port] (myand4) {}
				(2.3\ghs,5.2\gvs) node[and port] (myand5) {}
				(0,5.8\gvs) node[shape=coordinate] (top) {}
		
				let \p1=(myand3.out), \p2=(myand5.out) in
				(3.1\ghs,\y1) node[or port] (myor1) {}
				(3.1\ghs,\y2) node[or port] (myor2) {}
		
		
				% 'dot21' means first dot on second line from the left.
				($(mynot1.in)-(0.2\ghs,0)$) node[shape=coordinate] (dot11) {}
				(dot11 |- myand4.in 1) node[shape=coordinate] (dot12) {}
				(mynot1.out |- myand3.in 1) node[shape=coordinate] (dot21) {}
				(dot21 |- myand5.in 1) node[shape=coordinate] (dot22) {}
				($(mynot2.in)-(0.2\ghs,0)$) node[shape=coordinate] (dot31) {}
				(dot31 |- myand2.in 1) node[shape=coordinate] (dot32) {}
				(dot31 |- myand3.in 2) node[shape=coordinate] (dot33) {}
				($(dot31 |- myor2.in 1)+(0,0.3\gvs)$) node[shape=coordinate] (dot34) {}
				(dot31 |- myand1.in 1) node[shape=coordinate] (dot31bis) {}
				(mynot2.out |- myand1.in 1) node[shape=coordinate] (dot41) {}
				($(mynot3.in)-(0.2\ghs,0)$) node[shape=coordinate] (dot51) {}
				(dot51 |- myand1.in 2) node[shape=coordinate] (dot52) {}
				(dot51 |- myand2.in 2) node[shape=coordinate] (dot53) {}
				(dot51 |- myand5.in 2) node[shape=coordinate] (dot54) {}
				(mynot3.out |- myand4.in 2) node[shape=coordinate] (dot61) {}
				(dot61 |- myand1.in 2) node[shape=coordinate] (dot61bis) {}
				($(myand5.out)+(0.2\ghs,0)$) node[shape=coordinate] (dotandor) {}% Between the AND and OR gates.
				

				% NOT 1
				(mynot1.in) to [short, -*] (dot11)
				(dot11) -- +(0,-0.4\gvs)
				% Note on the syntax: '(dot11 |- top)' yields a pair of coordinates '(dot11.x, top.y)'. Neet.
				% Obviously, '-|' does the opposite: '(top.x, dot11.y)'.
				(dot11) to[short, -*] (dot12) to (dot11 |- top)
				(dot12) -- (myand4.in 1)
				(mynot1.out) to[short, -*] (dot21) to[short, -*] (dot22) to (dot22 |- top)
				(dot22) -- (myand5.in 1)
		
				% NOT 2
				(mynot2.in) to[short, -*] (dot31)
				(dot31) to[short, -*] (dot32) to[short, -*] (dot33) to[short, -*] (dot34) -- (dot31 |- top)
				(mynot2.out) -- (dot41) -- (mynot2.out |- top)
		
				% NOT 3
				(mynot3.in) to[short, -*] (dot51) -- (dot52) to[short, -*] (dot53) to[short, -*] (dot54) -- (dot51 |- top)
				(mynot3.out) to[short, -*] (dot61bis) -- (mynot3.out |- top)
		
				% ANDs
				% (dot52) -- (myand1.in 2)
				(dot61bis) -- (myand1.in 2)
				% (dot41) -- (myand1.in 1)
				(dot31bis) to [short, *-] (myand1.in 1)
				(dot53) -- (myand2.in 2)
				(dot32) -- (myand2.in 1)
				(dot33) -- (myand3.in 2)
				(dot21) -- (myand3.in 1)
				(dot61) -- (myand4.in 2)
				(dot12) -- (myand4.in 1)
				(dot54) -- (myand5.in 2)
				(dot22) -- (myand5.in 1)
				(dot34) -| (myor2.in 1)
				(myand5.out) to[short, -*] (dotandor) -- ($(myor2.in 1 |- myor2.out)+(1.2em,0)$)% 1.2 em: distance central pin-gate
				(myand4.out) -| (myor2.in 2)
				(dotandor) |- (myor1.in 1)
				(myand3.out) -- ($(myor1.in 1 |- myor1.out)+(1.2em,0)$)
				(myand2.out) -| (myor1.in 2)
				(myand1.out) -| ($(myor2.out |- myand1.out)+(0.5\ghs,0)$) node[anchor=south west] {\Large $Z$}
		
				% ORs
				(myor1.out) |- ($(dot51)-(0,0.2\gvs)$) -- (dot51)
				(myor2.out) -- +(0.2\ghs,0) |- ($(dot31)-(0,0.4\gvs)$) -- (dot31)
		
				% Labels
				(dot11) node[anchor=north east] {\Large $a$}
				(dot31) node[anchor=north east] {\Large $y_1$}
				(dot51) node[anchor=north east] {\Large $y_2$}
				(myor2.out) node[anchor=south west] {\Large $Y_1$}
				(myor1.out) node[anchor=south west] {\Large $Y_2$}
				;
			\end{circuitikz}
		\end{center}
		\reponse{
			\begin{align*}
				Y_1 & = y_1 + \overline{a}y_2 + a\overline{y_2}\\
				Y_2 & = \overline{a}y_2 + \overline{a}y_1 + y_1y_2\\
				Z & = y_1\overline{y_2}
			\end{align*}

			\begin{center}
				\begin{tabular}{|c|c|c|c|c|c|} \hline
				& \multicolumn{4}{c|}{$ab$} & \\ \hline
				$Y_1Y_2$ & 00 & 01 & 11 & 10 & $Z$ \\ \hline
				00 & 00 & 00 & 10 & 10 & 0 \\ \hline
				01 & 11 & 11 & 00 & 00 & 0 \\ \hline
				11 & 11 & 11 & 11 & 11 & 0 \\ \hline
				10 & 11 & 11 & 10 & 10 & 1 \\ \hline
				\end{tabular}
			\end{center}

			\begin{center}
			\begin{tikzpicture}[->,>=stealth',shorten >=1pt,auto,node distance=4.5cm,semithick,align=center]
			  \tikzstyle{every state}=[fill=white,text=black]

			  \node[state]         (A)              {$00$\\ Z=0};
			  \node[state]         (B) [right of=A] {$01$\\ Z=0};
			  \node[state]         (D) [below of=A] {$10$\\ Z=1};
			  \node[state]         (C) [below of=B] {$11$\\ Z=0};

			  \path (A) edge [bend right] node {10,11} (D)
						edge [loop left]  node {00,01} (A)
			        (B) edge [bend right] node {11,10} (A)
			        	edge [bend left]  node {00, 01} (C)
			        (C) edge [loop right]  node {00, 01, 11, 10} (D)
			        (D) edge [loop left]  node {00, 01} (D)
			            edge [bend right]  node {11, 10} (C);
			\end{tikzpicture}
			\end{center}
		}
	\end{enumerate}
\end{Q}

\begin{Q}
	Un ouvre-porte est muni d'un code secret commandé par deux boutons poussoirs $a$ et $b$.
	On considère que la variable liée à chaque bouton vaut 1 quand le bouton est enfoncé, et 0 quand il est relâché.
	$Z_1$ est la sortie qui, si elle vaut 1, commande l'ouvre-porte.
	Il n'est pas nécessaire que cette sortie reste à 1 pour qu'on puisse ouvrir la porte~; on peut imaginer qu'à l'instant où l'ouvre-porte est activé (lorsque le dernier bouton de la combinaison est enfoncé), la porte peut être librement ouverte pendant une courte période de temps.
	Le code consiste à enfoncer et relâcher $a$ deux fois de suite, puis $b$, puis encore $a$.
	Toute fausse manœuvre conduit à mettre la sortie $Z_2$ (alarme) à 1.
	Cette dernière reste activée peu importe l'état des variables d'entrée et nécessite donc qu'on réinitialise le système.
	Une fois le code entré, relâcher les deux boutons renvoie le système dans son état initial.

	Établir le graphe d'état et la table de Huffman correspondant à ce problème.
	\reponse{
		\begin{center}
			\begin{tabular}{|l|l|l|l|l|l|l|} \hline
				& \multicolumn{4}{c}{$ab$} & & \\ \hline
				& 00 & 01 & 11 & 10 & $Z_1$ & $Z_2$ \\ \hline
				1 & \textbf{1} & 9 & 9 & 2 & 0 & 0 \\ \hline
				2 & 3 & 9 & 9 & \textbf{2} & 0 & 0 \\ \hline
				3 & \textbf{3} & 9 & 9 & 4 & 0 & 0 \\ \hline
				4 & 5 & 9 & 9 & \textbf{4} & 0 & 0 \\ \hline
				5 & \textbf{5} & 6 & 9 & 9 & 0 & 0 \\ \hline
				6 & 7 & \textbf{6} & 9 & 9 & 0 & 0 \\ \hline
				7 & \textbf{7} & 9 & 9 & 8 & 0 & 0 \\ \hline
				8 & 1 & 9 & 9 & \textbf{8} & 1 & 0 \\ \hline
				9 & \textbf{9} & \textbf{9} & \textbf{9} & \textbf{9} & 0 & 1 \\ \hline
			\end{tabular}
		\end{center}

		\begin{center}
		\begin{tikzpicture}[->,>=stealth',shorten >=1pt,auto,node distance=4.5cm,semithick,align=center]
		  \tikzstyle{every state}=[fill=white,text=black]

		  \node[state]         (A)              {$1$};
		  \node[state]         (B) [right of=A] {$2$};
		  \node[state]         (C) [right of=B] {$3$};
		  \node[state]         (D) [below of=C] {$4$};
		  \node[state]         (E) [below of=D] {$5$};
		  \node[state]         (F) [left of=E] 	{$6$};
		  \node[state]         (G) [left of=F] 	{$7$};
		  \node[state]         (H) [above of=G] {$8$};
		  \node[state]         (I) [right of=H] {$9$};

		  \path	(A)	edge	[loop left]		node	{00}		(A)
		  			edge	[bend left]		node	{$ab = $10}		(B)
		  			edge					node	{11, 10}	(I)
		  		(B)	edge	[loop above]	node	{10}		(B)
		  			edge	[bend left]		node	{00}		(C)
		  			edge					node	{11, 01}	(I)
		  		(C)	edge	[loop right]	node	{00}		(C)
		  			edge	[bend left]		node	{10}		(D)
		  			edge					node	{11, 01}	(I)
		  		(D)	edge	[loop right]	node	{10}		(D)
		  			edge	[bend left]		node	{00}		(E)
		  			edge					node	{11,01}		(I)
		  		(E)	edge	[loop right]	node	{00}		(E)
		  			edge	[bend left]		node	{01}		(F)
		  			edge					node	{11, 10}	(I)
		  		(F)	edge	[loop below]	node	{01}		(F)
		  			edge	[bend left]		node	{00}		(G)
		  			edge					node	{11, 10}	(I)
		  		(G)	edge	[loop left]		node	{00}		(G)
		  			edge	[bend left]		node	{10}		(H)
		  			edge					node	{11, 01}	(I)
		  		(H)	edge	[loop left]		node	{10}		(H)
		  			edge	[bend left]		node	{00}		(A)
		  			edge					node	{11, 01}	(I);
		  % \path (A) edge 			  node {01,11} (D)
				% 	edge [loop left]  node {00,10} (A)
		  %       (B) edge [bend right] node {} (A)
		  %       (C) edge [bend left]  node {10} (D)
		  %       	edge [bend right] node {00, 01} (B)
		  %       	edge 			  node {11} (A)
		  %       (D) edge [loop left]  node {11,01, 10} (D)
		  %           edge [bend left]  node {00} (A);

		  %  \draw node[align=center] at (2.25,1) {00, 01, 11, 10};
		\end{tikzpicture}
		\end{center}

		Notez que lorsque l'on se trouve sur l'état \texttt{9}, l'état est stable pour toute les entrées. Afin de ne pas surcharger le graphe, les rétroactions sur l'état \texttt{9} ont donc été omises.
	}
\end{Q}

% \begin{Q}
% 	Construire la table des conditions d'équivalence pour la table de Huffman suivante.
% 	\begin{center}
% 		\begin{tabular}{|l|l|l|l|l|l|} \hline
% 			& \multicolumn{4}{c}{$ab$} & \\ \hline
% 			1 & \textbf{1} & 2 & 3 & 10 & 00 \\ \hline
% 			2 & 7 & \textbf{2} & 9 & 4 & 11 \\ \hline
% 			3 & 7 & 6 & \textbf{3} & 4 & 10 \\ \hline
% 			4 & 7 & 2 & 5 & \textbf{4} & 01 \\ \hline
% 			5 & 11 & 8 & \textbf{5} & 12 & 10 \\ \hline
% 			6 & 11 & \textbf{6} & 5 & 10 & 11 \\ \hline
% 			7 & \textbf{7} & 8 & 5 & 4 & 00 \\ \hline
% 			8 & 1 & \textbf{8} & 3 & 4 & 11 \\ \hline
% 			9 & 7 & 8 & \textbf{9} & 12 & 10 \\ \hline
% 			10 & 11 & 6 & 9 & \textbf{10} & 01 \\ \hline
% 			11 & \textbf{11} & 8 & 9 & 10 & 00 \\ \hline
% 			12 & 7 & 2 & 5 & \textbf{12} & 11 \\ \hline
% 		\end{tabular}
% 	\end{center}
% 	\reponse{
% 		\begin{center}
% 			\begin{tabular}{|c|c|c|c|c|c|c|c|c|c|c|c|} \hline%\cellcolor{green!25}
% 				2 & x &  &  &  &  &  &  &  &  &  & \\ \hline
% 				3 & x & x &  &  &  &  &  &  &  &  & \\ \hline
% 				4 & x & x & x &  &  &  &  &  &  &  & \\ \hline
% 				\multirow{3}{*}{5} & \multirow{3}{*}{x} & \multirow{3}{*}{x} & \cellcolor{red!25}7-11; & \multirow{3}{*}{x} &  &  &  &  &  &  & \\
% 				 &  &  & \cellcolor{red!25}6-8; &  &  &  &  &  &  &  & \\
% 				 &  &  & \cellcolor{red!25}4-12 &  &  &  &  &  &  &  & \\ \hline
% 				% 5 & x & x & \cellcolor{red!25}7-11;\\6-8; & x &  &  &  &  &  &  & \\ \hline
% 				\multirow{3}{*}{6} & \multirow{3}{*}{x} & \cellcolor{green!25}7-11; & \multirow{3}{*}{x} & \multirow{3}{*}{x} & \multirow{3}{*}{x} &  &  &  &  &  & \\
% 				 &  & \cellcolor{green!25}5-9; &  &  &  &  &  &  &  &  & \\
% 				 &  & \cellcolor{green!25}4-10 &  &  &  &  &  &  &  &  & \\ \hline
% 				\multirow{3}{*}{7} & \cellcolor{red!25}2-8; & \multirow{3}{*}{x} & \multirow{3}{*}{x} & \multirow{3}{*}{x} & \multirow{3}{*}{x} & \multirow{3}{*}{x} &  &  &  &  & \\
% 				 & \cellcolor{red!25}4-10 &  & & &  &  &  &  &  &  & \\
% 				 & \cellcolor{red!25} & & & &  &  &  &  &  &  & \\ \hline
% 				\multirow{3}{*}{8} & \multirow{3}{*}{x}  & \cellcolor{red!25}1-7; & \multirow{3}{*}{x}  & \multirow{3}{*}{x}  & \multirow{3}{*}{x}  & \cellcolor{red!25}7-11; & \multirow{3}{*}{x}  &  &  &  & \\
% 				& x & \cellcolor{red!25}3-9 & & & & \cellcolor{red!25}3-5; & &  &  &  & \\
% 				& x & \cellcolor{red!25} & & & & \cellcolor{red!25}4-10 & &  &  &  & \\ \hline
% 				\multirow{2}{*}{9} & \multirow{2}{*}{x} & \multirow{2}{*}{x} & \cellcolor{red!25}6-8; & \multirow{2}{*}{x} & \cellcolor{green!25}7-11; & \multirow{2}{*}{x} & \multirow{2}{*}{x} & \multirow{2}{*}{x} &  &  & \\
% 				 &  &  & \cellcolor{red!25}4-12 &  & \cellcolor{green!25} &  &  &  &  &  & \\ \hline
% 				10 & x & x & x & \cellcolor{green!25}7-11; & x & x & x & x & x &  & \\ \hline
% 				\multirow{2}{*}{11} & \cellcolor{red!25}2-8; & \multirow{2}{*}{x} & \multirow{2}{*}{x} & \multirow{2}{*}{x} & \multirow{2}{*}{x} & \multirow{2}{*}{x} & \cellcolor{green!25}5-9; & \multirow{2}{*}{x} & \multirow{2}{*}{x} & \multirow{2}{*}{x} & \\
% 				 & \cellcolor{red!25}3-9 &  &  &  &  &  & \cellcolor{green!25}4-10; &  &  &  & \\ \hline
% 				12 & \multirow{4}{*}{x} & \cellcolor{red!25}5-9; & \multirow{4}{*}{x} & \multirow{4}{*}{x} & \multirow{4}{*}{x} & \cellcolor{red!25}11-7; & \multirow{4}{*}{x} & \cellcolor{red!25}1-7; & \multirow{4}{*}{x} & \multirow{4}{*}{x} & \multirow{4}{*}{x} \\
% 				 &  & \cellcolor{red!25}4-12 &  &  &  & \cellcolor{red!25}10-12; &  & \cellcolor{red!25}2-8; &  &  &  \\
% 				 &  & \cellcolor{red!25} &  &  &  & \cellcolor{red!25}2-6 &  & \cellcolor{red!25}3-5; &  &  &  \\
% 				 &  & \cellcolor{red!25} &  &  &  & \cellcolor{red!25} &  & \cellcolor{red!25}4-12 &  &  &  \\ \hline
% 				 & 1 & 2 & 3 & 4 & 5 & 6 & 7 & 8 & 9 & 10 & 11 \\ \hline
% 			\end{tabular}
% 		\end{center}
% 	}
% \end{Q}
\begin{Q}
Utiliser une K-map pour simplifier l'équation suivante :\\
$f(a,b,c,d,e)=\sum_{m}(0,2,5,7,8,9,10,11,13,23,26,27,29)+\sum_{d}(3,12,15,18,19,21,22,31)$
	\reponse{
		\begin{center}
			\askmapv{$F = \overline{ace} + ce + \overline{a}b\overline{c} + \overline{c}d$}{abcde}{}{101-01011111-10-00--0--10011010-}{%
			\color{red}\put(0.1,0.1){\dashbox{0.1}(0.8,0.8){}}%
			\color{red}\put(3.1,0.1){\dashbox{0.1}(0.8,0.8){}}%
			\color{red}\put(0.1,3.1){\dashbox{0.1}(0.8,0.8){}}%
			\color{red}\put(3.1,3.1){\dashbox{0.1}(0.8,0.8){}}%
			\color{green}\put(1.1,1.1){\dashbox{0.2}(1.8,1.8){}}%
			\color{green}\put(5.1,1.1){\dashbox{0.2}(1.8,1.8){}}%
			\color{blue}\put(3.2,0.2){\dashbox{0.3}(0.6,3.6){}}%
			\color{orange}\put(0.3,0.3){\dashbox{0.2}(0.4,1.6){}}%
			\color{orange}\put(3.3,0.3){\dashbox{0.2}(0.4,1.6){}}%
			\color{orange}\put(4.3,0.3){\dashbox{0.2}(0.4,1.6){}}%
			\color{orange}\put(7.3,0.3){\dashbox{0.2}(0.4,1.6){}}%
			}
		\end{center}
	}

\end{Q}

\end{document}
